\section{Discussion}

% \subsection{Overlapping Rules} Already discussed in previous sections

This section discusses some insights that we gained from our work and contrasts
the scoping mechanisms we have employed with those in DK's algorithm.
We also discuss a way to improve the precision of their scope tracking.
Furthermore we discuss and sketch an extension of our algorithm with
an elaboration to a target calculus.

\begin{comment}
\subsection{Implementation}
Anything to say about the implementation? Do we have one?
\end{comment}

\subsection{Contrasting our scoping mechanisms with DK's scoping}\label{sec:discussion:scoping}

A nice feature of our worklists is that, simply by interleaving variable declarations and
judgment chains, they make the scope of variables
precise.  DK's algorithm deals with garbage collecting variables in a
different way: through type variable or existential variable
markers (as discussed in Section~\ref{ssec:DK_Algorithm}).  Despite
the sophistication employed in DK's algorithm to keep scoping precise,
there is still a chance that unused existential variables leak their
scope to an output context and accumulate indefinitely.
For example, the derivation of the judgment $(\lam x)~() \Lto 1$ is as follows
$$
\inferrule*
{
    \inferrule*
    {
        \inferrule*
        {
            \inferrule*
            {\ldots x \To \al \ldots \\ \ldots \al \le \bt \ldots}
            {\Gm, \al, \bt, x:\al \vdash x \Lto \bt \dashv \Gm_1, x:\al}
        }
        {\Gm \vdash \lam x \To \al \to \bt \dashv \Gm_1}
        \\
        \inferrule*
        {\ldots () \Lto \al \ldots}
        {\Gm_1 \vdash \appInf{\al \to \al}{()}{\al} \dashv \Gm_2}
    }
    {\Gm \vdash (\lam x)~() \To \al \dashv \Gm_2}
    \\
    \inferrule*
    {~}
    {\Gm_2 \vdash 1 \le 1 \dashv \Gm_2}
}
{\Gm \vdash (\lam x)~() \Lto 1 \dashv \Gm, \al = 1, \bt = \al}
$$
where $\Gm_1 := (\Gm, \al, \bt = \al$) solves $\bt$,
and $\Gm_2 := (\Gm, \al = 1, \bt = \al$) solves both $\al$ and $\bt$.

If the reader is not familiar with DK's algorithm,
he/she might be confused about the inconsistent types across judgment.
As an example, $(\lam x)~()$ synthesizes $\al$,
but the second premise of the subsumption rule uses $1$ for the result.
This is because a context application $[\Gm,\al=1,\bt=\al]\al = 1$ happens between the premises.

The existential variables $\al$ and $\bt$ are clearly not used after the subsumption rule,
but according to the algorithm, they are kept in the context
until some parent judgment recycles a block of variables,
or to the very end of a type inference task.
In that sense, DK's algorithm does not control the scoping of variables precisely.
%However, it is a minor issue that does not affect the soundness and completeness properties.

Two rules we may blame for not garbage collecting correctly are the inference rule for lambda
functions and an application inference rule:
$$
\inferrule*[right=$\mathtt{DK\_{\to}I{\To}}$]
{\Gamma, \al, \bt, x:\al \vdash e \Lto \bt \dashv \Delta, x:\al, \Theta}
{\Gamma \vdash \lam e \To \al \to \bt \dashv \Delta}
\qquad
\inferrule*[right=$\mathtt{DK\_\forall App}$]
{\Gamma, \al \vdash [\al/a] \appInf{A}{e}{C} \dashv \Delta}
{\Gamma \vdash \appInf{\all A}{e}{C} \dashv \Delta}
$$
In contrast, Rule 25 of our algorithm collects the existential variables
right after the second judgment chain,
and Rule 27 collects one existential variable similarly:
\[\Gm \Vdash \lam e \To_a \jg \rrule{25}
\Gm,\al,\bt \Vdash [\al\to\bt/a]\jg, x:\al \Vdash e\Lto \bt\]
\[\Gm \Vdash \appInfAlg{\all A}{e} \rrule{27} \Gm,\al \Vdash \appInfAlg{[\al/a]A}{e}\]
It seems impossible to achieve a similar outcome in DK's system by only modifying these two rules.
Taking $\mathtt{DK\_{\to}I{\To}}$ as an example,
the declaration or solution for $\al$ and $\bt$ may be referred to by subsequent judgments.
Therefore leaving $\al$ and $\bt$ in the output context is the only choice,
when the subsequent judgments cannot be consulted.

To fix the problem, one possible modification is on the algorithmic subsumption rule,
as garbage collection for a checking judgment is always safe:
$$
\inferrule*[right=$\mathtt{DK\_Sub}$]
{\Gamma, \blacktriangleright_{\al} \vdash e \To A \dashv \Theta \\
\Theta \vdash [\Theta]A \le [\Theta]B \dashv \Delta, \blacktriangleright_{\al}, \Delta'}
{\Gamma \vdash e \Lto B \dashv \Delta}
$$
Here we employ the markers in a way they were originally not intended for.
We create a dummy fresh existential $\al$ and add a marker to the input context of the inference judgment.
After the subtyping judgment is processed we look for the marker and drop everything afterwards.
We pick this rule because it is the only one where a checking judgment calls an inference judgment.
That is the point where our algorithm recycles variables---right after a judgment chain is satisfied.
After applying this patch, to the best of our knowledge,
DK's algorithm behaves equivalently to our algorithm in terms of variable scoping.
However, this exploits markers in a way they were not intended to be used and seems ad-hoc.

% \subsection{Scoped type variables} Not discussed for this paper

%-------------------------------------------------------------------------------
\subsection{Elaboration}

Type-inference algorithms are often extended with an
associated elaboration. For example, for languages with implicit
polymorphism, it is common to have an elaboration to a variant of
System F~\cite{reynolds1983types}, which recovers type information and explicit type
applications. Therefore a natural question is whether our algorithm
can also accommodate such elaboration.
While our algorithmic reduction does not elaborate to System F,
we believe that it is not difficult to extend the algorithm with a (type-directed) elaboration.
We explain the rough idea as follows.

Since the judgment form of our algorithmic worklist contains a collection of judgments,
elaboration expressions are also generated as a list of equal length to
the number of judgments (\emph{not judgment chains}) in the worklist.
As usual, subtyping judgments translate to coercions (denoted by $f$ and
represented by System F functions),
all three other types of judgments translate to terms in System F (denoted by $t$).

Let $\Phi$ be the elaboration list, containing translated type coercions and terms:
$$\Phi ::= \nil \mid \Phi, f \mid \Phi, t$$

Then the form of our algorithmic judgment becomes:
$$\Gamma \hookrightarrow \Phi$$

We take Rule 18 as an example, rewriting small-step reduction in a relational style,
$$
\inferrule*[right=$\mathtt{Translation\_18}$]
{\Gamma \Vdash e \To_a a \le B \hookrightarrow \Phi, f, t}
{\Gamma \Vdash e \Lto B \hookrightarrow \Phi, f t}
$$
As is shown in the conclusion of the rule,
a checking judgment at the top of the worklist corresponds to a top element for elaboration.
The premise shows that one judgment chain may relate to more than one elaboration elements,
and that the outer judgment, being processed before inner ones,
elaborates to the top element in the elaboration list.

Moreover, existential variables need special treatment, since they may be solved at any point,
or be recycled if not solved within its scope.
If an existential variable is solved, we not only propagate the solution to the other judgments,
but also replace occurrences in the elaboration list.
If an existential variable is recycled, meaning that it is not constrained,
we can pick any well-formed monotype as its solution.
The unit type $1$, as the simplest type in the system, is a good choice.
