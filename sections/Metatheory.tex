\section{Metatheory}

This section presents the metatheory of the algorithmic system
presented in the previous section. We show that three main results hold: 
\emph{soundness}, \emph{completeness} and \emph{decidability}.
These three results have been mechanically formalized and proved in the 
Abella theorem prover~\cite{}.

\subsection{Transfer to the Declarative System}
%\jimmy{The transfer rule outputs declarative judgment chains, which are later processed by another declarative relation, handling all the guessing.}

\begin{figure}[t]
\framebox{$\Gm \sto \Om$} $\Gm$ transfers to $\Om$.
\begin{gather*}
\inferrule*[right=$\mathtt{{\sto}}\Om$]
{~}
{\Om \sto \Om}
\quad
\inferrule*[right=$\mathtt{{\sto}\al}$]
{\Om\vdash\tau \\ \Om,[\tau/\al]\Gm \sto \Om}
{\Om,\al,\Gm \sto \Om}
\end{gather*}
\caption{Transfer rules}
\label{fig:trans}
\end{figure}

To aid formalizing the correspondence between the declarative and
algorithmic systems, we use a relation that transfers algorithmic
worklist judgements into corresponding declarative worklist judgements. In essence this
judgment instantiates every existential variable with a well-formed monotype.

The transfer rules are shown in Figure~\ref{fig:trans}.
Rule $\mathtt{{\sto}\al}$ replaces the first existential variable with a well-scoped monotype and
repeats the process in the resulting worklist.
Note that $\Om$, as a declarative worklist, contains no existential
variables. Therefore by recursively applying this rule, the
algorithmic worklist will finally become a declarative one.
In order to maintain well-formedness,
the substitution should affect all the judgments and term variable bindings in the scope of $\al$.

The transfer $\Gm\sto\Om$ is not deterministic.
% \bruno{Do you mean here
%   ``is not deterministic''? Would that be more appropriate?}
On the one hand, there are infinite possibilities to instantiate an existential variable,
thus a judment $\Gm$ could be transferred to different $\Om$'s.
On the other hand, any valid monotype in $\Om$ can be represented by an existential variable in $\Gm$,
thus different $\Gm$'s could be transferred to a single $\Om$.

Lemmas~\ref{lem:insert} and \ref{lem:extract}
generalize Rule $\mathtt{{\sto}\al}$ from substituting the first existential variable
to substituting any existential variable.

\begin{lemma}[Insert]\label{lem:insert}
If $\Gm_L\vdash \tau$ and $\Gm_L, [\tau/\al]\Gm_R \sto \Om$
, then $\Gm_L, \al, \Gm_R \sto \Om$.
\end{lemma}
\begin{lemma}[Extract]\label{lem:extract}
If $\Gm_L, \al, \Gm_R \sto \Om$
, then $\exists \tau$ s.t. $\Gm_L\vdash\tau$ and $\Gm_L, [\tau/\al]\Gm_R \sto \Om$.
\end{lemma}

\begin{figure}[t]
\framebox{$\|\Om\|$} Judgment erasure for $\Om$, returns a declarative context $\Psi$.
\begin{gather*}
\begin{aligned}
\|\nil\| &= \nil\\
\|\Om,a\| &= \|\Om\|, a\\
\|\Om,x:A\| &= \|\Om\|, x:A\\
\|\Om\Vdash\jg\| &= \|\Om\|
\end{aligned}
\end{gather*}

\framebox{$\Om \rto \Om'$} $\Om$ reduces to $\Om'$.
\begin{gather*}
\begin{aligned}
\Om,a &\rto \Om \qquad \Om,x:A \rto \Om\\
\Om\Vdash A\le B &\rto \Om \text{ when } \|\Om\| \vdash A\le B\\
\Om\Vdash e\Lto A &\rto \Om \text{ when } \|\Om\| \vdash e\Lto A\\
\Om\Vdash e\To_a \jg &\rto \Om\Vdash[A/a]\jg \text{ when } \|\Om\| \vdash e\To A\\
\Om\Vdash \appInfAlg{A}{e} &\rto \Om\Vdash[C/a]\jg \text{ when } \|\Om\| \vdash \appInf{A}{e}{C}\\
\end{aligned}
\end{gather*}
\caption{Declarative Worklist Typing.\bruno{I think your names for the
  two relations/definitions aren't great. I think what you're defining
here should be called transfer. The previous relation should probably
be called something different. Maybe something like Declarative
worklist instantiation/realization for the previous relation?}}
\label{fig:decl:worklist}
\end{figure}

Figure~\ref{fig:decl:worklist} relates the worklist-style declarative system
to the original declarative system.
A successful reduction means a valid variable declaration,
a valid subtyping or checking judgment, or a successful guessing for inferences.
Additionally, judgment chains are unfolded to match the original declarative rules.
Although $\Om$ is defined differently from $\Psi$,
our declarative worklist $\Om$ contains a ordered collection of variable definitions,
which naturally fits declarative subtyping and typing.
\bruno{I think that to make things precise what you want to show is
  that you can always define a erasure function that drops the
  judgements from the worklist, and retains only the variable
  declarations (thus becomes a declarative environment). Isn't this
  what you do in your formalization?}
\jimmy{TODO: a erasure function would make things more clear...}


\subsection{Non-overlapping Declarative System}
\label{sec:metatheory:non-overlapping}
\jimmy{Talk about the modified subsumption rule, reasoning about the unit rule,
and the subtyping $\forall$ overlapping}

The original DK's declarative system, shown in Figures \ref{fig:decl:sub} and \ref{fig:decl:typing},
has a few overlapping rules.
We propose a slightly modified declarative system, by changing a few rules in DK's declarative system:
\begin{enumerate}
\item Add a constraint to the subtyping rule $\mathtt{\forall L}$
$$
\inferrule*[right=$\mathtt{\forall L'}$]
{B \neq \all[b]B' \\ \Psi \vdash \tau \\ \Psi \vdash [\tau/a]A \le B}
{\Psi \vdash \all A \le B}
$$
\item Remove the redundant rule $\mathtt{Decl1I}$,
which could be easily derived by a combination of
$\mathtt{DeclSub}$, $\mathtt{Decl1I{\To}}$ and $\mathtt{{\le}Unit}$:
$$
\inferrule*[right=$\mathtt{DeclSub}$]
{
    \inferrule*[right=$\mathtt{Decl1I{\To}}$]
    {~}
    {\Psi \vdash () \To 1}
    \\
    \inferrule*[right=$\mathtt{{\le}Unit}$]
    {~}
    {\Psi \vdash 1 \le 1}
}
{\Psi \vdash () \Lto 1}
$$
\item Add constraints to the subsumption rule $\mathtt{DeclSub}$
$$
\inferrule*[right=$\mathtt{DeclSub'}$]
{e \neq \lam e' \\ A \neq \all A' \\ \Psi \vdash e \To A \\ \Psi \vdash A \le B}
{\Psi \vdash e \Lto B}
$$
\end{enumerate}

Now the new declarative system has no overlapping rules
and becomes more similar to the algorithmic system,
as the algorithmic rules contain similar constraints.
In addition, we proved that the modified declarative system is equivalent to the original one.

Modification (2) is relatively easy to justify, with the derivation given above:
the rule is redundant and can be replaced by a combination of three other rules.

Modificaitons (1) and (3) require inversion lemmas regarding rules that overlaps with them.
Rule $\mathtt{\forall L}$ initially overlaps with $\mathtt{\forall R}$ when facing judgment
$\Psi \vdash \all A \le \all[b]B$.
The following inversion lemma for Rule $\mathtt{\forall R}$ settles the overlapping:
\begin{lemma}[Invertibility of $\mathtt{\forall R}$]
If $\Psi \vdash A \le \all B$ then $\Psi, a \vdash A \le B$.
\end{lemma}
The lemma implies that giving a higher priority to $\mathtt{\forall R}$ does not affect the derivation.
Therefore the restriction $B \neq \all[b]B'$ is reasonable.

Rule $\mathtt{DeclSub}$ overlaps with both $\mathtt{Decl\forall I}$ and $\mathtt{Decl{\to}I}$.
Two similar inversion lemmas should be proven:
\begin{lemma}[Invertibility of $\mathtt{Decl\forall I}$]
If $\Psi \vdash e \Lto \all A$ then $\Psi, a \vdash e \Lto A$.
\end{lemma}
\begin{lemma}[Invertibility of $\mathtt{Decl{\to}I}$]
If $\Psi \vdash \lam e \Lto A \to B$ then $\Psi, x:A \vdash e \Lto B$.
\end{lemma}
These lemmas basically state that applying the specific rules is always no worse than
applying the more general Rule $\mathtt{DeclSub}$.
Proving both of the lemmas require an important property of the declarative system,
the subsumption lemma.
We first define the context subtyping relation in Figure~\ref{fig:context_subtyping},
and then show the subsumption lemma as follows.

\begin{figure}[t]
\framebox{$\Psi' \le \Psi$}
$$\begin{aligned}
\inferrule*[right=$\mathtt{CtxSubEmpty}$]
{~}{\nil \le \nil}\qquad
\inferrule*[right=$\mathtt{CtxSubTyVar}$]
{\Psi' \le \Psi}{\Psi', a \le \Psi, a}\qquad
\inferrule*[right=$\mathtt{CtxSubTmVar}$]
{\Psi' \le \Psi \\ \Psi \vdash A' \le A}{\Psi', x:A' \le \Psi, x:A}\qquad
\end{aligned}$$
\caption{Context Subtyping}
\label{fig:context_subtyping}
\end{figure}

\begin{lemma}[Subsumption]
Given $\Psi' \le \Psi$:
\begin{enumerate}
    \item If $\Psi \vdash e \Lto A$ and $\Psi \vdash A \le A'$ then $\Psi' \vdash e \Lto A$;
    \item If $\Psi \vdash e \To A$ then there exists
        $A'$ s.t. $\Psi \vdash A' \le A$ and $\Psi' \vdash e \To A$;
    \item If $\Psi \vdash \appInf{C}{e}{A}$ and $\Psi \vdash C' \le C$,
        then there exists $A'$ s.t. $\Psi \vdash A' \le A$ and $\Psi' \vdash \appInf{C'}{e}{A'}$.
\end{enumerate}
\end{lemma}

We tried to follow DK's proof on the subsumption lemma,
but the manual proof has several problems and is not able to be formalized.
Fortunately we fixed the issues and get the lemma proven.
Details about this issue can be found in Appendix~\ref{}.

\paragraph{Proof Sketch for Soundness and Completeness Theorems}
We now have three systems defined: the DK's declarative system,
the modified non-overlapping declarative system, and the algorithmic system.
First, we have got the statement that the non-overlapping declarative system,
as a restricted version, is equivalent to the DK's declarative system.
Then, the soundness of the algorithm can be directly shown against the original declarative system.
Finally, the completeness of the algorithm is shown against the non-overlapping declarative system,
which is equivalent the DK's declarative system.
As a result, we show that our algorithm is sound and complete with respect to DK's declarative system.

%-------------------------------------------------------------------------------
\subsection{Soundness}

Our algorithm is sound with respect to the declarative system.
For any worklist $\Gm$ that reduces successfully,
we can find a valid transfer $\Gm\sto\Om$ that solves all the existential variables,
such that $\Om$ also reduces successfully.
\begin{theorem}[Soundness]
If \emph{wf }$\Gm$ and $\Gm \redto \nil$, then $\exists \Om$ s.t. $\Gm\sto\Om$ and $\Om\redto\nil$.
\end{theorem}

The proof proceeds by induction on derivation of $\Gm\redto\nil$.
Interesting cases are the those involving existential variable instantiations,
including Rules 10, 11, 21 and 29.
Proper applications of lemmas \ref{lem:insert} and \ref{lem:extract}
help analyse the full instantiation of those existential variables.
For example, when $\al$ is solved to be $\al[1] \to \al[2]$ in the algorithm,
applying Extract gives two instantiations $\al[1] = \sigma$ and $\al[2] = \tau$.
Inserting back $\al = \sigma \to \tau$ for the induction case
matches the shape of induction hypothesis and thus finishes the corresponding case.

\bruno{Give a bit more informal explanation here. Something like: If
  $\Gm$ is reducible then we can show that there is always an $\Om$
  where all the corresponding declarative judgements hold.}
\bruno{I wonder if 1 or 2 examples (concrete instantiations of the
  theorem) would be helpful to complement the explanation.}

\begin{corollary}[Soundness, single judgment form]
Given \emph{wf }$\Psi$ and $\Gm = \Psi$:
\begin{enumerate}
    \item If $\Gm \Vdash A \le B \redto \nil$ then $\Psi \vdash A \le B$;
    \item If $\Gm \Vdash e \Lto A \redto \nil$ then $\Psi \vdash e \Lto A$;
    \item If $\Gm \Vdash e \To_a \jg \redto \nil$ for any $\jg$, then there exists $A$
        s.t. $\Psi \vdash e \To A$;
    \item If $\Gm \Vdash \appInfAlg{A}{e} \redto \nil$ for any $\jg$, then there exists $C$
        s.t. $\Psi \vdash \appInf{A}{e}{C}$.
\end{enumerate}
\end{corollary}

\subsection{Completeness}

Completeness of the algorithm means that any declarative system has an algorithmic counterpart.

\begin{theorem}[Completeness]
If \emph{wf }$\Gm$ and $\Gm\sto\Om$ and $\Om\redto\nil$, then $\Gm \redto \nil$.
\label{thm:completeness}
\end{theorem}

We prove completeness by induction on the derivation of $\Om\redto\nil$.
Here we refer to the non-overlapping declarative system without chaning any syntax.
Since the declarative worklist is reduced judgment-by-judgment
(shown in Figure~\ref{fig:decl:worklist}),
the induction always analyses the first judgment until it is satisfied.
In Abella, we use a different relation to get the correct induction scheme.

As the algorithmic system introduces existential variables,
a declarative rule may correspond to multiple algorithmic rules,
and thus we analyse each of the possible cases.
\jimmy{Cite DK's proof as the sketch.}

For subtyping, algorithmic Rules 10 and 11 require special treatment.
When the induction reaches the $\mathtt{{\le}{\to}}$ case,
the first judgment is of shape $A_1 \to A_2 \le B_1 \to B_2$.
One of the corresponding algorithmic judgment is $\al \le A \to B$.
However, the case analysis does not imply that $\al$ is fresh in $A$ and $B$,
therefore Rule~10 cannot be applied and the proof gets stuck.
The following lemma helps us out in those cases:
the success in declarative subtyping indicates the freshness of $\al$ in $A$ and $B$
in its corresponding algorithmic judgment.
In other words, the declarative system does not accept infinite types,
thus, as an example, we cannot find a monotype $\tau$ such that $\tau\le 1\to \tau$,
which could be transferred by $\al\le 1\to\al$.

\begin{lemma}[Prune Transfer for Instantiation]
If $(\Gm \Vdash \al \le A \to B) \sto (\Om \Vdash C \le A_1 \to B_1)$ and
$\Om \vdash C \le A_1 \to B_1$, then $\al\notin FV(A) \cup FV(B)$.
\end{lemma}
A symmetric lemma holds for $A\to B \le \al$.

All other cases are relatively easy to prove.
The Insert and Extract lemmas are applied when the algorithm uses existential variables,
but transferred to a monotype for the declarative system,
such as Rules 6, 10, 11, 12-17, 21 and 29.

The following corollary is derived immediately from Theorem~\ref{thm:completeness}.
\begin{corollary}[Completeness, single judgment form]
Given \emph{wf }$\Psi$ and $\Gm = \Psi$:
\begin{enumerate}
    \item If $\Psi \vdash A \le B$ then $\Gm \Vdash A \le B \redto \nil$;
    \item If $\Psi \vdash e \Lto A$ then $\Gm \Vdash e \Lto A \redto \nil$;
    \item If $\Psi \vdash e \To A$ then $\Gm \Vdash e \To_a 1 \le 1 \redto \nil$;
    \item If $\Psi \vdash \appInf{A}{e}{C}$ then $\Gm \Vdash \appInfAlg{A}{e}[a][1 \le 1] \redto \nil$.
\end{enumerate}
\end{corollary}

\subsection{Decidability}
\begin{theorem}[Decidability]
Given \emph{wf }$\Gm$, it is decidable whether $\Gm\redto\nil$ or not.
\end{theorem}

We have proven this theorem by a lexicographic group of induction measures\\
$\langle |\Gm|_e, |\Gm|_\Leftrightarrow, |\Gm|_\forall, |\Gm|_{\al}, |\Gm|_\to + |\Gm| \rangle$
on the worklist $\Gm$.
$|\cdot|_e$ and $|\cdot|_\Leftrightarrow$ measure the total ``size'' of terms
and the total ``difficulty'' of judgments, respectively;
$|\cdot|_\forall$ and $|\cdot|_\to$ count the total number of
universal quantifiers and function types, respectively;
$|\cdot|_{\al}$ counts the number of existential variables in the worklist.

\begin{definition}[Worklist Measures]
\begin{gather*}
\begin{aligned}
|\Gm|_e &= \textstyle\sum_{e\Lto A\in\Gm}|e|_e + \sum_{e\To_a \jg\in\Gm}|e|_e +
    \sum_{\appInfAlg{A}{e}\in\Gm}|e|_e\\
\text{where } |x|_e &= |()|_e = 1 \qquad |\lam e|_e = |e|_e + 1\\
|e_1~e_2|_e &= |e_1|_e + |e_2|_e + 1 \qquad |e:A|_e = |e| + 1\\[2mm]
%
|\Gm|_\Leftrightarrow &= 2\cdot\#_{e\Lto A\in\Gm} +
    \#_{e\To_a \jg\in\Gm} + 3\cdot\#_{\appInfAlg{A}{e}\in\Gm}\\[2mm]
|\Gm|_\forall &= \textstyle\sum_{e\Lto A\in\Gm}|A|_\forall + \sum_{\appInfAlg{A}{e}\in\Gm}|A|_\forall +
    \sum_{A\le B\in\Gm} |A|_\forall + |B|_\forall\\
\text{where } |1|_\forall &= |a|_\forall = |\al|_\forall = 1\\
|A\to B|_\forall &= |A|_\forall + |B|_\forall \qquad |\all A|_\forall = |A|_\forall + 1\\[2mm]
%
|\Gm|_\to &= \textstyle\sum_{e\Lto A\in\Gm}|A|_\to + \sum_{\appInfAlg{A}{e}\in\Gm}|A|_\to +
    \sum_{A\le B\in\Gm} |A|_\to + |B|_\to\\
\text{where } |1|_\to &= |a|_\to = |\al|_\to = 1\\
|A\to B|_\to &= |A|_\to + |B|_\to + 1 \qquad |\all A|_\to = |A|_\to\\[2mm]
%
|\Gm|_{\al} &= \#_{\al\in\Gm}\\
\end{aligned}
\end{gather*}
\end{definition}

It is not difficult to see that all but two algorithmic reduction rules
decreases the group of measures.
(The result of Rule 29 could be directly reduced by Rule 28, which decreases the measures.)
The two exceptions are Rules 10 and 11; they increases the number of existential variables
without decreasing the number of universal quantifiers.
However, at the same time, they both break the subtyping problem into two sub-problems
(Rule 7 applied right after Rule 10 or 11), in smaller sizes.
Therefore we argue that such split makes the worklist easier, by analyzing a subset of worklist,
which we call \emph{instantiation judgments} $\Gm_i$, with the help of $\Gm_\le$ for the proof.

\begin{definition}[$\Gm_\le, \Gm_i$]
\begin{gather*}
\begin{aligned}
\Gm_\le &:= \nil \mid \Gm_\le, a \mid \Gm_\le, \al \mid \Gm_\le, A\le B\\
\Gm_i &:= \nil \mid \Gm_i, \al\le A \mid \Gm_i, A\le\al \quad
    \text{where } \al\notin FV(A) \cup FV(\Gm_i)
\end{aligned}
\end{gather*}
\end{definition}

\begin{figure}
\framebox{$\Gm\jExt\Gm'$} $\Gm$ extends to $\Gm'$.
\begin{gather*}
\inferrule*[right=$\mathtt{\jExt id}$]
    {~}{\Gm\jExt\Gm}
\qquad
\inferrule*[right=$\mathtt{\jExt solve}$]
    {|A|_\forall = 0 \quad \Gm_L,[A/\al]\Gm_R \jExt \Gm'}{\Gm_L,\al,\Gm_R \jExt \Gm'}
\qquad
\inferrule*[right=$\mathtt{\jExt \al}$]
    {\Gm_L,\al,\Gm_R \jExt \Gm'}{\Gm_L,\Gm_R \jExt \Gm'}
\end{gather*}
\caption{Worklist Extension}\label{fig:worklist_ext}
\end{figure}

\paragraph{Intuition of instantiation decidability}
These instantiation judgments are the ones consumed and produced by Rules 10 and 11.
When there are universal quantifiers in the initial instantiation judgments $\Gm_i$,
the worklist $\Gm,\Gm_i$ will be multi-reduced to match Rule 8 or 9,
which decreases the number of universal quantifiers by 1, and returns a $\Gm',\Gm_\le$.
Otherwise, no more variables will be introduced,
and each instantiation judgment solves one existential variable,
thus finally there will be $|\Gm_i|$ existential variables solved,
with the worklist reduced to $\Gm'$.
$\Gm'$ basically refers to the original part of worklist $\Gm$.
During the reduction, minor changes to worklist might occur, such as
existential variable solving, insertion and instantiation,
but the main structure of it remains unchanged.

Shown in Figure~\ref{fig:worklist_ext},
worklist extension formally describe the minor changes to worklist during algorithmic reduction.
When the algorithm processes the worklist, the first judgment is processed,
and the rest of the worklist roughly remains its shape.
Worklist extension is formally defined by two cases:
existential variable solving ($\mathtt{\jExt solve}$) and insertion ($\mathtt{\jExt \al}$).
Existential variable instantiation, as another common form of worklist manipulation,
i.e. $\Gm_L,\al,\Gm_R \jExt \Gm_L, \al[1], \al[2], [\al[1] \to \al[2] / \al]\Gm_R$,
can be derived with the combination of the 3 rules.
Within expectation, some worklist measures are not changed modulo worklist extension.

\newcommand{\equivGm}[1]{|\Gm|_{#1} = |\Gm'|_{#1}}
\begin{lemma}[Measure Invariants of Worklist Extension]
If $\Gm\sto\Gm'$ then $\equivGm{e}$ and $\equivGm\Leftrightarrow$ and $\equivGm\forall$.
\end{lemma}

With all the definitions, we are able to prove the following instantiation decidability lemma.

\begin{lemma}[Instantiation Decidability]\label{lemma:inst:decidable}
For any well-formed algorithmic worklist $(\Gm, \Gm_i)$,
\begin{enumerate}[1)]
    \item When $|\Gm_i|_\forall = 0$, $(\Gm, \Gm_i) \redto \Gm'$,
        where $|\Gm'|_{\al} = |\Gm|_{\al} - |\Gm_i|$ and $\Gm\sto\Gm'$.
    \item When $|\Gm_i|_\forall > 0$, $(\Gm, \Gm_i) \redto (\Gm', \Gm_\le)$,
        where $|\Gm_\le|_\forall = |\Gm_i|_\forall - 1$ and $\Gm\sto\Gm'$.
\end{enumerate}
\end{lemma}

In short, after reducing any instantiation judgment prefix $\Gm_i$,
the algorithm either decreases the number of $\forall$'s
or solves one existential variable per instantiation judgment.
The proof of this lemma is by induction on the measure $2*|\Gm_i|_\to + |\Gm_i|$
of the instantiation judgment list $\Gm_i$.

In summary, the decidability theorem is shown through a lexicographic group of induction measures
$\langle |\Gm|_e, |\Gm|_\Leftrightarrow, |\Gm|_\forall, |\Gm|_{\al}, |\Gm|_\to + |\Gm| \rangle$.
The critical case is that, whenever we encounter an instantiation judgment at the top of the worklist,
we make use of Lemma~\ref{lemma:inst:decidable}, which reduces $|\Gm|_{\al}$ or $|\Gm|_\forall$
by consuming that instantiation judgment. Other cases are relatively straightforward.

Combining all the three main results, we know that the declarative system is decidable,
with the help of our algorithm.
\begin{corollary}[Decidability of Declarative Typing]
Given \emph{wf }$\Om$, it is decidable whether $\Om\redto\nil$ or not.
\end{corollary}
\subsection{Abella}
