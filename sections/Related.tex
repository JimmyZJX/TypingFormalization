\section{Related Work}

\paragraph{Polymorphic Type Inference Algorithms}
Higher-order polymorphism is a convenient and practical feature of programming language.
Since full type-inference for System F is undecidable~\cite{wells1999typability},
various decidable partial type-inference algorithms were developed.
The declarative system of this paper, proposed by DK, is a \emph{predicative} one,
where $\forall$'s only instantiate to mono-types.
The mono-type restriction on instantiation is considered reasonable and practical for most programs,
except for those that require sophisticated form of higher-order polymorphism.
In those cases, the bidirectional system accepts guidance through type annotations,
which also improves readability of the program and is typically not considered as a burden.

Impredicative System F allows instantiation with polymorphic types,
but unfortunately its subtyping system is already undecidable~\cite{tiuryn1996subtyping}.
Works on partial impredicative type-inference algorithms\cite{le2003ml,leijen2008hmf,vytiniotis2008fph}
navigate a variety of design tradeoffs for a decidable algorithm.
As a result, such algorithms tend to be more complicated, and thus less adopted in practise.

\jimmy{Guarded Impredicative}

\jimmy{Dunfield's new work}

