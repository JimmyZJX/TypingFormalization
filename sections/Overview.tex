\section{Overview}

This section starts by introducing DK's declarative type system. Then
it discusses several techniques that have been used in algorithmic
formulations, which have influenced our own algorithmic design.
Finally we introduce the novelties of our new algorithm.
In particular the support for inference judgements in
judgement worklists, and a new form of judgement worklist
that the unifies \emph{ordered contexts} and the worklists themselves. 

\subsection{DK's Declarative System}
% Subtyping and Typing. \jimmy{Can you prepare the figures. Subtyping is already in ITP.}

\begin{figure}[t]
\[
\begin{array}{l@{\qquad}lcl}
\text{Type variables}\qquad&a, b
\\[3mm]
\text{Types}\qquad&A, B, C &::=&\quad 1 \mid a \mid \forall a. A \mid A\to B\\
\text{Monotypes}\qquad&\tau,\sigma &::=&\quad 1 \mid a \mid \tau\to \sigma
\\[3mm]
\text{Expressions}\qquad&e &::=&\quad x \mid () \mid \lam e \mid e_1~e_2 \mid (e:A)
\\[3mm]
\text{Contexts}\qquad&\Psi &::=&\quad \nil \mid \Psi, a \mid \Psi, x:A
\end{array}
\]
\caption{Syntax of Declarative System}\label{fig:decl:syntax}
\end{figure}

\paragraph{Syntax.}
The syntax of DK's declarative system~\cite{dunfield2013complete} is shown in Figure~\ref{fig:decl:syntax}.
A declarative type $A$ is either the unit type $1$, a type variable $a$,
a universal quantification $\forall a. A$ or a function type $A \to B$.
Nested universal quantifiers are allowed for types,
but monotypes $\tau$ do not have any universal quantifier.
Terms include a unit term $()$, variables $x$, lambda-functions $\lam e$,
applications $e_1~e_2$ and annotations $(e:A)$.
Contexts $\Psi$ are sequences of type variable declarations and
term variables with its type declared $x:A$.

\begin{figure}[t]
%\centering \framebox{$\Psi \vdash A$}
%\begin{gather*}
%\inferrule*[right=$\mathtt{wf_d unit}$]
%    {~}{\Psi\vdash 1}
%\qquad
%\inferrule*[right=$\mathtt{wf_d var}$]
%    {a\in\Psi}{\Psi\vdash a}
%\qquad
%\inferrule*[right=$\mathtt{wf_d{\to}}$]
%    {\Psi\vdash A\quad \Psi\vdash B}
%    {\Psi\vdash A\to B}
%\qquad
%\inferrule*[right=$\mathtt{wf_d\forall}$]
%    {\Psi, a\vdash A}
%    {\Psi\vdash \forall a. A}
%\end{gather*}

\centering \framebox{$\Psi \vdash A \le B$}
\begin{gather*}
\inferrule*[right=$\mathtt{{\le}Var}$]
    {a\in\Psi}{\Psi\vdash a\le a}
\qquad
\inferrule*[right=$\mathtt{{\le}Unit}$]
    {~}{\Psi \vdash 1 \le 1}
\qquad
\inferrule*[right=$\mathtt{{\le}{\to}}$]
    {\Psi \vdash B_1 \le A_1 \quad \Psi \vdash A_2 \le B_2}
    {\Psi\vdash A_1\to A_2 \le B_1\to B_2}
\\
\inferrule*[right=$\mathtt{{\le}\forall L}$]
    {\Psi\vdash \tau \quad \Psi\vdash [\tau/a] A \le B}
    {\Psi\vdash \forall a. A \le B}
\qquad
\inferrule*[right=$\mathtt{{\le}\forall R}$]
    {\Psi, a\vdash A\le B}
    {\Psi\vdash A \le \forall a. B}
\end{gather*}
\caption{%Well-formedness of Declarative Types and 
Declarative Subtyping}\label{fig:decl:sub}
\end{figure}

\begin{figure}[t]
\begin{tabular}{rl}
    \framebox{$\Psi \vdash A \Lto B$} & $e$ checks against input type $A$.\\[0.5mm]
    \framebox{$\Psi \vdash A \To B$} & $e$ synthesizes output type $A$.\\[0.5mm]
    \framebox{$\Psi \vdash \appInf{A}{e}{C}$} & Applying a function of type $A$ to $e$ synthesizes type $C$.
\end{tabular}
\begin{gather*}
\inferrule*[right=$\mathtt{DeclVar}$]
    {(x:A)\in\Psi}{\Psi\vdash x\To A}
\qquad
\inferrule*[right=$\mathtt{DeclSub}$]
%e \neq \lam e' \quad B \neq \all B' \quad 
    {\Psi\vdash e\To A \quad \Psi\vdash A\le B}
    {\Psi \vdash e\Lto B}
\\
\inferrule*[right=$\mathtt{DeclAnno}$]
    {\Psi \vdash A \quad \Psi\vdash e\Lto A}
    {\Psi\vdash (e:A)\To A}
\qquad
\inferrule*[right=$\mathtt{Decl1I}$]
    {~}{\Psi\vdash () \Lto 1}
\qquad
\inferrule*[right=$\mathtt{Decl1I\To}$]
    {~}{\Psi\vdash () \To 1}
\\
\inferrule*[right=$\mathtt{Decl\forall I}$]
    {\Psi,a \vdash e \Lto A}
    {\Psi\vdash e\Lto \all A}
\qquad
\inferrule*[right=$\mathtt{Decl\forall App}$]
    {\Psi \vdash \tau \quad \Psi\vdash \appInf{[\tau/a]A}{e}{C} }
    {\Psi\vdash \appInf{\all A}{e}{C}}
\\
\inferrule*[right=$\mathtt{Decl\to I}$]
    {\Psi,x:A \vdash e\Lto B}
    {\Psi\vdash \lam e \Lto A \to B}
\qquad
\inferrule*[right=$\mathtt{Decl\to I\To}$]
    {\Psi\vdash \sigma\to\tau \quad \Psi,x:\sigma \vdash e\Lto \tau}
    {\Psi\vdash \lam e \To \sigma\to\tau}
\\
\inferrule*[right=$\mathtt{Decl\to E}$]
    {\Psi\vdash e_1\To A \quad \Psi\vdash \appInf{A}{e_2}{C}}
    {\Psi\vdash e_1~e_2 \To C}
\qquad
\inferrule*[right=$\mathtt{Decl\to App}$]
    {\Psi\vdash e \Lto A}
    {\Psi\vdash \appInf{A \to C}{e}{C}}
\end{gather*}
\caption{Declarative Typing}\label{fig:decl:typing}
\end{figure}
\bruno{Some of the rule names are messed up.}

%The DK subtyping relation was adopted from \citet{odersky1996putting}.

\paragraph{Declarative Subtyping}
Figure~\ref{fig:decl:sub} shows DK's declarative subtyping judgment $\Psi \vdash A \le B$,
which was adopted from \citet{odersky1996putting}. It compares the
degree of polymorphism between between $A$ and $B$ in DK's implicit polymorphic type system. 
Essentially, if $A$ can always be instantiated to match any instantiation of $B$,
then A is ``at least as polymorphic as'' $B$. We also 
say that $A$ is ``more polymorphic than'' $B$ and write $A \le B$.

Subtyping rules $\mathtt{{\le}Var}$, $\mathtt{{\le}Unit}$ and $\mathtt{{\le}{\to}}$
handle simple cases that do not involve universal quantifiers.
The subtyping rule for function types $\mathtt{{\le}{\to}}$ is standard,
being covariant on the return type and contravariant on the argument type.
Rule $\mathtt{{\le}\forall R}$ states that if $A$ is a subtype of $B$
in the context $\Psi, a$, where $a$ is fresh in $A$, then $A\le\all B$.
Intuitively, if $A$ is more general than $\all B$ (where the universal quantifier
already indicates that $\all B$ is a general type),
then $A$ must instantiate to $[\tau/a]B$ for every $\tau$.

The most interesting rule is $\mathtt{{\le}\forall L}$.
If some instantiation of $\all A$, $[\tau/a]A$, is a subtype of $B$,
then $\all A \le B$.
The monotype $\tau$ we used to instantiate $a$ is \emph{guessed} in this
declarative rule, but the algorithmic system does not guess and defers the
instantiation until it can determine the monotype deterministically.
The fact that $\tau$ is a monotype rules out the possibility of
polymorphic (or impredicative) instantiation.
However this restriction ensures that the subtyping relation remains
decidable. Allowing an arbitrary type (rather than a monotype) in rule $\mathtt{{\le}\forall L}$
is know to give rise to an undecidable subtyping relation~\cite{tiuryn1996subtyping}.
Peyton Jones et al.~\cite{jones2007practical} also impose the restriction of
predicative instantiation in their type system.
Both systems are adopted by several practical programming languages.

\paragraph{Declarative Typing}
The bidirectional type system, shown in Figure~\ref{fig:decl:typing}, has three judgments.
The checking judgment $\Psi\vdash e\Lto A$ checks expression $e$ against the type $A$ in the context $\Psi$.
The synthesis judgment $\Psi\vdash e\To A$ synthesizes the type $A$ of expression $e$ in the context $\Psi$.
The application judgment $\Psi\vdash \appInf{A}{e}{C}$ synthesizes the type $C$ of the application of a function of type $A$
(which could be polymorphic) to the argument $e$.

Many rules are standard.
Rule $\mathtt{DeclVar}$ looks up term variables in the context.
Rules $\mathtt{Decl1I}$ and $\mathtt{Decl1I{\To}}$ respectively check and synthesize the unit type.
Rule $\mathtt{DeclAnno}$ synthesizes the annotated type $A$ of the annotated expression $(e:A)$
and checks that $e$ has type $A$.
Checking an expression $e$ against a polymorphic type $\all A$ in the context $\Psi$ succeeds
if $e$ checks against $A$ in the extended context $(\Psi, a)$.
The subsumption rule $\mathtt{DeclSub}$ depends on the subtyping relation,
and changes mode from checking to synthesis: if $e$ synthesizes type $A$ and $A\le B$
($A$ is more polymorphic than $B$), then $e$ checks against $B$.
If a checking problem does not match any other rules,
this rule can be applied to synthesize a type instead and then
check whether the synthesized type entails the checked type.

Lambda abstractions are the hardest construct of the bidirectional
type system to deal with. 
Checking $\lam e$ against function type $A\to B$ is easy:
we check the body $e$ against $B$ in the context extended with $x:A$.
However, synthesizing a lambda-function is a lot less easy, and 
this type system only synthesizes monotypes $\sigma\to\tau$.

Application $e_1~e_2$ is handled by Rule $\mathtt{Decl{\to}E}$,
which first synthesizes the type $A$ of the function $e_1$.
If $A$ is a function type $B\to C$, Rule $\mathtt{Decl{\to}App}$ is applied;
it checks the argument $e_2$ against $B$ and returns type $C$.
The synthesized type of function $e_1$ can also be polymorphic, of the form $\all A$.
In that case, we instantiate $A$ to $[\tau/a]A$ with a monotype $\tau$ % (which is also guessed)
using according to Rule $\mathtt{Decl{\to}I{\To}}$.
If $[\tau/a]A$ is a function type, Rule $\mathtt{Decl{\to}App}$ proceeds;
if $[\tau/a]A$ has another universal quantifier, Rule $\mathtt{Decl{\to}I{\To}}$ is recursively applied.

\jimmy{Possibly some discussions about the typing rules}

\paragraph{Overlapping Rules}
\jimmy{Mention the resolution of overlapping briefly, and then point to algorithm and metatheory}
Some of the declarative rules overlaps with each other.
Declarative subtyping rules $\mathtt{{\le}\forall L}$ and $\mathtt{{\le}\forall R}$
both match the conclusion $\Psi\vdash \all A \le \all B$.
In such case, choosing $\mathtt{{\le}\forall R}$ first is always better,
since we introduce the type variable $a$ to the context earlier,
which gives more flexibility on the choice of $\tau$.

Declarative typing rule $\mathtt{DeclSub}$ overlaps with
both $\mathtt{Decl\forall I}$ and $\mathtt{Decl\to I}$.
However, we could argue that more specific rules are always the best choices,
i.e. $\mathtt{Decl\forall I}$ and $\mathtt{Decl\to I}$ should have
higher priority to $\mathtt{DeclSub}$.
For example, $\Psi\vdash \lam x \Lto \all a\to a$ succeeds if derived from
Rule $\mathtt{Decl\forall I}$, but fails when applied to $\mathtt{DeclSub}$:
$$
\inferrule*[right={$\mathtt{Decl\forall I}$}]
	{\inferrule*[Right={$\mathtt{Decl\to I}$}]
		{\Psi,a,x:a \vdash x \Lto a}
		{\Psi,a \vdash \lam x \Lto a \to a}
	}
	{\Psi\vdash \lam x \Lto \all a \to a}
$$
$$
\inferrule*[right={$\mathtt{DeclSub}$}]
	{
		\inferrule*[right=$\mathtt{Decl\to I\To}$]
			{\Psi \vdash \blue\sigma\to \blue\tau \quad \Psi,x:\blue\sigma\vdash e \Lto \blue\tau}
			{\Psi \vdash \lam x \To \blue\sigma\to \blue\tau}\quad
		\inferrule*[Right=$\mathtt{{\le}\forall R}$]
			{\inferrule*[Right=$\mathtt{{\le}{\to}}$]
				{
					\inferrule*[Right=$?$]
						{\text{Impossible! }\blue\sigma \neq a}
						{\Psi,a \vdash a \le \blue\sigma}
					\quad \Psi,a \vdash \blue\tau \le a
				}
				{\Psi,a \vdash \blue\sigma\to \blue\tau \le a \to a}
			}
			{\Psi\vdash \blue\sigma\to \blue\tau\le \all a \to a}
	}
{\Psi\vdash \lam x \Lto \all a \to a}
$$

Rule $\mathtt{Decl\to I}$ is also better at handling higher-order types.
When the lambda-expression to be inferred has a polymorphic input type,
such as $\all a \to a$,
$\mathtt{DeclSub}$ may not derive some judgments.
For example, $\Psi,id:\all a\to a \vdash \lam[f] f~id~(f~()) \Lto (\all a\to a) \to 1$
requires the argument of the lambda-expression to be a polymorphic type,
otherwise it could not be applied to both $id$ and $()$.
If Rule $\mathtt{DeclSub}$ was chosen for derivation,
the type of its argument is restricted by Rule $\mathtt{Decl\to I\To}$,
which is not a polymorphic type.
By contrast,
Rule $\mathtt{Decl\to I}$ keeps the polymorphic argument type $\all a\to a$,
and will successfully derive the judgment.

%-------------------------------------------------------------------------------
\subsection{DK's Algorithm}

DK's algorithm version revolves around their notion of \emph{algorithmic context}.
\[
\begin{array}{l@{\qquad}lcl}
\text{Algorithmic Contexts}\qquad&\Gamma,\Delta,\Theta &::=&\quad \nil \mid
\Gamma, a \mid \Gamma, \hat{\alpha} \mid \Gamma, \hat{\alpha} = \tau \mid
\Gamma, \blacktriangleright_{\hat{\alpha}}
\end{array}
\]
In addition to the regular (universally quantified) type variables $a$, the
algorithmic context also contains \emph{existential} type variables
$\hat{\alpha}$. These are placeholders for monotypes $\tau$ that are still to
be determined by the inference algorithm. When the existential variable is
``solved'', its entry in the context is replaced by the assignment
$\hat{\alpha} = \tau$. With the notation $[\Gamma]$ an algorithmic context
$\Gamma$ denotes the substitution that replaces all its solved existential type
variables with their solution.

All algorithmic judgments thread an algorithmic context. They have the form
$\Gamma \vdash \ldots \dashv \Delta$, where $\Gamma$ is the input context and
$\Delta$ is the output context:
$\Gamma \vdash A \le B \dashv \Delta$  for subtyping, 
$\Gamma \vdash e \Leftarrow A \dashv \Delta$  for type checking, and so on. 
The output context is a functional update of the input context that records newly
introduced existentials and solutions.

Solutions are incrementally propagated by applying the algorithmic output
context of a previous task as substitution to the next task. This can be seen
in the subsumption rule:
\[
\inferrule*
  {\Gamma \vdash e \Rightarrow A \dashv \Theta \\ 
   \Theta \vdash [\Theta]A \le [\Theta]B \dashv \Delta
  }
  { \Gamma \vdash e \Leftarrow B \dashv \Delta}
\]
The inference task yields an output context $\Theta$ which is applied as a substitution
to the types $A$ and $B$ before performing the subtyping check to propagate any solutions
of existential variables that appear in $A$ and $B$.

\paragraph{Markers for scoping.}
The sequential order of entries in the algoritmic context, in combination with
the threading of contexts,  does not perfectly capture the scoping of all
existential variables. For this reason the DK algorithm uses scope markers
$\blacktriangleright_{\hat{\alpha}}$ in a few places. An example is given in the following
rule:
\[
\inferrule*
  {\Gamma,\blacktriangleright_{\hat{\alpha}}, \hat{\alpha} \vdash [\hat{\alpha}/a]A \le B \dashv \Delta,\blacktriangleright_{\hat{\alpha}},\Theta}
  {\Gamma \vdash \forall a. A \le B \dashv \Delta}
\]
To indicate that the scope of $\hat{\alpha}$ is local to the subtyping check
$[\hat{\alpha}/a]A \le B$, the marker is pushed onto its input stack and popped
from the output stack together with the subsequent part $\Theta$, which may
refer to $\hat{\alpha}$. (Remember that later entries may refer to earlier
ones, but not vice versa.) This way $\hat{\alpha}$ does not escape its scope.


At first sight, the DK algorithm would seem a good basis for mechanisation. After
all it comes with a careful description and extensive manual proofs.
Unfortunately, we ran into several obstacles that have prompted us to formulate
a different, more mechanisation-friendly algorithm.

%- - - - - - - - - - - - - - - - - - - - - - - - - - - - - - - - - - - - - - - - 
\paragraph{Broken Metatheory} % Talk about the lemmas that are false, show counterexample.
While going through the manual proofs of DK's algorithm, we found several
problems.  Indeed, two proofs of lemmas---Lemma 19 (Extension Equality
Preservation) and Lemma 14 (Subsumption)--- wrongly apply induction hypotheses
in several cases. Fortunately, we have found simple work-arounds that fix these
proofs without affecting the appeals to these lemmas.

% The proof of Lemma 19 (Extension Equality Preservation) applies a wrong
% induction hypothesis for the ${\longrightarrow}\mathtt{Uvar}$ case.
% Fortunately the lemma could be proven without the well-formedness conditions
% $\Gm\vdash A$ and $\Gm\vdash B$.
% Similar problem happens for Lemma 14 (Subsumption), where many applications of
% induction hypotheses are not correct.
% For manual proofs, induction hypotheses are not automatically generated by programs,
% thus are easily misused.

More seriously, we have also found a lemma that simply does not hold, 
Lemma 29 (Parallel Admissibility). See Appendix~\ref{} for a detailed explanation
and counterexample. This lemma is a cornerstone of the two metatheoretical results 
of the algorithm, soundness and completeness with respect to the declarative system.
In particular, both instantiation soundness (i.e. a part of subtyping
soundness) and typing completeness directly require the broken lemma.
Moreover, Lemma 54 (Typing Extension) also requires the broken lemma and is
itself used 13 times in the proof of typing soundness and completeness.
Unfortunately, we have not yet found a way to fix this problem.
\footnote{We found the issue with Lemma 29 in 2016 in an earlier attempt to mechanically
  formalize DK's algorithm. The authors acknowledged the problem after we contacted them through email.
  Although they briefly mentioned that it should be possible to use a weaker lemma instead they did
  not go into details.
}
\bruno{We should acknowledge Ningning for finding this problem, but I think
  this may be against the anonymous submission? Perhaps we do not mention her name
in the submission, and if accepted we mention her?}
\tom{You have communicated this to the authors and not received a fix, right?
     We could mention that too then.}

%- - - - - - - - - - - - - - - - - - - - - - - - - - - - - - - - - - - - - - - - 
\paragraph{Complex Scoping and Propagation}

Besides the problematic lemmas in DK's metatheory, there are other reasons to
look for an alternative algorithmic formulation of the type system that is more
amenable to mechanisation. Indeed, two aspects that are particularly
challenging to mechanise are the scoping of universal and existential type
variables, and the propagation of the instantiation of existential type
variables across judgments. 

DK is already quite disciplined on these accounts, in particular compared to
traditional constraint-based type-inference algorithms like Algorithm W which
features an implicit global scope for all type variables. Indeed, DK uses its
explicit and ordered context $\Gamma$ that tracks the relative scope of universal and
existential variables and it is careful to only instantiate existential
variables in a well-scoped manner.

Moreover, DK's algorithm carefully propagates instantiations by recording them
into the context and threading this context through all judgments. 
% two This means that every judgments takes two contexts---an input and an output
% context---rather than the conventional single context. The output context
% records any new variable instantations; to propagate these instantiations to
% the remaining judgements, their predecessor's output context---which is their
% input context---is applied to them as a substitution.
While this works well on paper, this approach is still fairly involved and thus
hard to reason about in a mechanised setting. Indeed, the instantiations have
to be recorded in the context and are applied incrementally to each remaining
judgment in turn, rather than immediately to all remaining judgments at once.
Also, as we have mentioned above, the stack discipline of the ordered contexts
does not mesh well with the use of output contexts; explicit marker entries are
needed in two places to demarcate the end of an existential variable's scope.
These are compelling reasons to look for a simpler algorithm that is easier to
reason about in a mechanised setting.


%-------------------------------------------------------------------------------
\subsection{Judgment Lists}

To avoid the problem of incrementally applying a substitution to remaining
tasks, we can find inspiration in the formulation of constraint solving
algorithms. For instance, the well-known unification
algorithm by \citet{unification} decomposes the problem of unifying two terms $s \stackrel{.}{=} t$ into a set
of related unification problems between pairs of terms $s_i \stackrel{.}{=} t_i$. These smaller
problems are not tackled independently, but kept together in a set $S$. 
The algorithm itself is typically formulated as a small-step-style state
transition system $S \rightarrowtail S'$ that rewrites the set of unification
problems until it is in solved form or until a contradiction has been found.
For instance, the variable elimination rule is written as:
\[
   x \stackrel{.}{=} t, S  ~~\rightarrowtail~~  x \stackrel{.}{=} t, [t/x]S   \qquad\qquad{if}~x \not\in t~\text{and}~{x \in S}
\]
Because the whole set is explicitly available, the variable $x$ can be
simultaneously substituted everywhere.

In the above standard unification problem, all variables are implicitly in the same
global scope. In contrast, recent unification algorithms for dependently-typed languages
are more explicit about scopes. For instance, \citet{Reed2009} represents a unification
problem as $\Delta \vdash P$ where $P$ is a set of equations to be solved and $\Delta$ is
a (modal) context. \citet{Abel2011higher} even use multiple contexts within a unification problem.
Such a problem is denoted $\Delta \Vdash \mathcal{K}$ where the meta-context
$\Delta$ contains all the typings of meta-variables in the constraint set
$\mathcal{K}$. The latter consists of constraints like $\Psi \vdash M = N : C$
that are equipped with their individual context $\Psi$. While acurrately tracking
the scoping of regular and meta-variables, this approach is not ideal because it
repeatedly copies contexts when decomposing a unification problem, and later has to 
substitute solutions into these copies.

% %-------------------------------------------------------------------------------
% \subsection{Small-Step Unification}
% % Mention some algorithms for unification for 
% % dependent types that use a small-step approach. Credit them later for some ideas that 
% % we also employ.
% There are literals that make use of small-step unification for dependently typed
% inference and reconstruction algorithms~\cite{Reed2009,Abel2011higher}.
% These approaches collect a list of constraints and process one at a time.
% Similar to DK's algorithm, unification variables are used to represent the types to guess.
% In order to solve unification variables to correctly scoped types
% (typically constrained by the context when the variable is introduced),
% the context information should be kept with the variables.
% Judgments that represent unifications of terms might also require such a context
% to keep the constraint collection well-formed.
% 
% As an example, Abel et al. defines their constraint $K$ by
% \begin{gather*}
% \begin{aligned}
% K &::= {\top} \mid {\bot} &&\text{Trivial constraint and inconsistency.}\\
%     & \ \mid \  \Psi\vdash M = N : C &&\text{Unify term $M$ with $N$.}\\
%     & \ \mid \  \Psi\mid R:A \vdash E = E' &&\text{Unify evaluation context $E$ with $E'$.}\\
%     & \ \mid \  \Psi\vdash u\leftarrow M: C &&\text{Solution for $u$ (metavariable) found.}
% \end{aligned}
% \end{gather*}
% where the unification constraints and metavariable solutions are bound by a proper context.
% A unification problem in their system is described by $\Delta \Vdash \overline{K}$,
% where $\Delta$ is a collection of metavariables with their defined scopes.
% This approach clearly states the scroping of each unification problem,
% therefore rules out invalid instantiations to metavariables.
% 
% However, such ``duplicated contexts'' are not ideal for our formalization.
% Following DK's algorithm, an existential variable $\al$ defined in the context
% could be decomposed into a function type $\al[1] \to \al[2]$,
% so the declaration of $\al$ should be replaced by two declarations $\al[1], \al[2]$.
% Such operation causes all the context that contains $\al$ to change,
% which brings difficulty on the synchronization of multiple contexts,
% as variable declarations are not centralized.
% 
% We find a way of encoding multiple contexts by ``compressing'' them into a single worklist.
% As our type inference rules applied to a judgment,
% we typically keep the context or add variable declarations to the old context
% before analysing its sub-judgment(s).
% This enables us to write the judgment worklist in such a form:
% $$\Gm, \{\text{variable declarations}\}, \{\text{judgment}\},
% \{\text{variable declarations}\}, \{\text{judgment}\}, \ldots .$$
% Typically, we simplify the right-most judgment and push back smaller tasks,
% where the old ``context'' automatically gets inherited.
% When some rules solve or partially solve an existential variable,
% we could easily propagate the solution to all the judgments,
% and safely modify its declaration, such as remove from the worklist.
% New judgment(s) may also introduce new \emph{local} variables by declaring
% that variable right before the new judgment(s).
% After the new judgments are satisfied, these local variables are properly recycled.
% 
% % \jimmy{TODO explain the duplicated context with some examples (their judgment form)}
% % Some issues to point out: Duplicated contexts (rather than shared contexts), which make 
% % the formalization harder since requires ``synchronizing'' things ...
% 
%-------------------------------------------------------------------------------
\subsection{Single-Context Worklist Algorithm for Subtyping}

In recent work, \citep{itp2018} have shown how to mechanise a variant of DK's
subtyping algorithm and shown it correct with respect to DK's declarative
subtyping judgment. This approach overcomes the problem's in DK's formulation
by using a \emph{worklist}-based judgement of the form $$\Gamma \vdash \Omega$$
where $\Omega$ is a worklist, i.e. a sequence, of subtyping problems of the
form $A \leq B$.  The judgement is defined by case analysis on the first
element of $\Omega$ and recursively processes the worklist until it is empty.
Using both a single common ordered context $\Gamma$ and a worklist $\Omega$ greatly
simplifies propagating the instantiation of type variables that happen in one
subtyping problem to the remaining ones.

Unfortunately, this work does not solve all problems. In particular, it has two
major limitations that prevent it from scaling to the whole DK system. 

\paragraph{Scoping Garbage} Firstly, the single common type ordered context 
$\Gamma$ does not accurately reflect the type and unification variables
currently in scope. Instead, it is an overapproximation that steadily accrues
variables, and only drops those unification variables that are instantiated.
In other words, $\Gamma$ contains ``garbage'' that is no longer in scope.
This complicates establishing the relation with the declarative system.


\paragraph{No Inference Judgments} 
Secondly and more importantly, the approach only deals with a judgement for
\emph{checking} whether one type is the subtype of another. While this may not
so be difficult to scale to the \emph{checking} mode of term typing $\Gamma
\vdash e \Leftarrow A$, it clearly lacks the functionality to support the
\emph{inference} mode of term typing $\Gamma \vdash e \Rightarrow A$. Indeed,
the latter requires not only the communication of unification variable
instantiations from one typing problem to another, but also an inferred type.

%-------------------------------------------------------------------------------
\subsection{Algorithmic Type Inference for Higher-Ranked Types: Key Ideas}

Our new algorithmic type system addresses builds on the work above, but
addresses the open problems. 

\paragraph{Scope Tracking}
Firstly, we avoid scoping garbage by blending the ordered context and the
worklist into a single syntactic sort $\Gamma$, our algorithmic worklist. This
algorithmic worklist interleaves (type and term) variables with \emph{work}
like checking or inferring types of expressions. The interleaving keeps track
of the variable scopes in the usual, natural way: each variables is in scope of
anything in front of it in the worklist. If there is nothing in front, the
variable is no longer needed and can be popped from the worklist. This way, no
garbage builds up.

\jimmy{Compare with DK's algo: subsumption rule might cause leak}

\paragraph{Inference Judgements}
To express the DK inference judgements, we use a novel form of work entries in
the worklist: our algorithmic judgement chains $\omega$. In its simplest form,
such a judgement chain is just a check, like a subtyping check $A \leq B$ or a
term typecheck $e \Leftarrow A$.  The non-trivial forms of chains capture an
inference task together with the the work that depends on its outcome. For
instance, a type inference task takes the form $e \Rightarrow_a \omega$. It
denotes that we need to infer the type, say $A$, of expression $e$ and use it
in the chain $\omega$ by subsituting it for the placeholder type variable $a$.


% Our algorithm borrows some ideas from previous work, while adding new ones. 
% A small-step style processing worklists; \emph{Judgment Chains}; others. 

% Importantly we now deal with inference judgement.

