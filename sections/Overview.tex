\section{Overview}

\bruno{I think some text/ideas-for-text can be borrowed from ITP.}

\subsection{DK's Declarative System}
Subtyping and Typing. \jimmy{Can you prepare the figures. Subtyping is already in ITP.}

\paragraph{Metatheory} Talk about the lemmas that are false, show counterexample. 

\paragraph{Other Challenges} Scoping, Output Contexts?

\subsection{Small-Step Unification} Mention some algorithms for unification for 
dependent types that use a small-step approach. Credit them later for some ideas that 
we also employ.

Some issues to point out: Duplicated contexts (rather than shared contexts), which make 
the formalization harder since requires ``synchronizing'' things ...

%-------------------------------------------------------------------------------
\subsection{Worklist Algorithm for Subtyping}

In recent work~\cite{itp2018}, we have shown how to mechanise a variant of DK's
subtyping algorithm and shown it correct with respect to DK's declarative
subtyping judgment. This approach overcomes the problem's in DK's formulation
by using a \emph{worklist}-based judgement of the form $\Gamma \vdash \Omega$
where $\Omega$ is a worklist, i.e. a sequence, of subtyping problems of the
form $A \leq B$.  The judgement is defined by case analysis on the first
element of $\Omega$ and recursively processes the worklist until it is empty.
Using both a single common environment $\Gamma$ and a worklist $\Omega$ greatly
simplifies propagating the instantiation of type variables that happen in one
subtyping problem to the remaining ones.

Unfortunately, this work does not solve all problems. In particular, it has two
major limitations that prevent it from scaling to the whole DK system. 

\paragraph{Scoping Garbage} Firstly, the single common type environment
$\Gamma$ does not accurately reflect the type and unification variables
currently in scope. Instead, it is an overapproximation that steadily accrues
variables, and only drops those unification variables that are instantiated.
In other words, $\Gamma$ contains ``garbage'' that is no longer in scope.
This complicates establishing the relation with the declarative system.


\paragraph{No Inference Judgments} 
Secondly and more importantly, the approach only deals with a judgement for
\emph{checking} whether one type is the subtype of another. While this may not
so be difficult to scale to the \emph{checking} mode of term typing $\Gamma
\vdash e \Leftarrow A$, it clearly lacks the functionality to support the
\emph{inference} mode of term typing $\Gamma \vdash e \Rightarrow A$. Indeed,
the latter requires not only the communication of unification variable
instantiations from one typing problem to another, but also an inferred type.

%-------------------------------------------------------------------------------
\subsection{Algorithmic Type Inference for Higher-Ranked Types: Key Ideas}

Our algorithm borrows some ideas from previous work, while adding new ones. 
A small-step style processing worklists; \emph{Judgment Chains}; others. 

