\section{Conclusion}

This is the conclusion.

% \jimmy{Discuss the efficient algorithm and its equivalence}
% The algorithm is already efficient

Outline:
\begin{itemize}
    \item difficulties following DK's formalization
    \item a blend of ideas from various sources DK's ordered contexts
    \item list of judgements + some novel ideas
    \item lessons learned: problems identified in proofs and scoping
    \item Future work
\end{itemize}



\paragraph{Future work}
\begin{itemize}
    \item Scoped type variables?
    \item Go beyond predicative instantiations
    \item More complicated features, such as dependent type
\end{itemize}

We have not yet explore lexically-scoped type variables~\cite{jones2003lexically}
as a practical extension to our system.
Formalization of a different variable scoping scheme is also
a challenging task for formal proofs via proof assistants.

Moreover, the garbage collection process for type variables and
existential variables do worth further attention:
the inference of $\lam x$ leaves an unsolved $\al$ in the context,
which we simply remove the variable after it is no longer refered to.
Is it possible that we infer better types by making better use of that piece of information?
How do we deal with, at least some ``trivial'' impredicative instantiations?

Last but not least, a further expectation is that our system would
adapt to various complicated type inference algorithms,
such as ones for dependent types.
