%% For double-blind review submission, w/o CCS and ACM Reference (max
%% submission space)
%\documentclass[acmsmall,review]{acmart}\settopmatter{printfolios=true}
%%\documentclass[acmsmall,review,anonymous]{acmart}\settopmatter{printfolios=true,printccs=false,printacmref=false}
%% For double-blind review submission, w/ CCS and ACM Reference
%\documentclass[acmsmall,review,anonymous]{acmart}\settopmatter{printfolios=true}
%% For single-blind review submission, w/o CCS and ACM Reference (max submission space)
%\documentclass[acmsmall,review]{acmart}\settopmatter{printfolios=true,printccs=false,printacmref=false}
%% For single-blind review submission, w/ CCS and ACM Reference
%\documentclass[acmsmall,review]{acmart}\settopmatter{printfolios=true}
%% For final camera-ready submission, w/ required CCS and ACM Reference
\documentclass[acmsmall]{acmart}\settopmatter{}

%% Remove almost every footnotes and page numbers, for grammar check use.
%\documentclass[acmsmall,anonymous]{acmart}\settopmatter{printfolios=true,printccs=false,printacmref=false}
%\settopmatter{printacmref=false} % Removes citation information below abstract
%\renewcommand\footnotetextcopyrightpermission[1]{} % removes footnote with conference information in first column
%\pagestyle{plain} % removes running headers
%\pagenumbering{gobble}


%% Journal information
%% Supplied to authors by publisher for camera-ready submission;
%% use defaults for review submission.
%\acmJournal{PACMPL}
%\acmVolume{1}
%\acmNumber{ICFP} % CONF = POPL or ICFP or OOPSLA
%\acmArticle{1}
%\acmYear{2018}
%\acmMonth{1}
%\acmDOI{} % \acmDOI{10.1145/nnnnnnn.nnnnnnn}
%\startPage{1}

%% Copyright information
%% Supplied to authors (based on authors' rights management selection;
%% see authors.acm.org) by publisher for camera-ready submission;
%% use 'none' for review submission.
%\setcopyright{none}
%\setcopyright{acmcopyright}
%\setcopyright{acmlicensed}
%\setcopyright{rightsretained}
%\copyrightyear{2018}           %% If different from \acmYear

%% Bibliography style
\bibliographystyle{ACM-Reference-Format}
%% Citation style
%% Note: author/year citations are required for papers published as an
%% issue of PACMPL.
\citestyle{acmauthoryear}   %% For author/year citations


%%%%%%%%%%%%%%%%%%%%%%%%%%%%%%%%%%%%%%%%%%%%%%%%%%%%%%%%%%%%%%%%%%%%%%
%% Note: Authors migrating a paper from PACMPL format to traditional
%% SIGPLAN proceedings format must update the '\documentclass' and
%% topmatter commands above; see 'acmart-sigplanproc-template.tex'.
%%%%%%%%%%%%%%%%%%%%%%%%%%%%%%%%%%%%%%%%%%%%%%%%%%%%%%%%%%%%%%%%%%%%%%


%% Some recommended packages.
\usepackage{booktabs}   %% For formal tables:
                        %% http://ctan.org/pkg/booktabs
\usepackage{subcaption} %% For complex figures with subfigures/subcaptions
                        %% http://ctan.org/pkg/subcaption

\usepackage{amsmath}
\usepackage{amssymb}
\usepackage{amstext}
\usepackage{enumerate}
%\usepackage{enumitem}

\usepackage{multirow}
\usepackage{makecell}
\usepackage{color}

\usepackage{xparse}
\usepackage{listings}
\usepackage{lstabella}

\usepackage{mathpartir}
\usepackage{comment}

\usepackage{mathtools}
\usepackage{graphicx}

\usepackage{pdflscape}
\usepackage{afterpage}
% \usepackage{rotating}

%\usepackage{tikz}


\newcommand\mynote[3]{\textcolor{#2}{#1: #3}}
\newcommand\bruno[1]{\mynote{Bruno}{red}{#1}}
\newcommand\jimmy[1]{\mynote{Jimmy}{blue}{#1}}
\newcommand\tom[1]{\mynote{Tom}{magenta}{#1}}

\newcommand\wh{\widehat}

\newcommand\jg{\omega} % judgment
\newcommand\toto{\rightrightarrows}
\newcommand\To{\Rightarrow}
\newcommand\Lto{\Leftarrow}
\newcommand\TTo{\mathrel{\mathrlap{\To}\mkern1mu\To}}
\newcommand\sto{\rightsquigarrow}
\newcommand\tto{\rightarrowtail}

\newcommand\Gm{\Gamma}
\newcommand\Om{\Omega}

\newcommand\nil\cdot

% just to fool TexStudio
\providecommand\inferrule{}

\newcommand\rto{\longrightarrow}
\newcommand\redto{\longrightarrow^*}
\newcommand\rrule[1]{~\longrightarrow_{\makebox[0pt][l]{$\scriptstyle#1$}\hphantom{00}}}

\newcommand\jExt{\rightharpoonup}

\newcommand{\tRed}[1]{\textcolor{red}{#1}}

\newcommand{\lam}{} %[2]{\lambda {#1}.~{#2}}
\RenewDocumentCommand \lam {O{x} m} {\lambda {#1}.~{#2}}

\newcommand{\all}{} %[2]{\forall {#1}.~{#2}}
\RenewDocumentCommand \all {O{a} m} {\forall {#1}.~{#2}}

\newcommand{\appInf}[3]{{#1}\bullet{#2}\TTo{#3}}
\newcommand{\appInfAlg}{} %[4]{{#1}\bullet{#2}\TTo_{#3} {#4}}
\RenewDocumentCommand \appInfAlg {m m O{a} O{\jg}} {{#1}\bullet{#2}\TTo_{#3} {#4}}

\newcommand\al{}
\newcommand\bt{}

\RenewDocumentCommand \al {O{}} {\wh\alpha_{#1}}
\RenewDocumentCommand \bt {O{}} {\wh\beta_{#1}}

\newcommand{\blue}[1]{\textcolor{blue}{#1}}





% Appendix only
%\usepackage[28-]{pagesel}
% Main article with references
%\usepackage[-27]{pagesel}


%%% The following is specific to ICFP '19 and the paper
%%% 'A Mechanical Formalization of Higher-Ranked Polymorphic Type Inference'
%%% by Jinxu Zhao, Bruno C. d. S. Oliveira, and Tom Schrijvers.
%%%
\setcopyright{acmcopyright}
\acmPrice{}
\acmDOI{10.1145/3341716}
\acmYear{2019}
\copyrightyear{2019}
\acmJournal{PACMPL}
\acmVolume{3}
\acmNumber{ICFP}
\acmArticle{112}
\acmMonth{8}


\begin{document}

%% Title information
\title{A Mechanical Formalization of Higher-Ranked Polymorphic
  Type Inference}         %% [Short Title] is optional;
                                        %% when present, will be used in
                                        %% header instead of Full Title.
%\titlenote{with title note}             %% \titlenote is optional;
                                        %% can be repeated if necessary;
                                        %% contents suppressed with 'anonymous'
%\subtitle{Subtitle}                     %% \subtitle is optional
%\subtitlenote{with subtitle note}       %% \subtitlenote is optional;
                                        %% can be repeated if necessary;
                                        %% contents suppressed with 'anonymous'


%% Author information
%% Contents and number of authors suppressed with 'anonymous'.
%% Each author should be introduced by \author, followed by
%% \authornote (optional), \orcid (optional), \affiliation, and
%% \email.
%% An author may have multiple affiliations and/or emails; repeat the
%% appropriate command.
%% Many elements are not rendered, but should be provided for metadata
%% extraction tools.

%% Author with single affiliation.
\author{Jinxu Zhao}
%\authornote{with author1 note}          %% \authornote is optional;
                                        %% can be repeated if necessary
%\orcid{nnnn-nnnn-nnnn-nnnn}             %% \orcid is optional
\affiliation{
  %\position{Position1}
  \department{Department of Computer Science}              %% \department is recommended
  \institution{The University of Hong Kong}            %% \institution is required
  %\streetaddress{Pokfulam}
  \city{Hong Kong}
  %\state{State1}
  %\postcode{Post-Code1}
  \country{China}                    %% \country is recommended
}
\email{jxzhao@cs.hku.hk}          %% \email is recommended

%% Author with two affiliations and emails.
\author{Bruno C. d. S. Oliveira}
%\authornote{with author1 note}          %% \authornote is optional;
%% can be repeated if necessary
%\orcid{nnnn-nnnn-nnnn-nnnn}             %% \orcid is optional
\affiliation{
	%\position{Position1}
	\department{Department of Computer Science}              %% \department is recommended
	\institution{The University of Hong Kong}            %% \institution is required
	%\streetaddress{Pokfulam}
	\city{Hong Kong}
	%\state{State1}
	%\postcode{Post-Code1}
	\country{China}                    %% \country is recommended
}
\email{bruno@cs.hku.hk}          %% \email is recommended

%% Author with two affiliations and emails.
\author{Tom Schrijvers}
%\authornote{with author2 note}          %% \authornote is optional;
%% can be repeated if necessary
%\orcid{nnnn-nnnn-nnnn-nnnn}             %% \orcid is optional
\affiliation{
	%\position{Position2a}
	\department{Department of Computer Science}             %% \department is recommended
	\institution{KU Leuven}           %% \institution is required
	%\streetaddress{Street2a Address2a}
	\city{Leuven}
	%\state{State2a}
	%\postcode{Post-Code2a}
	\country{Belgium}                   %% \country is recommended
}
\email{tom.schrijvers@cs.kuleuven.be}


%% Abstract
%% Note: \begin{abstract}...\end{abstract} environment must come
%% before \maketitle command
\begin{abstract}
Modern functional programming languages, such as Haskell or OCaml,
use sophisticated forms of type inference. While an
important topic in the Programming Languages research, there is little
work on the mechanization of the metatheory of type inference in
theorem provers. In particular we are unaware of any complete
formalization of the type inference algorithms that are the backbone of modern 
functional languages. 

This paper presents the first full mechanical formalization of the metatheory for
higher-ranked polymorphic type inference. The system that we formalize
is the bidirectional type system by Dunfield and
Krishnaswami (DK). The DK type system has two variants (a declarative
and an algorithmic one) that have been \emph{manually} proven
\emph{sound}, \emph{complete} and \emph{decidable}. We present 
a mechanical formalization in the Abella theorem prover 
of DK's declarative type system with a novel algorithmic system. We have
a few reasons to use a new algorithm. Firstly, our new algorithm
employs \emph{worklist judgments}, which precisely capture the
scope of variables and simplify the formalization of scoping in a
theorem prover. Secondly, while DK's original formalization comes with very
well-written manual proofs, there are several details missing and some
incorrect proofs,
which complicate the task of writing a mechanized proof. 
Despite the use of a different algorithm we prove the
same results as DK, although with significantly different proofs and
proof techniques. Since such type inference algorithms are quite
subtle and have a complex metatheory, mechanical
formalizations are an important advance in type-inference research.
\end{abstract}


%% 2012 ACM Computing Classification System (CSS) concepts
%% Generate at 'http://dl.acm.org/ccs/ccs.cfm'.
\begin{CCSXML}
<ccs2012>
<concept>
<concept_id>10011007.10011006.10011008</concept_id>
<concept_desc>Software and its engineering~General programming languages</concept_desc>
<concept_significance>500</concept_significance>
</concept>
<concept>
<concept_id>10003456.10003457.10003521.10003525</concept_id>
<concept_desc>Social and professional topics~History of programming languages</concept_desc>
<concept_significance>300</concept_significance>
</concept>
</ccs2012>
\end{CCSXML}

\ccsdesc[500]{Software and its engineering~General programming languages}
\ccsdesc[300]{Social and professional topics~History of programming languages}
%% End of generated code


%% Keywords
%% comma separated list
\keywords{type inference, higher-rank polymorphism, mechanization}  %% \keywords are mandatory in final camera-ready submission


%% \maketitle
%% Note: \maketitle command must come after title commands, author
%% commands, abstract environment, Computing Classification System
%% environment and commands, and keywords command.
\maketitle

\section{Introduction}

Modern functional programming languages, such as Haskell or OCaml,
use sophisticated forms of type inference. The type systems of these languages are
descendants of
Hindley-Milner~\cite{hindley1969principal,milner1978theory,damas1982principal}, 
which was revolutionary at the
time in allowing type-inference to proceed without any type
annotation. The traditional Hindley-Milner type system supports
top-level \emph{implicit (parametric) polymorphism}~\cite{reynolds1983types}. With implicit
polymorphism, type arguments of polymorphic functions are
automatically instantiated. Thus implicit polymorphism and the absence of
type annotations mean that the Hindley-Milner type system 
strikes a great balance between expressive power and usability. 

As functional languages evolved the need for more expressive
power has motivated language designers to look beyond Hindley-Milner.
In particular one popular direction is to allow 
\emph{higher-ranked polymorphism} where polymorphic types can
occur anywhere in a type signature.  
An important challenge is that full type inference for higher-ranked
polymorphism is know to be undecidable~\cite{wells1999typability}. Therefore some type
annotations are necessary to guide type inference. In response to this 
challenge several decidable type systems requiring some annotations 
have been proposed~\cite{,dunfield2013complete,jones2007practical,Serrano2018,le2003ml,leijen2008hmf,vytiniotis2008fph}.
Two closely related type systems that 
support \emph{predicative} higher-ranked type inference were proposed 
by Peyton Jones et al.~\cite{jones2007practical} and Dunfield and
Krishnaswami~\cite{dunfield2013complete} (henceforth denoted as DK). 
These type systems are
popular among language designers and their ideas have been adopted by
several modern functional languages, including Haskell, PureScript~\cite{PureScript} and
Unison~\cite{Unison} among others.
In those type systems
type annotations are required for polymorphic arguments of functions,
but other type annotations can be omitted. A canonical example (here written in Haskell) is:
\begin{verbatim}
    hpoly = (\f :: forall a. a -> a) -> (f 1, f 'c')
\end{verbatim}
The function \verb|hpoly| cannot be
type-checked in Hindley-Milner.  The type of \verb|hpoly| is the rank-2 type:
\verb|(forall a. a -> a) -> (Int, Char)|. Notably (and unlike
Hindley-Milner) the lambda argument \verb|f| requires a
\emph{polymorphic} type annotation.
This annotation is needed because the single universal quantifier
does not appear at the top-level. Instead it is used to quantify a
type variable \verb|a| used in the first argument of the
function. 
Despite these additional annotations, Peyton Jones et al. and DK's
type inference algorithms preserve many of the desirable properties 
of Hindley-Milner. For example the applications of \verb|f| implicitly 
instantiate the polymorphic type arguments of \verb|f|.

Although type inference is important in practice and receives a lot of
attention in
academic research, there is little work on mechanically formalizing
such advanced forms of type inference in theorem provers.
The remarkable exception is work done on the formalization of 
certain parts of Hindley-Milner type inference~\cite{naraschewski1999type,
dubois2000proving,dubois1999certification,urban2008nominal,
garrigue2015certified}. However
there is still no formalization of the higher-ranked type systems
that are employed by modern languages like Haskell.
This is at
odds with the current trend of mechanical formalizations in
programming language research. In particular both the POPLMark
challenge~\cite{aydemir2005mechanized} and
CompCert ~\cite{leroy2012compcert} have significantly promoted
the use of theorem provers to model various aspects of programming
languages. Today papers in various programming language venues routinely
use theorem provers to mechanically formalize: \emph{dynamic and
  static semantics} and their correctness properties~\cite{aydemir2008engineering},
\emph{compiler correctness}~\cite{leroy2012compcert}, \emph{correctness of
  optimizations}~\cite{Bertot04}, \emph{program analisys}~\cite{Chang2006}
or proofs involving \emph{logical relations}~\cite{abel2018}. The
main argument for mechanical formalizations is a simple one. Proofs
for programming languages tend to be \emph{long}, \emph{tedious} and
\emph{error-prone}. In such proofs it is very easy to make mistakes
that may invalidate the whole development. Furthermore, readers and
reviewers often do not have time to look at the proofs carefully to
check their correctness. Therefore errors can go unnoticed for a
long time.  Mechanical formalizations provide, in principle, a natural
solution for these problems. Theorem provers can automatically check and
validate the proofs, which removes the burden from checking from both
the person doing the proofs as well as readers or reviewers.

This paper presents the first full mechanical formalization of the
metatheory for higher-ranked polymorphic type inference.
The system
that we formalize is the bidirectional type system by Dunfield and
Krishnaswami (DK)~\cite{dunfield2013complete}.
We chose DK's type system because it is
quite elegant, well-documented and it comes with detailed manually
written proofs. Furthermore the system is adopted in practice by a few
real implementations of functional languages, including PureScript and
Unison. The DK type system has two variants: a declarative
and an algorithmic one. The two variants have been been
\emph{manually} proved to be \emph{sound}, \emph{complete} and
\emph{decidable}.
We present a mechanical formalization in the Abella theorem prover for
DK's declarative type system using a different algorithm. While our
initial goal was to formalize both DK's declarative and algorithmic
versions, we faced technical challenges with the latter, prompting us to find
an alternative formulation.

The first challenge that we faced were missing details as well as
a few incorrect proofs and lemmas in DK's formalization. While DK's
original formalization comes with very well written manual proofs,
there are still several details missing. These complicate the task of
writing a mechanically verified proof. Moreover some proofs and
lemmas are wrong and, in some cases, it is not clear to us how to fix them.
%%\footnote{
%%Perhaps, with some additional thought, a work-around can be found, but, as our algorithm
%%differs significantly from DK's, it would not further our mechanisation goal and so we did not pursue this.}
Despite the problems in DK's manual formalization,
we believe that these problems do not
invalidate their work and that their results are still true. In fact we have nothing but praise for their detailed
and clearly written metatheory and proofs, which provided invaluable
help to our own work.
We expect that for most non-trivial manual
proofs similar problems exist, so this should not be understood as a sign of sloppiness
on their part. Instead it should be an indicator that reinforces the arguments
for mechanical formalizations: manual formalizations are error-prone due to the multiple
tedious details involved in them.
There are several other examples of manual formalizations that were found to have
similar problems. For example, Klein et al.~\cite{KleinRunYourResearch}
mechanized formalizations
in Redex for nine ICFP 2009 papers and all were found to have mistakes.

Another challenge was variable binding. Type inference algorithms
typically do not rely simply on local environments but instead
propagate information across judgements. While local environments are
well-studied in mechanical formalizations, there is little work on how
to deal with the complex forms of binding employed by type inference algorithms
in theorem provers. To
keep track of variable scoping, DK's algorithmic version employs input
and output contexts to track information that is discovered throught
type inference. However modelling output contexts in a theorem prover
is non-trivial.

Due to those two challenges, our work takes a different approach by refining and
extending the idea of \emph{worklists judgments}~\cite{itp2018},
proposed recently to mechanically formalize an algorithm for
\emph{polymorphic subtyping}~\cite{odersky1996putting}. A key innovation in our work is how
to adapt the idea of worklist judgments to
\emph{inference judgements}, which are not needed for polymorphic
subtyping, but are necessary for type-inference.  The idea is to use a \emph{continuation
passing style} to enable the transfer of inferred information across
judgements. A further refinement to the idea of worklist judgements is
the \emph{unification between ordered
  contexts~\cite{gundry2010type,dunfield2013complete} and worklists}.  This
enables precise scope tracking of free variables in
judgements. Furthermore it avoids the duplication of context
information across judgments in worklists that occurs in other
techniques~\cite{Reed2009,Abel2011higher}.
Despite the use of a different algorithm we prove the
same results as DK, although with significantly different proofs and
proof techniques. The calculus and its metatheory
have been fully formalized in the Abella theorem prover~\cite{AbellaDesc}.

In summary, the contributions of this paper are:

\begin{itemize}

\item {\bf A full mechanical formalization of type inference with
  higher-ranked types:} Our work presents the first full mechanical formalization
  for type inference of higher ranked types. The formalization is done in the
  Abella theorem prover~\cite{AbellaDesc} and it is available
  online.\footnote{{\bf Note to reviewers:} Due to the anonymous submission process
  it is sent as part of the supplementary materials.}

\item {\bf A new algorithm for DK's type system:} Our work proposes a novel algorithm that implements
  DK's declarative bidirectional type system. We prove
  \emph{soundness}, \emph{completeness} and
  \emph{decidability}. 

\item {\bf Worklists with inference judgments:} One technical contribution is the
  support for inference judgments using worklists. The idea is to
  use a continuation passing style to enable the transfer of inferred information across
  judgements. 

\item {\bf Unification of worklists and contexts:} Another technical contribution is the unification
  between ordered contexts and worklists. This enables precise scope tracking
  of variables in judgements, and avoids the duplication of context information across
  judgments in worklists.

\begin{comment}
\jimmy{Notes @20190211 4 points of novalty:\\
1) Dealing with inference judgements and CPS-style chains\\
2) The form of the judgment itself with a single shared context\\
3) The way we deal with scope (which may follow from 2)\\
4) Immediate substitution (judgment list)
}
\end{comment}

\end{itemize}

\section{Overview}

This section starts by introducing DK's declarative type system. Then
it discusses several techniques that have been used in algorithmic
formulations, and which have influenced our own algorithmic design.
Finally we introduce the novelties of our new algorithm.
In particular the support for inference judgements in
judgement worklists, and a new form of judgement worklist
that the unifies \emph{ordered contexts} and the worklists themselves. 

\subsection{DK's Declarative System}
% Subtyping and Typing. \jimmy{Can you prepare the figures. Subtyping is already in ITP.}

\begin{figure}[t]
\[
\begin{array}{l@{\qquad}lcl}
\text{Type variables}\qquad&a, b
\\[3mm]
\text{Types}\qquad&A, B, C &::=&\quad 1 \mid a \mid \forall a. A \mid A\to B\\
\text{Monotypes}\qquad&\tau,\sigma &::=&\quad 1 \mid a \mid \tau\to \sigma
\\[3mm]
\text{Expressions}\qquad&e &::=&\quad x \mid () \mid \lam e \mid e_1~e_2 \mid (e:A)
\\[3mm]
\text{Contexts}\qquad&\Psi &::=&\quad \nil \mid \Psi, a \mid \Psi, x:A
\end{array}
\]
\caption{Syntax of Declarative System}\label{fig:decl:syntax}
\end{figure}

\paragraph{Syntax.}
The syntax of DK's declarative system~\cite{dunfield2013complete} is shown in Figure~\ref{fig:decl:syntax}.
A declarative type $A$ is either the unit type $1$, a type variable $a$,
a universal quantification $\forall a. A$ or a function type $A \to B$.
Nested universal quantifiers are allowed for types,
but monotypes $\tau$ do not have any universal quantifier.
Terms include a unit term $()$, variables $x$, lambda-functions $\lam e$,
applications $e_1~e_2$ and annotations $(e:A)$.
Contexts $\Psi$ are sequences of type variable declarations and
term variables with its type declared $x:A$.

\begin{figure}[t]
%\centering \framebox{$\Psi \vdash A$}
%\begin{gather*}
%\inferrule*[right=$\mathtt{wf_d unit}$]
%    {~}{\Psi\vdash 1}
%\qquad
%\inferrule*[right=$\mathtt{wf_d var}$]
%    {a\in\Psi}{\Psi\vdash a}
%\qquad
%\inferrule*[right=$\mathtt{wf_d{\to}}$]
%    {\Psi\vdash A\quad \Psi\vdash B}
%    {\Psi\vdash A\to B}
%\qquad
%\inferrule*[right=$\mathtt{wf_d\forall}$]
%    {\Psi, a\vdash A}
%    {\Psi\vdash \forall a. A}
%\end{gather*}

\centering \framebox{$\Psi \vdash A \le B$}
\begin{gather*}
\inferrule*[right=$\mathtt{{\le}Var}$]
    {a\in\Psi}{\Psi\vdash a\le a}
\qquad
\inferrule*[right=$\mathtt{{\le}Unit}$]
    {~}{\Psi \vdash 1 \le 1}
\qquad
\inferrule*[right=$\mathtt{{\le}{\to}}$]
    {\Psi \vdash B_1 \le A_1 \quad \Psi \vdash A_2 \le B_2}
    {\Psi\vdash A_1\to A_2 \le B_1\to B_2}
\\
\inferrule*[right=$\mathtt{{\le}\forall L}$]
    {\Psi\vdash \tau \quad \Psi\vdash [\tau/a] A \le B}
    {\Psi\vdash \forall a. A \le B}
\qquad
\inferrule*[right=$\mathtt{{\le}\forall R}$]
    {\Psi, a\vdash A\le B}
    {\Psi\vdash A \le \forall a. B}
\end{gather*}
\caption{%Well-formedness of Declarative Types and 
Declarative Subtyping}\label{fig:decl:sub}
\end{figure}

\begin{figure}[t]
\begin{tabular}{rl}
    \framebox{$\Psi \vdash A \Lto B$} & $e$ checks against input type $A$.\\[0.5mm]
    \framebox{$\Psi \vdash A \To B$} & $e$ synthesizes output type $A$.\\[0.5mm]
    \framebox{$\Psi \vdash \appInf{A}{e}{C}$} & Applying a function of type $A$ to $e$ synthesizes type $C$.
\end{tabular}
\begin{gather*}
\inferrule*[right=$\mathtt{DeclVar}$]
    {(x:A)\in\Psi}{\Psi\vdash x\To A}
\qquad
\inferrule*[right=$\mathtt{DeclSub}$]
%e \neq \lam e' \quad B \neq \all B' \quad 
    {\Psi\vdash e\To A \quad \Psi\vdash A\le B}
    {\Psi \vdash e\Lto B}
\\
\inferrule*[right=$\mathtt{DeclAnno}$]
    {\Psi \vdash A \quad \Psi\vdash e\Lto A}
    {\Psi\vdash (e:A)\To A}
\qquad
\inferrule*[right=$\mathtt{Decl1I}$]
    {~}{\Psi\vdash () \Lto 1}
\qquad
\inferrule*[right=$\mathtt{Decl1I{\To}}$]
    {~}{\Psi\vdash () \To 1}
\\
\inferrule*[right=$\mathtt{Decl\forall I}$]
    {\Psi,a \vdash e \Lto A}
    {\Psi\vdash e\Lto \all A}
\qquad
\inferrule*[right=$\mathtt{Decl\forall App}$]
    {\Psi \vdash \tau \quad \Psi\vdash \appInf{[\tau/a]A}{e}{C} }
    {\Psi\vdash \appInf{\all A}{e}{C}}
\\
\inferrule*[right=$\mathtt{Decl{\to}I}$]
    {\Psi,x:A \vdash e\Lto B}
    {\Psi\vdash \lam e \Lto A \to B}
\qquad
\inferrule*[right=$\mathtt{Decl{\to}I{\To}}$]
    {\Psi\vdash \sigma\to\tau \quad \Psi,x:\sigma \vdash e\Lto \tau}
    {\Psi\vdash \lam e \To \sigma\to\tau}
\\
\inferrule*[right=$\mathtt{Decl{\to} E}$]
    {\Psi\vdash e_1\To A \quad \Psi\vdash \appInf{A}{e_2}{C}}
    {\Psi\vdash e_1~e_2 \To C}
\qquad
\inferrule*[right=$\mathtt{Decl{\to}App}$]
    {\Psi\vdash e \Lto A}
    {\Psi\vdash \appInf{A \to C}{e}{C}}
\end{gather*}
\caption{Declarative Typing}\label{fig:decl:typing}
\end{figure}

%The DK subtyping relation was adopted from \citet{odersky1996putting}.

\paragraph{Declarative Subtyping}
Figure~\ref{fig:decl:sub} shows DK's declarative subtyping judgment $\Psi \vdash A \le B$,
which was adopted from \citet{odersky1996putting}. It compares the
degree of polymorphism between between $A$ and $B$ in DK's implicit polymorphic type system. 
Essentially, if $A$ can always be instantiated to match any instantiation of $B$,
then A is ``at least as polymorphic as'' $B$. We also 
say that $A$ is ``more polymorphic than'' $B$ and write $A \le B$.

Subtyping rules $\mathtt{{\le}Var}$, $\mathtt{{\le}Unit}$ and $\mathtt{{\le}{\to}}$
handle simple cases that do not involve universal quantifiers.
The subtyping rule for function types $\mathtt{{\le}{\to}}$ is standard,
being covariant on the return type and contravariant on the argument type.
Rule $\mathtt{{\le}\forall R}$ states that if $A$ is a subtype of $B$
in the context $\Psi, a$, where $a$ is fresh in $A$, then $A\le\all B$.
Intuitively, if $A$ is more general than $\all B$ (where the universal quantifier
already indicates that $\all B$ is a general type),
then $A$ must instantiate to $[\tau/a]B$ for every $\tau$.

The most interesting rule is $\mathtt{{\le}\forall L}$.
If some instantiation of $\all A$, $[\tau/a]A$, is a subtype of $B$,
then $\all A \le B$.
The monotype $\tau$ we used to instantiate $a$ is \emph{guessed} in this
declarative rule, but the algorithmic system does not guess and defers the
instantiation until it can determine the monotype deterministically.
The fact that $\tau$ is a monotype rules out the possibility of
polymorphic (or impredicative) instantiation.
However this restriction ensures that the subtyping relation remains
decidable. Allowing an arbitrary type (rather than a monotype) in rule $\mathtt{{\le}\forall L}$
is know to give rise to an undecidable subtyping relation~\cite{tiuryn1996subtyping}.
\citet{jones2007practical} also impose the restriction of
predicative instantiation in their type system.
Both systems are adopted by several practical programming languages.

\paragraph{Declarative Typing}
The bidirectional type system, shown in Figure~\ref{fig:decl:typing}, has three judgments.
The checking judgment $\Psi\vdash e\Lto A$ checks expression $e$ against the type $A$ in the context $\Psi$.
The synthesis judgment $\Psi\vdash e\To A$ synthesizes the type $A$ of expression $e$ in the context $\Psi$.
The application judgment $\Psi\vdash \appInf{A}{e}{C}$ synthesizes the type $C$ of the application of a function of type $A$
(which could be polymorphic) to the argument $e$.

Many rules are standard.
Rule $\mathtt{DeclVar}$ looks up term variables in the context.
Rules $\mathtt{Decl1I}$ and $\mathtt{Decl1I{\To}}$ respectively check and synthesize the unit type.
Rule $\mathtt{DeclAnno}$ synthesizes the annotated type $A$ of the annotated expression $(e:A)$
and checks that $e$ has type $A$.
Checking an expression $e$ against a polymorphic type $\all A$ in the context $\Psi$ succeeds
if $e$ checks against $A$ in the extended context $(\Psi, a)$.
The subsumption rule $\mathtt{DeclSub}$ depends on the subtyping relation,
and changes mode from checking to synthesis: if $e$ synthesizes type $A$ and $A\le B$
($A$ is more polymorphic than $B$), then $e$ checks against $B$.
If a checking problem does not match any other rules,
this rule can be applied to synthesize a type instead and then
check whether the synthesized type entails the checked type.
Lambda abstractions are the hardest construct of the bidirectional
type system to deal with. 
Checking $\lam e$ against function type $A\to B$ is easy:
we check the body $e$ against $B$ in the context extended with $x:A$.
However, synthesizing a lambda-function is a lot less easy, and 
this type system only synthesizes monotypes $\sigma\to\tau$.

Application $e_1~e_2$ is handled by Rule $\mathtt{Decl{\to}E}$,
which first synthesizes the type $A$ of the function $e_1$.
If $A$ is a function type $B\to C$, Rule $\mathtt{Decl{\to}App}$ is applied;
it checks the argument $e_2$ against $B$ and returns type $C$.
The synthesized type of function $e_1$ can also be polymorphic, of the form $\all A$.
In that case, we instantiate $A$ to $[\tau/a]A$ with a monotype $\tau$ % (which is also guessed)
using according to Rule $\mathtt{Decl{\to}I{\To}}$.
If $[\tau/a]A$ is a function type, Rule $\mathtt{Decl{\to}App}$ proceeds;
if $[\tau/a]A$ has another universal quantifier, Rule $\mathtt{Decl{\to}I{\To}}$ is recursively applied.

\paragraph{Overlapping Rules}
Some of the declarative rules overlaps with each other.
Declarative subtyping rules $\mathtt{{\le}\forall L}$ and $\mathtt{{\le}\forall R}$
both match the conclusion $\Psi\vdash \all A \le \all B$.
In such case, choosing $\mathtt{{\le}\forall R}$ first is always better,
since we introduce the type variable $a$ to the context earlier,
which gives more flexibility on the choice of $\tau$.
Declarative typing rule $\mathtt{DeclSub}$ overlaps with
both $\mathtt{Decl\forall I}$ and $\mathtt{Decl\to I}$.
However, we could argue that more specific rules are always the best choices,
i.e. $\mathtt{Decl\forall I}$ and $\mathtt{Decl\to I}$ should have
higher priority to $\mathtt{DeclSub}$. We will come back to this
topic in Section~\ref{sec:metatheory}.
\begin{comment}
For example, $\Psi\vdash \lam x \Lto \all a\to a$ succeeds if derived from
Rule $\mathtt{Decl\forall I}$, but fails when applied to $\mathtt{DeclSub}$:
$$
\inferrule*[right={$\mathtt{Decl\forall I}$}]
	{\inferrule*[Right={$\mathtt{Decl\to I}$}]
		{\Psi,a,x:a \vdash x \Lto a}
		{\Psi,a \vdash \lam x \Lto a \to a}
	}
	{\Psi\vdash \lam x \Lto \all a \to a}
$$
$$
\inferrule*[right={$\mathtt{DeclSub}$}]
	{
		\inferrule*[right=$\mathtt{Decl\to I\To}$]
			{\Psi \vdash \blue\sigma\to \blue\tau \quad \Psi,x:\blue\sigma\vdash e \Lto \blue\tau}
			{\Psi \vdash \lam x \To \blue\sigma\to \blue\tau}\quad
		\inferrule*[Right=$\mathtt{{\le}\forall R}$]
			{\inferrule*[Right=$\mathtt{{\le}{\to}}$]
				{
					\inferrule*[Right=$?$]
						{\text{Impossible! }\blue\sigma \neq a}
						{\Psi,a \vdash a \le \blue\sigma}
					\quad \Psi,a \vdash \blue\tau \le a
				}
				{\Psi,a \vdash \blue\sigma\to \blue\tau \le a \to a}
			}
			{\Psi\vdash \blue\sigma\to \blue\tau\le \all a \to a}
	}
{\Psi\vdash \lam x \Lto \all a \to a}
$$

Rule $\mathtt{Decl\to I}$ is also better at handling higher-order types.
When the lambda-expression to be inferred has a polymorphic input type,
such as $\all a \to a$,
$\mathtt{DeclSub}$ may not derive some judgments.
For example, $\Psi,id:\all a\to a \vdash \lam[f] f~id~(f~()) \Lto (\all a\to a) \to 1$
requires the argument of the lambda-expression to be a polymorphic type,
otherwise it could not be applied to both $id$ and $()$.
If Rule $\mathtt{DeclSub}$ was chosen for derivation,
the type of its argument is restricted by Rule $\mathtt{Decl\to I\To}$,
which is not a polymorphic type.
By contrast,
Rule $\mathtt{Decl\to I}$ keeps the polymorphic argument type $\all a\to a$,
and will successfully derive the judgment.
\end{comment}

%-------------------------------------------------------------------------------
\subsection{DK's Algorithm}\label{ssec:DK_Algorithm}

DK's algorithm version revolves around their notion of \emph{algorithmic context}.
\[
\begin{array}{l@{\qquad}lcl}
\text{Algorithmic Contexts}\qquad&\Gamma,\Delta,\Theta &::=&\quad \nil \mid
\Gamma, a \mid \Gamma, \al \mid \Gamma, \al = \tau \mid
\Gamma, \blacktriangleright_{\al}
\end{array}
\]
In addition to the regular (universally quantified) type variables $a$, the
algorithmic context also contains \emph{existential} type variables
$\al$. These are placeholders for monotypes $\tau$ that are still to
be determined by the inference algorithm. When the existential variable is
``solved'', its entry in the context is replaced by the assignment
$\al = \tau$. With the notation $[\Gamma]$ an algorithmic context
$\Gamma$ denotes the substitution that replaces all its solved existential type
variables with their solution.

All algorithmic judgments thread an algorithmic context. They have the form
$\Gamma \vdash \ldots \dashv \Delta$, where $\Gamma$ is the input context and
$\Delta$ is the output context:
$\Gamma \vdash A \le B \dashv \Delta$  for subtyping, 
$\Gamma \vdash e \Leftarrow A \dashv \Delta$  for type checking, and so on. 
The output context is a functional update of the input context that records newly
introduced existentials and solutions.

Solutions are incrementally propagated by applying the algorithmic output
context of a previous task as substitution to the next task. This can be seen
in the subsumption rule:
\[
\inferrule*
  {\Gamma \vdash e \Rightarrow A \dashv \Theta \\ 
   \Theta \vdash [\Theta]A \le [\Theta]B \dashv \Delta
  }
  { \Gamma \vdash e \Leftarrow B \dashv \Delta}
\]
The inference task yields an output context $\Theta$ which is applied as a substitution
to the types $A$ and $B$ before performing the subtyping check to propagate any solutions
of existential variables that appear in $A$ and $B$.

\paragraph{Markers for scoping.}
The sequential order of entries in the algoritmic context, in combination with
the threading of contexts,  does not perfectly capture the scoping of all
existential variables. For this reason the DK algorithm uses scope markers
$\blacktriangleright_{\al}$ in a few places. An example is given in the following
rule:
\[
\inferrule*
  {\Gamma,\blacktriangleright_{\al}, \al \vdash [\al/a]A \le B \dashv \Delta,\blacktriangleright_{\al},\Theta}
  {\Gamma \vdash \forall a. A \le B \dashv \Delta}
\]
To indicate that the scope of $\al$ is local to the subtyping check
$[\al/a]A \le B$, the marker is pushed onto its input stack and popped
from the output stack together with the subsequent part $\Theta$, which may
refer to $\al$. 
(Remember that later entries may refer to earlier
ones, but not vice versa.) This way $\al$ does not escape its scope.
\begin{comment}
A type variable may have a similar functionality to the scoping markers.
An example rule that checks an expression against a polymorphic type is as follows:
$$
\inferrule*
{\Gm, a \vdash e \Lto A \dashv \Delta, a, \Theta}
{\Gm \vdash e \Lto \all A \dashv \Delta}
$$
\end{comment}

Due to the complication of its scoping control method,
it is hard to argue the preciseness of scoping for each variable when designing algorithms.
Actually we found a scoping issue related to the subsumption rule,
which might cause existential variables to leak across judgments.
A detailed discussion on this issue with a possible fix is in Section~\ref{sec:discussion:scoping}.


At first sight, the DK algorithm would seem a good basis for mechanisation. After
all it comes with a careful description and extensive manual proofs.
Unfortunately, we ran into several obstacles that have prompted us to formulate
a different, more mechanisation-friendly algorithm.

%- - - - - - - - - - - - - - - - - - - - - - - - - - - - - - - - - - - - - - - - 
\paragraph{Broken Metatheory} % Talk about the lemmas that are false, show counterexample.
While going through the manual proofs of DK's algorithm, we found several
problems.  Indeed, two proofs of lemmas---Lemma 19 (Extension Equality
Preservation) and Lemma 14 (Subsumption)--- wrongly apply induction hypotheses
in several cases. Fortunately, we have found simple work-arounds that fix these
proofs without affecting the appeals to these lemmas.

% The proof of Lemma 19 (Extension Equality Preservation) applies a wrong
% induction hypothesis for the ${\longrightarrow}\mathtt{Uvar}$ case.
% Fortunately the lemma could be proven without the well-formedness conditions
% $\Gm\vdash A$ and $\Gm\vdash B$.
% Similar problem happens for Lemma 14 (Subsumption), where many applications of
% induction hypotheses are not correct.
% For manual proofs, induction hypotheses are not automatically generated by programs,
% thus are easily misused.

More seriously, we have also found a lemma that simply does not hold, 
Lemma 29 (Parallel Admissibility). See Appendix~\ref{} for a detailed explanation
and counterexample. This lemma is a cornerstone of the two metatheoretical results 
of the algorithm, soundness and completeness with respect to the declarative system.
In particular, both instantiation soundness (i.e. a part of subtyping
soundness) and typing completeness directly require the broken lemma.
Moreover, Lemma 54 (Typing Extension) also requires the broken lemma and is
itself used 13 times in the proof of typing soundness and completeness.
Unfortunately, we have not yet found a way to fix this problem.
\footnote{We found the issue with Lemma 29 in 2016 in an earlier attempt to mechanically
  formalize DK's algorithm. The authors acknowledged the problem after we contacted them through email.
  Although they briefly mentioned that it should be possible to use a weaker lemma instead they did
  not go into details.
}

%- - - - - - - - - - - - - - - - - - - - - - - - - - - - - - - - - - - - - - - - 
\paragraph{Complex Scoping and Propagation}

Besides the problematic lemmas in DK's metatheory, there are other reasons to
look for an alternative algorithmic formulation of the type system that is more
amenable to mechanisation. Indeed, two aspects that are particularly
challenging to mechanise are the scoping of universal and existential type
variables, and the propagation of the instantiation of existential type
variables across judgments. 
DK is already quite disciplined on these accounts, in particular compared to
traditional constraint-based type-inference algorithms like Algorithm $\mathcal{W}$~\cite{milner1978theory} which
features an implicit global scope for all type variables. Indeed, DK uses its
explicit and ordered context $\Gamma$ that tracks the relative scope of universal and
existential variables and it is careful to only instantiate existential
variables in a well-scoped manner.

Moreover, DK's algorithm carefully propagates instantiations by recording them
into the context and threading this context through all judgments. 
% two This means that every judgments takes two contexts---an input and an output
% context---rather than the conventional single context. The output context
% records any new variable instantations; to propagate these instantiations to
% the remaining judgements, their predecessor's output context---which is their
% input context---is applied to them as a substitution.
While this works well on paper, this approach is still fairly involved and thus
hard to reason about in a mechanised setting. Indeed, the instantiations have
to be recorded in the context and are applied incrementally to each remaining
judgment in turn, rather than immediately to all remaining judgments at once.
Also, as we have mentioned above, the stack discipline of the ordered contexts
does not mesh well with the use of output contexts; explicit marker entries are
needed in two places to demarcate the end of an existential variable's scope.
These are compelling reasons to look for a simpler algorithm that is easier to
reason about in a mechanised setting.


%-------------------------------------------------------------------------------
\subsection{Judgment Lists}\label{sec:overview:list}

To avoid the problem of incrementally applying a substitution to remaining
tasks, we can find inspiration in the formulation of constraint solving
algorithms. For instance, the well-known unification
algorithm by \citet{unification} decomposes the problem of unifying two terms $s \stackrel{.}{=} t$ into a set
of related unification problems between pairs of terms $s_i \stackrel{.}{=} t_i$. These smaller
problems are not tackled independently, but kept together in a set $S$. 
The algorithm itself is typically formulated as a small-step-style state
transition system $S \rightarrowtail S'$ that rewrites the set of unification
problems until it is in solved form or until a contradiction has been found.
For instance, the variable elimination rule is written as:
\[
   x \stackrel{.}{=} t, S  ~~\rightarrowtail~~  x \stackrel{.}{=} t, [t/x]S   \qquad\qquad{if}~x \not\in t~\text{and}~{x \in S}
\]
Because the whole set is explicitly available, the variable $x$ can be
simultaneously substituted everywhere.

In the above unification problem, all variables are implicitly in the same
global scope. Recent unification algorithms for dependently-typed languages
are more explicit about scopes. For instance, \citet{Reed2009} represents a unification
problem as $\Delta \vdash P$ where $P$ is a set of equations to be solved and $\Delta$ is
a (modal) context. \citet{Abel2011higher} even use multiple contexts within a unification problem.
Such a problem is denoted $\Delta \Vdash \mathcal{K}$ where the meta-context
$\Delta$ contains all the typings of meta-variables in the constraint set
$\mathcal{K}$. The latter consists of constraints like $\Psi \vdash M = N : C$
that are equipped with their individual context $\Psi$. While acurrately tracking
the scoping of regular and meta-variables, this approach is not ideal because it
repeatedly copies contexts when decomposing a unification problem, and
later it has to 
substitute solutions into these copies.

% %-------------------------------------------------------------------------------
% \subsection{Small-Step Unification}
% % Mention some algorithms for unification for 
% % dependent types that use a small-step approach. Credit them later for some ideas that 
% % we also employ.
% There are literals that make use of small-step unification for dependently typed
% inference and reconstruction algorithms~\cite{Reed2009,Abel2011higher}.
% These approaches collect a list of constraints and process one at a time.
% Similar to DK's algorithm, unification variables are used to represent the types to guess.
% In order to solve unification variables to correctly scoped types
% (typically constrained by the context when the variable is introduced),
% the context information should be kept with the variables.
% Judgments that represent unifications of terms might also require such a context
% to keep the constraint collection well-formed.
% 
% As an example, Abel et al. defines their constraint $K$ by
% \begin{gather*}
% \begin{aligned}
% K &::= {\top} \mid {\bot} &&\text{Trivial constraint and inconsistency.}\\
%     & \ \mid \  \Psi\vdash M = N : C &&\text{Unify term $M$ with $N$.}\\
%     & \ \mid \  \Psi\mid R:A \vdash E = E' &&\text{Unify evaluation context $E$ with $E'$.}\\
%     & \ \mid \  \Psi\vdash u\leftarrow M: C &&\text{Solution for $u$ (metavariable) found.}
% \end{aligned}
% \end{gather*}
% where the unification constraints and metavariable solutions are bound by a proper context.
% A unification problem in their system is described by $\Delta \Vdash \overline{K}$,
% where $\Delta$ is a collection of metavariables with their defined scopes.
% This approach clearly states the scroping of each unification problem,
% therefore rules out invalid instantiations to metavariables.
% 
% However, such ``duplicated contexts'' are not ideal for our formalization.
% Following DK's algorithm, an existential variable $\al$ defined in the context
% could be decomposed into a function type $\al[1] \to \al[2]$,
% so the declaration of $\al$ should be replaced by two declarations $\al[1], \al[2]$.
% Such operation causes all the context that contains $\al$ to change,
% which brings difficulty on the synchronization of multiple contexts,
% as variable declarations are not centralized.
% 
% We find a way of encoding multiple contexts by ``compressing'' them into a single worklist.
% As our type inference rules applied to a judgment,
% we typically keep the context or add variable declarations to the old context
% before analysing its sub-judgment(s).
% This enables us to write the judgment worklist in such a form:
% $$\Gm, \{\text{variable declarations}\}, \{\text{judgment}\},
% \{\text{variable declarations}\}, \{\text{judgment}\}, \ldots .$$
% Typically, we simplify the right-most judgment and push back smaller tasks,
% where the old ``context'' automatically gets inherited.
% When some rules solve or partially solve an existential variable,
% we could easily propagate the solution to all the judgments,
% and safely modify its declaration, such as remove from the worklist.
% New judgment(s) may also introduce new \emph{local} variables by declaring
% that variable right before the new judgment(s).
% After the new judgments are satisfied, these local variables are properly recycled.
% 
% % \jimmy{TODO explain the duplicated context with some examples (their judgment form)}
% % Some issues to point out: Duplicated contexts (rather than shared contexts), which make 
% % the formalization harder since requires ``synchronizing'' things ...
% 
%-------------------------------------------------------------------------------
\subsection{Single-Context Worklist Algorithm for Subtyping}

In recent work, \citet{itp2018} have shown how to mechanise a variant of DK's
subtyping algorithm and shown it correct with respect to DK's declarative
subtyping judgment. This approach overcomes the problem's in DK's formulation
by using a \emph{worklist}-based judgement of the form $$\Gamma \vdash \Omega$$
where $\Omega$ is a worklist (or sequence) of subtyping problems of the
form $A \leq B$.  The judgement is defined by case analysis on the first
element of $\Omega$ and recursively processes the worklist until it is empty.
Using both a single common ordered context $\Gamma$ and a worklist $\Omega$ greatly
simplifies propagating the instantiation of type variables in one
subtyping problem to the remaining ones.

Unfortunately, this work does not solve all problems. In particular, it has two
major limitations that prevent it from scaling to the whole DK system. 

\paragraph{Scoping Garbage} Firstly, the single common type ordered context 
$\Gamma$ does not accurately reflect the type and unification variables
currently in scope. Instead, it is an overapproximation that steadily accrues
variables, and only drops those unification variables that are instantiated.
In other words, $\Gamma$ contains ``garbage'' that is no longer in scope.
This complicates establishing the relation with the declarative system.


\paragraph{No Inference Judgments} 
Secondly and more importantly, the approach only deals with a judgement for
\emph{checking} whether one type is the subtype of another. While this may not
so be difficult to scale to the \emph{checking} mode of term typing $\Gamma
\vdash e \Leftarrow A$, it clearly lacks the functionality to support the
\emph{inference} mode of term typing $\Gamma \vdash e \Rightarrow A$. Indeed,
the latter requires not only the communication of unification variable
instantiations from one typing problem to another, but also an inferred type.

%-------------------------------------------------------------------------------
\subsection{Algorithmic Type Inference for Higher-Ranked Types: Key Ideas}

Our new algorithmic type system addresses builds on the work above, but
addresses the open problems.

\paragraph{Scope Tracking}
We avoid scoping garbage by blending the ordered context and the
worklist into a single syntactic sort $\Gamma$, our algorithmic worklist. This
algorithmic worklist interleaves (type and term) variables with \emph{work}
like checking or inferring types of expressions. The interleaving keeps track
of the variable scopes in the usual, natural way: each variable is in scope of
anything in front of it in the worklist. If there is nothing in front, the
variable is no longer needed and can be popped from the worklist. This way, no
garbage (i.e. variables out-of-scope) builds up.

$$\begin{aligned}
\text{Algorithmic judgement chain}\qquad&\jg &::=&\quad A \le B \mid e\Lto A \mid e\To_{a} \jg \mid \appInfAlg{A}{e}\\
\text{Algorithmic worklist}\qquad&\Gm &::=&\quad \nil \mid \Gm, a \mid \Gm, \al \mid \Gm, x: A \mid \Gm \Vdash \jg\\
\end{aligned}$$

For example, suppose that the top judgment of the following worklist
checks the identity function against $\all a \to a$:
$$\Gm \Vdash \lam x \Lto \all a \to a$$
To proceed, two auxiliary variables $a$ and $x$ are introduced to help the type checking:
$$\Gm, a, x:a \Vdash x \Lto a$$
which will be satisfied after several steps, and the worklist becomes
$$\Gm, a, x:a$$
Since the variable declarations $a, x:a$ are only used for a judgment already processed,
they can be safely removed, leaving the remaining worklist $\Gm$ to be further reduced.

Our worklist can be seen as an all-in-one stack,
containing variable declarations and subtyping/typing judgments.
The stack is an enriched form of ordered context,
and it has a similar variable scoping scheme.
%The well-formedness condition, shown in Figure~\ref{fig:alg:syntax},
%states that every variable must be declared prior to it usage.
%Therefore, if any variable declaration is the top element,
%then no usage should appear in the rest of the worklist, so it can be removed.

% $$\Gm \Vdash \all a \to a \le \all a \to a$$
% $$\Gm, a, \al \Vdash \al \to \al \le a \to a$$

\paragraph{Inference Judgements}
To express the DK's inference judgements, we use a novel form of work entries in
the worklist: our algorithmic judgement chains $\omega$. In its simplest form,
such a judgement chain is just a check, like a subtyping check $A \leq B$ or a
term typecheck $e \Leftarrow A$. 
However, the non-trivial forms of chains capture an
inference task together with the the work that depends on its outcome. For
instance, a type inference task takes the form $e \Rightarrow_a \omega$.
This form models a \emph{continuation passing style}:
it
denotes that we need to infer the type, say $A$, for expression $e$ and use it
in the chain $\omega$ by substituting it for the placeholder type variable $a$.

Take the following worklist as an example
$$\al \Vdash \blue{(\lam x)~() \To_a a \le \al} , x:\al, \bt \Vdash \blue{\al \le \bt} \Vdash \blue{\bt \le 1}$$
There are three (blue-colored) judgment chains in the worklist,
where the first and second judgment chains (from the right) are two subtyping judgments,
and the third judgment chain, $(\lam x)~() \To_a a \le \al$,
is a sequence of an inference judgment followed by a subtyping judgment.

The algorithm first analyses the two subtyping judgments and
will find the best solutions $\al = \bt = 1$
(please refer to Figure~\ref{fig:alg} for detailed derivations).
Then we substitute every instance of $\al$ and $\bt$ to $1$,
so the variable declarations can be safely removed from the worklist.
Now we reduce the worklist to the following state
$$\nil \Vdash \blue{(\lam x)~() \To_a a \le 1}, x:1$$
which has a term variable declaration as the top element.
%Because no judgments before the declaration refer to $x$,
%as a basic property of an ordered context
%(Figure~\ref{fig:alg:syntax} shows its well-formedness criteria),
%we remove all declarations before the first judgment chain to collect unused variables.
%As a result all garbage variables are collected
%once the judgment chains refering to them are processed,
%just as what we have mentioned above about the scope tracking.
After removing the garbage term variable declaration from the worklist, we process the last judgment chain from the left,
namely an inference judgment $(\lam x)~()\To~?$, with the unit type $1$ as its result.
Finally the last judgment becomes $1 \le 1$, and is satisfied as a base case.


% Our algorithm borrows some ideas from previous work, while adding new ones. 
% A small-step style processing worklists; \emph{Judgment Chains}; others. 

% Importantly we now deal with inference judgement.


%\input{sections/Declarative.tex}
\section{Algorithmic System}

This section introduces a novel algorithmic system that implements 
DK's declarative specification. The new algorithm extends the idea
of worklists proposed by \citet{itp2018} in two ways. Firstly,
unlike \citet{itp2018}'s worklists, the scope of variables is precisely tracked
and variables do not escape their scope. This is achieved by unifying algorithmic contexts and the worklists themselves.
Secondly, our algorithm also
accounts for the type system (and not just subtyping). To deal with
inference judgments that arise in the type system we employ a \emph{continuation
passing style} to enable the transfer of inferred information across
judgments in a worklist.

\subsection{Syntax and Well-Formedness}

Figure~\ref{fig:alg:syntax} shows the syntax and well-formedness
judgments used by the algorithm. Similarly to the declarative system 
the well-formedness rules are unsurprising and merely ensure 
well-scopedness.

\begin{figure}
\begin{gather*}
\begin{aligned}
% \text{Type variables}\qquad&a, b\\
\text{Existential variables}\qquad&\al, \bt
\\
%\text{Types}\qquad&A', B', C' &::=&\quad 1 \mid a \mid \forall x. A' \mid A'\to B'\\
% \text{Mono-types}\qquad&\tau &::=&\quad 1 \mid a \mid \tau_1\to \tau_2\\
%\text{Context}\qquad&\Psi &::=&\quad \nil \mid \Psi, a \mid \Psi, x:A' \color{red} \\[2mm]
% \text{Declarative Judgments}\qquad&\jo &::=&\quad \nil \mid \P\vdash A' \le B' \jc \jo\\
% &&&\quad \mid \P\vdash e\Leftarrow A' \jc \jo \mid \P\vdash e\To A' \jc \jo \mid \P\vdash A' \bullet e\TTo C' \jc \jo\\
% \noalign{\jimmy{\text{What do the markers do: $\G \vdash,a \vdash,b\vdash \j_1;\j_2;\j_3$ means $\G,a,b\vdash \j_1 \land \G,a\vdash \j_1,\j_2 \land \G\vdash \j_3$}}} \\[2mm]
\text{Algorithmic types}\qquad&A, B, C &::=& \quad 1 \mid a \mid \all A \mid A\to B \mid \al % \quad 1 \mid a \mid \al \mid \forall a. A \mid A\to B
\\
% \text{Terms}\qquad&e&::=&\quad x \mid () \mid \lam{e} \mid e_1~e_2 \mid (e:A)
% \\[2mm]
\text{Algorithmic judgment chain}\qquad&\jg &::=&\quad A \le B \mid e\Lto A \mid e\To_{a} \jg \mid \appInfAlg{A}{e}
\\
\text{Algorithmic worklist}\qquad&\Gm &::=&\quad \nil \mid \Gm, a \mid \Gm, \al \mid \Gm, x: A \mid \Gm \Vdash \jg%\\
% \text{Declarative worklist}\qquad&\Om &::=&\quad \nil \mid \Om, a \mid \Om, x: A \mid \Om \Vdash \jg
\end{aligned}
\end{gather*}

\centering \framebox{$\Gm \vdash A$} Well-formed algorithmic type
\begin{gather*}
\inferrule*[right=$\mathtt{wf\_ unit}$]
    {~}{\Gm\vdash 1}
\quad
\inferrule*[right=$\mathtt{wf\_ var}$]
    {a\in\Gm}{\Gm\vdash a}
\quad
\inferrule*[right=$\mathtt{wf\_ exvar}$]
    {\al\in\Gm}{\Gm\vdash \al}
\quad
\inferrule*[right=$\mathtt{wf\_{\to}}$]
    {\Gm\vdash A\quad \Gm\vdash B}
    {\Gm\vdash A\to B}
\quad
\inferrule*[right=$\mathtt{wf\_\forall}$]
    {\Gm, a\vdash A}
    {\Gm\vdash \forall a. A}
\end{gather*}

\centering \framebox{$\Gm \vdash e$} Well-formed algorithmic expression
\begin{gather*}
\inferrule*[right=$\mathtt{wf\_ tmvar}$]
    {x:A\in\Gm}{\Gm\vdash x}
\qquad
\inferrule*[right=$\mathtt{wf\_ tmunit}$]
    {~}{\Gm\vdash ()}
\qquad
\inferrule*[right=$\mathtt{wf\_ abs}$]
    {\Gm,x:A\vdash e}
    {\Gm\vdash \lam e}
\\
\inferrule*[right=$\mathtt{wf\_ app}$]
    {\Gm\vdash e_1 \quad \Gm\vdash e_2}
    {\Gm\vdash e_1~e_2}
\qquad
\inferrule*[right=$\mathtt{wf\_ anno}$]
    {\Gm\vdash A \quad \Gm\vdash e}
    {\Gm\vdash (e:A)}
\end{gather*}

\framebox{$\Gm\vdash\jg$} Well-formed algorithmic judgment
\begin{gather*}
\inferrule*[right=$\mathtt{wf{\le}}$]
{\Gm\vdash A \\ \Gm\vdash B}
{\Gm\vdash A\le B}
\qquad
\inferrule*[right=$\mathtt{wf{\Lto}}$]
{\Gm\vdash e \\ \Gm\vdash A}
{\Gm\vdash e \Lto A}
\\
\inferrule*[right=$\mathtt{wf{\To}}$]
{\Gm\vdash e \\ \Gm, a\vdash \jg}
{\Gm\vdash e \To_a \jg}
\qquad
\inferrule*[right=$\mathtt{wf{\TTo}}$]
{\Gm\vdash A \\ \Gm, a\vdash \jg \\ \Gm\vdash e}
{\Gm\vdash \appInfAlg{A}{e}}
\end{gather*}

\framebox{$\text{wf }\Gm$} Well-formed algorithmic worklist
\begin{gather*}
\inferrule*[right=$\mathtt{wf\nil}$]
{~}{\text{wf }\nil}
\qquad
\inferrule*[right=$\mathtt{wf_a}$]
{\text{wf }\Gm}
{\text{wf }\Gm, a}
\qquad
\inferrule*[right=$\mathtt{wf_{\al}}$]
{\text{wf }\Gm}
{\text{wf }\Gm, \al}
\qquad
\inferrule*[right=$\mathtt{wf_{of}}$]
{\text{wf }\Gm \\ \Gm\vdash A}
{\text{wf }\Gm, x:A}
\qquad
\inferrule*[right=$\mathtt{wf_{\jg}}$]
{\text{wf }\Gm \\ \Gm\vdash\jg}
{\text{wf }\Gm \Vdash\jg}
\end{gather*}
\caption{Extended Syntax and Well-Formedness for the Algorithmic System}\label{fig:alg:syntax}
\end{figure}

\paragraph{Existential Variables} 
The algorithmic system inherits the syntax of terms and types from 
the declarative system. It only introduces one additional feature.
In order to find unknown types $\tau$ in the declarative system, the
algorithmic system extends the declarative types $A$ with \emph{existential variables} $\al, \bt$.
They behave like unification variables,
but their scope is restricted by their position in the
algorithmic worklist rather than being global.
Any existential variable $\al$ should only be solved to
a type that is well-formed with respect to the worklist to which $\al$ has been added.
The point is that the monotype $\tau$, represented by the corresponding existential variable,
is always well-formed under the corresponding declarative context.
In other words, the position of $\al$'s reflects the well-formedness restriction of $\tau$'s.

%% A \le B \mid e\Lto A \mid e\To_{a} \jg \mid \appInfAlg{A}{e}

\paragraph{Judgment Chains} 
Judgment chains $\jg$, or judgments for short, are the core components of our algorithmic
type-checking. There are four kinds
of judgments in our system: subtyping ($A \le B$), checking ($e\Lto
A$), inference ($e\To_{a} \jg$) and
application inference ($\appInfAlg{A}{e}$).  Subtyping and checking are relatively simple,
since their result is only success or failure. However both inference and
application inference return a type that is used in subsequent judgments. We use a
continuation-passing-style encoding to accomplish this. For example, the judgment
chain $e \To_a (a \le B)$ contains two judgments: first we want to
infer the type of the expression $e$, and then check if that type is a
subtype of $B$. The \emph{unknown} type of $e$ is represented by a
type variable $a$, which is used as a placeholder in the second judgment to denote the 
type of $e$.

\paragraph{Worklist Judgments} Our algorithm has a non-standard form.
We combine traditional contexts and judgment(s) into a single sort, the \emph{worklist} $\Gm$.
The worklist is an \emph{ordered} collection of both variable bindings and judgments. The order captures the scope:
only the objects that come after a variable's binding in the worklist can refer to it.
For example, $[\nil, a, x:a \Vdash x \Lto a]$ is a valid worklist,
but $[\nil \Vdash \underline{x} \Lto \underline{a}, x:\underline{a}, a]$ is not
(the underlined symbols refer to out-of-scope variables).

\paragraph{Hole Notation}
We use the syntax $\Gm[\Gm_M]$ to denote the worklist $\Gm_L,\Gm_M,\Gm_R$,
where $\Gm$ is the worklist $\Gm_L,\bullet,\Gm_R$ with a hole ($\bullet$).
Hole notations with the same name implicitly share the same structure $\Gm_L$ and $\Gm_R$.
A multi-hole notation splits the worklist into more parts.
For example, $\Gm[\al][\bt]$ means $\Gm_1,\al,\Gm_2,\bt,\Gm_3$.

\subsection{Algorithmic System}

\newcounter{algRuleCounter}
\newcommand \algrule {\stepcounter{algRuleCounter}\rrule{\arabic{algRuleCounter}}}

% TODO rules too long
\begin{figure}[htp]
\hfill \framebox{$\Gm\rto \Gm'$} \hfill $\Gm$ reduces to $\Gm'$.
%\jimmy{Whenever there is a change on judgment count, marker should be properly added/removed. Decidability: all valid judgment chain should be reduced to nil.}
\begin{gather*}
\begin{aligned}
\Gm, a &\algrule \Gm \qquad
\Gm, \al \algrule \Gm \qquad
\Gm, x:A \algrule \Gm
\\[3mm]
\Gm \Vdash 1\le 1 &\algrule \Gm\\
\Gm \Vdash a\le a &\algrule \Gm\\
\Gm \Vdash \al\le \al &\algrule \Gm\\
\Gm \Vdash A_1\to A_2 \le B_1\to B_2 &\algrule \Gm \Vdash A_2 \le B_2 \Vdash B_1\le A_1\\
\Gm \Vdash \all A\le B &\algrule \Gm,\al \Vdash [\al/a]A\le B \quad\text{when } B \neq \all B'\\
\Gm \Vdash A\le \all[b]B &\algrule \Gm,b \Vdash A\le B
\\[3mm]
%\\
%\text{Let } \color{red}\Gm[\al \toto B // G_M] &:= \Gm_L, G_M, [B/\al]\Gm_R, \text{ when } \Gm[\al] = \Gm_L,\al,\Gm_R \ (G_M\text{ defaults to }\nil)\\
\Gm[\al] \Vdash \al \le A\to B &\algrule [\al[1]\to\al[2]/\al] (\Gm[\al[1], \al[2]] \Vdash \al[1]\to \al[2] \le A \to B)\\
 &\qquad\qquad \text{when }\al\notin FV(A)\cup FV(B)\\
\Gm[\al] \Vdash A\to B \le \al &\algrule [\al[1]\to \al[2]/\al] (\Gm[\al[1], \al[2]] \Vdash A \to B \le \al[1]\to \al[2])\\
 &\qquad\qquad \text{when }\al\notin FV(A)\cup FV(B)
 \\[3mm]
\Gm[\al][\bt] \Vdash \al \le \bt &\algrule [\al/\bt](\Gm[\al][])\\
\Gm[\al][\bt] \Vdash \bt \le \al &\algrule [\al/\bt](\Gm[\al][])\\
\Gm[a][\bt] \Vdash a \le \bt &\algrule [a/\bt](\Gm[a][])\\
\Gm[a][\bt] \Vdash \bt \le a &\algrule [a/\bt](\Gm[a][])\\
\Gm[\bt] \Vdash 1 \le \bt &\algrule [1/\bt](\Gm[])\\
\Gm[\bt] \Vdash \bt \le 1 &\algrule [1/\bt](\Gm[])
\\[3mm]
\Gm \Vdash e \Lto B &\algrule \Gm \Vdash e\To_a a\le B \quad
    \text{when } e \neq \lam e' \text{ and } B \neq \all B'\\
% \Gm \Vdash () \Lto 1 &\algrule \Gm\\
\Gm \Vdash e\Lto \all A &\algrule \Gm,a \Vdash e\Lto A\\
\Gm \Vdash \lam e \Lto A\to B &\algrule \Gm, x:A  \Vdash e \Lto B\\
\Gm[\al] \Vdash \lam e \Lto \al &\algrule [\al[1]\to \al[2] / \al](\Gm[\al[1],\al[2]], x:\al[1] \Vdash e \Lto \al[2])
% \quad\text{\jimmy{Additional}}
\\[3mm]
\Gm \Vdash x\To_a \jg &\algrule \Gm \Vdash [A/a] \jg \quad \text{when } (x:A)\in \Gm\\
\Gm \Vdash (e:A)\To_a \jg &\algrule \Gm \Vdash [A/a]\jg \Vdash e \Lto A\\
\Gm \Vdash ()\To_a \jg &\algrule \Gm \Vdash [1/a]\jg\\
\Gm \Vdash \lam e \To_a \jg &\algrule
    \Gm,\al,\bt \Vdash [\al\to\bt/a]\jg, x:\al \Vdash e\Lto \bt\\
\Gm \Vdash e_1\ e_2 \To_a \jg &\algrule \Gm \Vdash e_1\To_b (\appInfAlg{b}{e_2})
\\[3mm]
\Gm \Vdash \appInfAlg{\all A}{e} &\algrule \Gm,\al \Vdash \appInfAlg{[\al/a]A}{e}\\
\Gm \Vdash \appInfAlg{A\to C}{e} &\algrule \Gm \Vdash [C/a]\jg \Vdash e \Lto A\\
\Gm[\al] \Vdash \appInfAlg{\al}{e} &\algrule
    [\al[1]\to\al[2]/\al](\Gm[\al[1], \al[2]] \Vdash \appInfAlg{\al[1]\to\al[2]}{e})
%	[\al[1]\to\al[2]/\al](\Gm[\al[1], \al[2]] \Vdash [\al[2]/a]\jg \Vdash e\Lto \al[1])\\
% &\color{magenta} \makebox[0pt]{\qquad or} \phantom{{}\rrule{}{}}
\end{aligned}
\end{gather*}
\caption{Algorithmic Typing}\label{fig:alg}
\end{figure}

The algorithmic typing reduction rules, defined in Figure~\ref{fig:alg}, have
the form $\Gm \rto \Gm'$.
% The worklists $\Gm, \Gm'$ contain both variable declarations and judgments.
The reduction process treats the worklist as a stack.  In every step it pops
the first judgment from the worklist for processing and possibly pushes new
judgments onto the worklist.  The syntax $\Gm\redto \Gm'$ denotes multiple
reduction steps. 

$$\inferrule*[right=$\mathtt{{\redto} id}$]
{~}{\Gm\redto\Gm}
\qquad
\inferrule*[right=$\mathtt{{\redto} step}$]
{\Gm \rto \Gm_1 \quad \Gm_1 \redto \Gm'}{\Gm \redto \Gm'}$$

\noindent In the case that $\Gm\redto\nil$ this corresponds to successful type checking.

% \jimmy{The implicit freshness conditions for algorithmic system}
Please note that when a new variable is introduced in the right-hand side worklist $\Gm'$,
we implicitly pick a fresh one,
since the right-hand side can be seen as the premise of the reduction.

Rules 1-3 pop variable declarations that are essentially garbage.
That is variables that are
out of scope for the remaining judgments in the worklist.
All other rules concern a judgment at the front of the worklist. Logically we
can discern 6 groups of rules.

\paragraph{{\bf 1. Algorithmic subtyping}}
We have six subtyping rules (Rules 4-9) that are similar to their
declarative counterparts. For instance, Rule 7 consumes a subtyping
judgment and pushes two back to the worklist.  Rule 8 differs from
declarative Rule $\mathtt{{\le}{\forall}L}$ by introducing an existential
variable $\al$ instead of guessing the monotype $\tau$
instantiation. Each existential variable is later solved to a
monotype $\tau$ with the same context, so the final solution $\tau$ of
$\al$ should be well-formed under $\Gm$.

\paragraph{Worklist Variable Scoping}
Rules 8 and 9 involve variable declarations and demonstrate how our
worklist treats variable scopes. Rule 8 introduces an existential
variable $\al$ that is only visible to the judgment $[\al/a]A \le B$.
Reduction continues until all the subtyping judgments in front of $\al$
are satisfied.  Finally we can safely remove $\al$ since no occurrence
of $\al$ could have leaked into the left part of the worklist.  Moreover,
the algorithm garbage-collects the $\al$
variable at the right time: it leaves the environment immediately
after being unreferenced completely for sure.

\paragraph{Example} Consider the derivation of the subtyping judgment
$(1\to 1)\to 1 \le (\all 1\to 1) \to 1$:
\begin{gather*}
\begin{aligned}
  & \nil\vdash (1\to 1)\to 1 \le (\all 1 \to 1) \to 1\\
\rrule{7}  & \nil \Vdash 1 \le 1 \Vdash \all 1 \to 1 \le 1 \to 1\\
\rrule{8}  & \nil \Vdash 1 \le 1 ,\al \Vdash 1 \to 1 \le 1 \to 1\\
\rrule{7}  & \nil \Vdash 1 \le 1 ,\al \Vdash 1 \le 1 \Vdash 1 \le 1\\
\rrule{4} & \nil \Vdash 1 \le 1 ,\al \Vdash 1 \le 1\\
\rrule{4} & \nil \Vdash 1 \le 1 ,\al\\
\rrule{2} & \nil \Vdash 1 \le 1\\
\rrule{4} & \nil
\end{aligned}
\end{gather*}
First, the subtyping of two function types is split into two judgments by Rule 7:
a covariant subtyping on the return type and a contravariant subtyping on the argument type.
Then we apply Rule 8 to reduce the $\forall$ quantifier on the left side.
The rule introduces an existential variable $\al$ to the context (even though 
the type $\all 1 \to 1$ does not actually refer to the quantified type
variable $a$).
In the following 3 steps we satisfy the judgment $1 \to 1 \le 1 \to 1$ by Rules 7, 4 and 4.

Now the existential variable $\al$, introduced before but still unsolved,
is at the top of the worklist and Rule 2 garbage-collects it.
The process is carefully designed within the algorithmic rules:
when $\al$ is introduced earlier by Rule 8,
we foresee the recycling of $\al$ after all the judgments (potentially)
requiring $\al$ have been processed.
Finally Rule 4 reduces one of the base cases and finishes the subtyping derivation.

\paragraph{\bf 2. Existential decomposition.}
Rules 10 and 11 are algorithmic versions of Rule $\mathtt{{\le}{\to}}$; they
both partially instantiate $\al$ to function types.
The domain $\al[1]$ and range $\al[2]$ of the new function type are not determined:
they are fresh existential variables with the same scope as $\al$.
We replace $\al$ in the worklist with  $\al[1], \al[2]$.
To propagate the instantiation to the rest of the worklist and maintain well-formedness,
every reference to $\al$ is replaced by $\al[1] \to \al[2]$.
The \emph{occurs-check} condition prevents divergence as usual.
For example, without it $\al \le 1 \to \al$ would diverge.

\paragraph{\bf 3. Solving existentials} Rules 12-17 are base cases where an existential variable is solved.
They all remove an existential variable and substitute the
variable with its solution in the remaining worklist. Importantly the rules
respect the scope of existential variables. For example, Rule 12 
states that an existential variable $\al$ can be solved with another
existential variable $\bt$ only if $\bt$ occurs after $\al$.

One may notice that the subtyping relation for simple types is just equivalence,
which is true according to the declarative system.
The DK's system works in a similar way.

\paragraph{\bf 4. Checking judgments.}
Rules 18-21 deal with checking judgments.
Rule 18 is similar to $\mathtt{DeclSub}$, but rewritten in the
continuation-passing-style.
The side conditions $e \neq \lam e'$ and $B \neq \all B'$ 
prevent overlap with Rules 19, 20 and 21;
this is further discussed at the end of this section.
% \bruno{The side condition needs to be explained, and some mention of
%   the overlapping is probably needed here.}
Rules 19 and 20 adapt their declarative counterparts to the worklist style.
Rule 21 is a special case of $\mathtt{Decl\to I}$,
dealing with the case when the input type is an existential variable,
representing a monotype \emph{function} as in the declarative system
(it must be a function type, since the expression $\lam e$ is a function).
% \bruno{Rule 21
%   deserves some more explanation. What is the motivation for this
%   rule? You may want to give some concrete example to motivate it.}
The same instantiation technique as in Rules 10 and 11 applies.
The declarative checking rule $\mathtt{Decl1I}$ does not have a direct counterpart in the algorithm, 
because Rules 18 and 24 can be combined to give the same result.

\paragraph{Rule 21 Design Choice}
The addition of Rule 21 is a design choice we have made to simplify the side
condition of Rule 18 (which avoids overlap). It also streamlines the algorithm
and the metatheory as we now treat all cases
where we can see that an existential variable should be instantiated to a
function type  (i.e., Rules 10, 11, 21 and 29) uniformly.

The alternative would have been to omit Rule 21 and drop the condition on $e$
in Rule 18. The modified Rule 18 would then handle $\Gm \Vdash \lam e \Lto \al$
and yield $\Gm \Vdash \lam e \To_a a \le \al$, which would be further processed
by Rule 25 to $\Gm, \bt[1], \bt[2] \Vdash \bt[1] \to \bt[2] \le \al, x:\bt[1] \Vdash e \Lto \bt[2]$.
As a subtyping constraint between monotypes is simply equality, $\bt[1] \to \bt[2] \le \al$
must end up equating $\bt[1] \to \bt[2]$ with $\al$ and thus have the same effect as Rule 21, but in a more roundabout
fashion.

In comparison,
DK's algorithmic subsumption rule has no restriction on the expression $e$,
and they do not have a rule that explicitly handles the case $\lam e \Lto \al$.
Therefore the only way to check a lambda function against an existential variable
is by applying the subsumption rule, which further breaks into
type inference of a lambda function and a subtyping judgment.
% A combination of Rules $18'$ (Rule 18 without the restriction on $e$) and 25 and further derivations
% may result in a similar effect to Rule 21.
% Rule 21 guesses a monotype for the lambda function $\lam e$, so as Rule 25.
% The only difference is that a subtyping relation is introduced by Rule $18'$,
% and is to be processed after the type inference:
% $$\Gm \Vdash \lam e \Lto \al \rrule{18'} \Gm \Vdash \lam e \To_a a \le \al
% \rrule{25} \Gm, \bt[1], \bt[2] \Vdash \bt[1] \to \bt[2] \le \al, x:\bt[1] \Vdash e \Lto \bt[2]$$
% Since two monotypes with a valid subtyping relation indicates equality,
% eventually $\bt[1] \to \bt[2]$ and $\al$ should be unified with each other.
% DK's algorithm takes a similar approach to such variant,
% and we believe that they are equivalent.
% 
% Choosing Rule 21 simplifies the side condition on Rule 18,
% the metatheory and the design: whenever an input type could be a function type,
% the algorithmic rule should always take care of
% the possibility when the type being a single existential variable.
% This pattern applies to all the possible cases, including Rules 10, 11, 21 and 29.

\paragraph{\bf 5. Inference judgments.}
Inference judgments behave differently compared with subtyping and checking judgments:
they \emph{return} a type instead of only accepting or rejecting.
For the algorithmic system, where guesses are involved,
it may happen that the output type of an inference judgment refers to new existential variables,
such as Rule 25.
In comparison to Rule 8 and 9, where new variables are only referred by the sub-derivation,
Rule 25 introduces variables $\al, \bt$ that affect the remaining judgment chain.
This rule is carefully designed so that the output variables are bound by earlier declarations,
thus the well-formedness of the worklist is preserved,
and the garbage will be collected at the correct time.
By making use of the continuation-passing-style judgment chain,
inner judgments always share the context with their parent judgment.

\begin{comment}
Old text:
The design of our judgment chain is closely related to the shape of
the judgments
\jimmy{requires further clarification}
\bruno{What property? Not very clear}.
Subtyping and checking do not return anything, so variables cannot
leak anyway, as applied to Rules 8 and 9.
\bruno{Is the discussion that follows in the right place? We just jump
  to
  rule 26. Perhaps we can wait until we talk about inference to
  discuss those issues?}
However, inference and application inference may return a type that contains new variables.
Take Rule 26 as an example, if it simply returns $\al \to \bt$,
passes that to the next judgment and continues the type-checking process,
variables $\al$ and $\bt$ in the next judgment are out of scope
and will break the well-formedness of the worklist.
\end{comment}

Rules 22-26 deal with type inference judgments, written in continuation-passing-style.
When an inference judgment succeeds with type $A$,
the algorithm continues to work on the inner-chain $\jg$ by
assigning $A$ to its placeholder variable $a$.
Rule 23 infers an annotated expression by changing into checking mode,
therefore another judgment chain is created.
Rule 24 is a base case,
where the unit type $1$ is inferred and thus passed to its child judgment chain.
Rule 26 infers the type of an application by
firstly inferring the type of the function $e_1$,
and then leaving the rest work to an application inference judgment,
which passes $a$, representing the return type of the application,
to the remainder of the judgment chain $\jg$.

Rule 25 infers the type of a lambda expression by introducing $\al, \bt$
as the input and output types of the function, respectively.
After checking the body $e$ under the assumption $x:\al$,
the return type might reflect more information than simply $\al \to \bt$
through propagation when existential variables are solved or partially solved.
The variable scopes are maintained during the process:
the assumption of argument type ($x:\al$) is recycled after checking against the function body;
the existential variables used by the function type ($\al,\bt$) are only visible in the remaining chain $[\al\to\bt/a]\jg$.
The recycling process of Rule 25 differs from DK's corresponding rule significantly,
and we further discuss the differences in Section~\ref{sec:discussion:scoping}.

% \bruno{You have some note ``differs'' on the rules. Should that
%   deserve an explanation here?}

\paragraph{\bf 6. Application inference judgments}
Finally, Rules 27-29 deal with application inference judgments.
Rules 27 and 28 behaves similarly to declarative rules $\mathtt{Decl\forall App}$ and $\mathtt{Decl\to App}$.
Rule 29 instantiates $\al$ to the function type $\al[1] \to \al[2]$, just like Rules 10, 11 and 21.

\paragraph{Example}
Figure~\ref{fig:alg:sample} shows a sample derivation of the algorithm.
It checks the application $(\lam x)~()$ against the unit type.
According to the algorithm, we apply Rule 18 (subsumption), changing to inference mode.
Type inference of the application breaks into two steps by Rule 26:
first we infer the type of the function,
and then the application inference judgment helps to determine the return type.
In the following 5 steps the type of the identity function, $\lam x$, is inferred to be $\al \to \al$:
checking the body of the lambda function (Rule 25),
switching from check mode to inference mode (Rule 18),
inferring the type of a term variable (Rule 22),
solving a subtyping between existential variables (Rule 12) and
garbage collecting the term variable $x$ (Rule 3).

After that, Rule 28 changes the application inference judgment to
a check of the argument against the input type $\al$ and returns the output type $\al$.
Checking $()$ against the existential variable $\al$ solves $\al$ to the unit type $1$
through Rules 18, 24 and 16.
Immediately after $\al$ is solved, the algorithm replaces every occurrence of $\al$ with $1$.
Therefore the worklist remains $1 \le 1$, which is finished off by Rule 4.
Finally, the empty worklist indicates the success of the whole derivation.

\begin{figure}
\begin{gather*}
\begin{aligned}
           & \nil \Vdash (\lam x)~() \Lto 1\\
\rrule{18} & \nil \Vdash (\lam x)~() \To_a a\le 1\\
\rrule{26} & \nil \Vdash (\lam x) \To_b (\appInfAlg{b}{()}[a][a\le 1])\\
\rrule{25} & \nil,\al,\bt \Vdash \appInfAlg{\al\to\bt}{()}[a][a\le 1], x:\al \Vdash x\Lto \bt\\
\rrule{18} & \nil,\al,\bt \Vdash \appInfAlg{\al\to\bt}{()}[a][a\le 1], x:\al \Vdash x\To_b b\le \bt\\
\rrule{22} & \nil,\al,\bt \Vdash \appInfAlg{\al\to\bt}{()}[a][a\le 1], x:\al \Vdash \al\le \bt\\
\rrule{12} & \nil,\al \Vdash \appInfAlg{\al\to\al}{()}[a][a\le 1], x:\al\\
\rrule{3}  & \nil,\al \Vdash \appInfAlg{\al\to\al}{()}[a][a\le 1]\\
\rrule{28} & \nil,\al \Vdash \al\le 1 \Vdash () \Lto \al\\
\rrule{18} & \nil,\al \Vdash \al\le 1 \Vdash () \To_a a\le \al\\
\rrule{24} & \nil,\al \Vdash \al\le 1 \Vdash 1\le \al\\
\rrule{16} & \nil \Vdash 1\le 1\\
\rrule{4}  & \nil
\end{aligned}
\end{gather*}
\caption{A Sample Derivation for Algorithmic Typing}
\label{fig:alg:sample}
\end{figure}

In summary, our type checking algorithm accepts $(\lam x)~() \Lto 1$.

\paragraph{Non-overlapping and Deterministic Reduction}
An important feature of our algorithmic rules is that they are directly
implementable. Indeed, although written in the form of reduction rules, they do
not overlap and are thus deterministic.
%This has the added benefit that our % INCORRECT!
%algorithm runs in polynomial time, which makes it suitable for practical use.

Consider in particular Rules 8 and 9, which correspond to the declarative rules
$\mathtt{{\le}\forall L}$ and $\mathtt{{\le}\forall R}$. While those
declarative rules both match the goal $\all A\le \all[b]B$,
we have eliminated this overlap in the algorithm by restricting Rule 8
($B\neq\all B'$) and thus always applying Rule 9 to $\all A\le \all[b]B$.

Similarly, the declarative rule $\mathtt{DeclSub}$ overlaps highly with the
other checking rules. Its algorithmic counterpart is Rule 18. Yet, we have
avoided the overlap with other algorithmic checking rules by adding
side-conditions to Rule 18, namely $e\neq\lam e'$ and $B\neq\all B'$.

These restrictions have not been imposed arbitrarily:
we formally prove that the restricted algorithm is still complete.
In Section~\ref{sec:metatheory:non-overlapping} we discuss the relevant metatheory,
with the help of a non-overlapping version of the declarative system.


\section{Metatheory}\label{sec:metatheory}

This section presents the metatheory of the algorithmic system
presented in the previous section. We show that three main results hold: 
\emph{soundness}, \emph{completeness} and \emph{decidability}.
These three results have been mechanically formalized and proved in the 
Abella theorem prover~\cite{AbellaDesc}.

%-------------------------------------------------------------------------------
\subsection{Declarative Worklist and Transfer}
%\jimmy{The transfer rule outputs declarative judgment chains, which are later processed by another declarative relation, handling all the guessing.}

\begin{figure}[t]
\begin{gather*}
\begin{aligned}
\text{Declarative worklist}\qquad&\Om &::=&\quad \nil \mid \Om, a \mid \Om, x: A \mid \Om \Vdash \jg
\end{aligned}
\end{gather*}
\hfill \framebox{$\Gm \sto \Om$} \hfill $\Gm$ instantiates to $\Om$.
\begin{gather*}
\inferrule*[right=$\mathtt{{\sto}}\Om$]
{~}
{\Om \sto \Om}
\quad
\inferrule*[right=$\mathtt{{\sto}\al}$]
{\Om\vdash\tau \\ \Om,[\tau/\al]\Gm \sto \Om}
{\Om,\al,\Gm \sto \Om}
\end{gather*}
\caption{Declarative Worklists and Instantiation}
\label{fig:trans}
\end{figure}

To aid formalizing the correspondence between the declarative and algorithmic
systems, we introduce the notion of a declarative worklist $\Om$, defined in
Figure~\ref{fig:trans}. A declarative worklist $\Om$ has the same structure as an
algorithmic worklist $\Gm$, but does not contain any extistential variables $\al$.

\paragraph{Worklist instantiation.}
The relation $\Gm \sto \Om$ expresses that the algorithmic worklist $\Gm$ can 
be instiantiated to the declarative worklist $\Om$, by appropriately instantiating
all existential variables $\al$ in $\Gm$ with well-scoped monotypes $\tau$.
The rules of this instantiation relation are shown in Figure~\ref{fig:trans} too.
Rule $\mathtt{{\sto}\al}$ replaces the first existential variable with a well-scoped monotype and
repeats the process on the resulting worklist until no existential variable remains
and thus the algorithmic worklist has become a declarative one.
In order to maintain well-scopedness
the substitution is applied to all the judgments and term variable bindings in the scope of $\al$.

Observe that the instantiation $\Gm\sto\Om$ is not deterministic.
% \bruno{Do you mean here
%   ``is not deterministic''? Would that be more appropriate?}
From left to right, there are infinitely many possibilities to instantiate an existential variable and
thus infinitely many declarative worklists that one can get from an algorithmic one.
In the other direction, any valid monotype in $\Om$ can be abstracted to an
existential variable in $\Gm$. Thus different $\Gm$'s can be instantiated to the same
$\Om$.

Lemmas~\ref{lem:insert} and \ref{lem:extract}
generalize Rule $\mathtt{{\sto}\al}$ from substituting the first existential variable
to substituting any existential variable.

\begin{lemma}[Insert]\label{lem:insert}
If $\Gm_L\vdash \tau$ and $\Gm_L, [\tau/\al]\Gm_R \sto \Om$
, then $\Gm_L, \al, \Gm_R \sto \Om$.
\end{lemma}
\begin{lemma}[Extract]\label{lem:extract}
If $\Gm_L, \al, \Gm_R \sto \Om$
, then $\exists \tau$ s.t. $\Gm_L\vdash\tau$ and $\Gm_L, [\tau/\al]\Gm_R \sto \Om$.
\end{lemma}

\begin{figure}[t]
\hfill \framebox{$\|\Om\|$} \hfill Judgment erasure.
\begin{gather*}
\begin{aligned}
\|\nil\| &= \nil\\
\|\Om,a\| &= \|\Om\|, a\\
\|\Om,x:A\| &= \|\Om\|, x:A\\
\|\Om\Vdash\jg\| &= \|\Om\|
\end{aligned}
\end{gather*}

\hfill \framebox{$\Om \rto \Om'$} \hfill Declarative transfer.
\begin{gather*}
\begin{aligned}
\Om,a &\rto \Om \\  \Om,x:A & \rto \Om\\
\Om\Vdash A\le B &\rto \Om &\text{ when } \|\Om\| \vdash A\le B\\
\Om\Vdash e\Lto A &\rto \Om & \text{ when } \|\Om\| \vdash e\Lto A\\
\Om\Vdash e\To_a \jg &\rto \Om\Vdash[A/a]\jg & \text{ when } \|\Om\| \vdash e\To A\\
\Om\Vdash \appInfAlg{A}{e} &\rto \Om\Vdash[C/a]\jg & \text{ when } \|\Om\| \vdash \appInf{A}{e}{C}\\
\end{aligned}
\end{gather*}
\caption{Declarative Transfer.}
\label{fig:decl:worklist}
\end{figure}

\paragraph{Declarative transfer.}
Figure~\ref{fig:decl:worklist} defines a relation $\Om \rto \Om'$,
which transfers all judgements in the declarative worklists to the
declarative type system. This relation checks that every judgement entry in the worklist
holds using a corresponding conventional declarative judgement. 
The typing contexts of declarative judgements are recovered using an
auxiliary erasure function $\|\Om\|$. The erasure function
simply drops all judgment entries from the worklist, keeping only variable
and type variable declarations.

%-------------------------------------------------------------------------------
\subsection{Non-Overlapping Declarative System}
\label{sec:metatheory:non-overlapping}

DK's declarative system, shown in Figures \ref{fig:decl:sub} and
\ref{fig:decl:typing}, has a few overlapping rules. In contrast, our algorithm
has removed all overlap; at most one rule applies in any given situation.
This discrepancy makes it more difficult to relate the two systems.

To simplify matters, we introduce an intermediate system that is still declarative
in nature, but has no overlap. This intermediate system differs only in a few 
rules from DK's declarative system; these are the changes:
\begin{enumerate}
\item Restrict the shape of $B$ in the rule $\mathtt{\forall L}$ subtyping rule:
$$
\inferrule*[right=$\mathtt{\forall L'}$]
{B \neq \all[b]B' \\ \Psi \vdash \tau \\ \Psi \vdash [\tau/a]A \le B}
{\Psi \vdash \all A \le B}
$$
\item Drop the redundant rule $\mathtt{Decl1I}$,
which can be easily derived by a combination of
$\mathtt{DeclSub}$, $\mathtt{Decl1I{\To}}$ and $\mathtt{{\le}Unit}$:
$$
\inferrule*[right=$\mathtt{DeclSub}$]
{
    \inferrule*[right=$\mathtt{Decl1I{\To}}$]
    {~}
    {\Psi \vdash () \To 1}
    \\
    \inferrule*[right=$\mathtt{{\le}Unit}$]
    {~}
    {\Psi \vdash 1 \le 1}
}
{\Psi \vdash () \Lto 1}
$$
\item Restrict the shapes of $e$ and $A$ in the subsumption rule $\mathtt{DeclSub}$:
$$
\inferrule*[right=$\mathtt{DeclSub'}$]
{e \neq \lam e' \\ A \neq \all A' \\ \Psi \vdash e \To A \\ \Psi \vdash A \le B}
{\Psi \vdash e \Lto B}
$$
\end{enumerate}
The resulting declarative system has no overlapping rules
and more closely resembles the aglorithmic system, which contains
the same shape constraints.

We have proven soundness and completeness of the non-overlapping declarative
system with respect to the overlapping one to establish their equivalence.
Thus the restrictions do not change the expressive power of the system.
Modification (2) is relatively easy to justify, with the derivation given above:
the rule is redundant and can be replaced by a combination of three other rules.
Modifications (1) and (3) require inversion lemmas for the rules that overlap.
Firstly, Rule $\mathtt{\forall L}$ overlaps with Rule $\mathtt{\forall R}$ for the judgment
$\Psi \vdash \all A \le \all[b]B$.
The following inversion lemma for Rule $\mathtt{\forall R}$ resolves the overlap:
\begin{lemma}[Invertibility of $\mathtt{\forall R}$]
If $\Psi \vdash A \le \all B$ then $\Psi, a \vdash A \le B$.
\end{lemma}
The lemma implies that preferring Rule $\mathtt{\forall R}$ does not affect the derivability of the judgment.
Therefore the restriction $B \neq \all[b]B'$ in $\mathtt{\forall L'}$ is valid.

Secondly, Rule $\mathtt{DeclSub}$ overlaps with both $\mathtt{Decl\forall I}$ and $\mathtt{Decl{\to}I}$.
We have proven two inversion lemmas for these overlaps:
\begin{lemma}[Invertibility of $\mathtt{Decl\forall I}$]
If $\Psi \vdash e \Lto \all A$ then $\Psi, a \vdash e \Lto A$.
\end{lemma}
\begin{lemma}[Invertibility of $\mathtt{Decl{\to}I}$]
If $\Psi \vdash \lam e \Lto A \to B$ then $\Psi, x:A \vdash e \Lto B$.
\end{lemma}
These lemmas express that applying the more specific rules, rather than the more general rule $\mathtt{DeclSub}$, does not reduce the expressive power.

The proofs of the above two lemmas rely on an important property of the
declarative system, the \emph{subsumption lemma}. To be able to formulate this
lemma, Figure~\ref{fig:context_subtyping} introduces the \emph{context
subtyping relation} $\Psi \le \Psi'$. Context $\Psi$ subsumes context $\Psi'$ if they
bind the same variables in the same order, but the types $A$ of the term variables $x$
in the former are subtypes of types $A'$ assigned to those term variables in the latter.
Now we can state the subsumption lemma:

\begin{figure}[t]
\framebox{$\Psi' \le \Psi$}
$$\begin{aligned}
\inferrule*[right=$\mathtt{CtxSubEmpty}$]
{~}{\nil \le \nil}\qquad
\inferrule*[right=$\mathtt{CtxSubTyVar}$]
{\Psi' \le \Psi}{\Psi', a \le \Psi, a}\qquad
\inferrule*[right=$\mathtt{CtxSubTmVar}$]
{\Psi' \le \Psi \\ \Psi \vdash A' \le A}{\Psi', x:A' \le \Psi, x:A}\qquad
\end{aligned}$$
\caption{Context Subtyping}
\label{fig:context_subtyping}
\end{figure}

\begin{lemma}[Subsumption]
Given $\Psi' \le \Psi$:
\begin{enumerate}
    \item If $\Psi \vdash e \Lto A$ and $\Psi \vdash A \le A'$ then $\Psi' \vdash e \Lto A$;
    \item If $\Psi \vdash e \To A$ then there exists
        $A'$ s.t. $\Psi \vdash A' \le A$ and $\Psi' \vdash e \To A$;
    \item If $\Psi \vdash \appInf{C}{e}{A}$ and $\Psi \vdash C' \le C$,
        then there exists $A'$ s.t. $\Psi \vdash A' \le A$ and $\Psi' \vdash \appInf{C'}{e}{A'}$.
\end{enumerate}
\end{lemma}
This lemma expresses that any derivation in a context $\Psi$ has a corresponding derivation in any context
$\Psi'$ that it subsumes.

We have tried to follow DK's manual proof of this lemma,
but we discovered several problems in their reasoning that we have been able to address.
Fortunately we have found a different way to prove the lemma.
The details of this issue can be found in Appendix~\ref{}.\bruno{reference}

\paragraph{Three-Way Soundness and Completeness Theorems}
We now have three systems that can be related: DK's overlapping declarative system,
our non-overlapping declarative system, and our algorithmic system.
We have already establishes the first relation, that the two declarative
systems are equivalent.

In what follows, we will establish the soundness of our algorithm directly
against the original overlapping declarative system. However, we have found
that showing completeness of the algorithm is easier against the
non-overlapping declarative system. Of course, as a corollary, it follows that
our algorithm is also complete with respect to DK's declarative system.

%-------------------------------------------------------------------------------
\subsection{Soundness}

Our algorithm is sound with respect to DK's declarative system.
For any worklist $\Gm$ that reduces successfully,
there is a valid instantiation $\Om$ that transfers all judgements successfully to the declarative system.
\begin{theorem}[Soundness]
If \emph{wf }$\Gm$ and $\Gm \redto \nil$, then $\exists \Om$ s.t. $\Gm\sto\Om$ and $\Om\redto\nil$.
\end{theorem}

The proof proceeds by induction on derivation of $\Gm\redto\nil$.
Interesting cases are the those involving existential variable instantiations,
including Rules 10, 11, 21 and 29.
Applications of Lemmas \ref{lem:insert} and \ref{lem:extract}
help analyse the full instantiation of those existential variables.
For example, when $\al$ is solved to $\al[1] \to \al[2]$ in the algorithm,
applying the Extract lemma gives two instantiations $\al[1] = \sigma$ and $\al[2] = \tau$.
It follows that $\al = \sigma \to \tau$, which enables the induction hypothesis
and finishes the corresponding case. Some immediate corollaries which
show the soundness for specific judgement forms are: 


\begin{corollary}[Soundness, single judgment form]
Given \emph{wf }$\Gm$:
\begin{enumerate}
    \item If $\Gm \Vdash A \le B \redto \nil$\\
        then there exist $A', B', \Om$ s.t.
        $\Gm \Vdash A \le B \sto \Om \Vdash A' \le B'$ and $\|\Om\| \vdash A' \le B'$;
    \item If $\Gm \Vdash e \Lto A \redto \nil$\\
        then there exist $e', A', \Om$ s.t.
        $\Gm \Vdash e \Lto A \sto \Om \Vdash e \Lto A'$ and $\|\Om\| \vdash e \Lto A'$;
    \item If $\Gm \Vdash e \To_a \jg \redto \nil$ for any $\jg$\\
        then there exists $\Om, \jg', A$ s.t.
        $\Gm \sto \Om$ and $\|\Om\| \vdash e \To A$;
    \item If $\Gm \Vdash \appInfAlg{A}{e} \redto \nil$ for any $\jg$\\
        then there exists $\Om, \jg', A', C$ s.t.
        $\Gm \Vdash \appInfAlg{A}{e} \sto \Om \Vdash \appInfAlg{A'}{e}[a][\jg']$
            and $\|\Om\| \vdash \appInf{A'}{e}{C}$.
\end{enumerate}
\end{corollary}

%-------------------------------------------------------------------------------
\subsection{Completeness}

Completeness of our algorithm means that any derivation in the
declarative system has an algorithmic counterpart.

\begin{theorem}[Completeness]
If \emph{wf }$\Gm$ and $\Gm\sto\Om$ and $\Om\redto\nil$, then $\Gm \redto \nil$.
\label{thm:completeness}
\end{theorem}

We prove completeness by induction on the derivation of $\Om\redto\nil$
and use the non-overlapping declarative system.
Since the declarative worklist is reduced judgment by judgment
(shown in Figure~\ref{fig:decl:worklist}),
the induction always analyses the first judgment until it is satisfied.

As the algorithmic system introduces existential variables,
a declarative rule may correspond to multiple algorithmic rules,
and thus we analyse each of the possible cases.
\jimmy{Cite DK's proof as the sketch.}

Most cases are relatively easy to prove.
The Insert and Extract lemmas are applied when the algorithm uses existential variables,
but transferred to a monotype for the declarative system,
such as Rules 6, 10, 11, 12-17, 21 and 29.
Algorithmic Rules 10 and 11 require special treatment.
When the induction reaches the $\mathtt{{\le}{\to}}$ case,
the first judgment is of shape $A_1 \to A_2 \le B_1 \to B_2$.
One of the corresponding algorithmic judgments is $\al \le A \to B$.
However, the case analysis does not imply that $\al$ is fresh in $A$ and $B$,
therefore Rule~10 cannot be applied and the proof gets stuck.
The following lemma helps us out in those cases:
the success in declarative subtyping indicates the freshness of $\al$ in $A$ and $B$
in its corresponding algorithmic judgment.
In other words, the declarative system does not accept infinite types.
A symmetric lemma holds for $A\to B \le \al$.
%thus, as an example, we cannot find a monotype $\tau$ such that $\tau\le 1\to \tau$,
%which could be transferred by $\al\le 1\to\al$.

\begin{lemma}[Prune Transfer for Instantiation]
If $(\Gm \Vdash \al \le A \to B) \sto (\Om \Vdash C \le A_1 \to B_1)$ and
$\|\Om\| \vdash C \le A_1 \to B_1$, then $\al\notin FV(A) \cup FV(B)$.
\end{lemma}

The following corollary is derived immediately from Theorem~\ref{thm:completeness}.
\begin{corollary}[Completeness, single judgment form]
Given \emph{wf }$\Gm$ containing no judgment chains:
\begin{enumerate}
    \item If $\Gm \Vdash A \le B \sto \Om \Vdash A' \le B'$ and $\|\Om\| \vdash A' \le B'$
        then $\Gm \Vdash A \le B \redto \nil$;
    \item If $\Gm \Vdash e \Lto A \sto \Om \Vdash e \Lto A'$ and $\|\Om\| \vdash e \Lto A'$
        then $\Gm \Vdash e \Lto A \redto \nil$;
    \item If $\Gm \sto \Om$ and exists $A$ s.t. $\|\Om\| \vdash e \To A$
        then $\Gm \Vdash e \To_a 1 \le 1 \redto \nil$;
    \item If $\Gm \Vdash \appInfAlg{A}{e}[a][1 \to 1] \sto \Om \Vdash \appInfAlg{A'}{e}[a][1 \to 1]$
        and exists $C$ s.t. $\|\Om\| \vdash \appInf{A'}{e}{C}$
        \\
        then $\Gm \Vdash \appInfAlg{A}{e}[a][1 \le 1] \redto \nil$.
\end{enumerate}
\end{corollary}

%-------------------------------------------------------------------------------
\subsection{Termination and Decidability}

Finally, we show that our algorithm terminates. In other words:
\begin{theorem}[Termination]
Given a well-formed input $\Gm$, the algorithm terminates.
\end{theorem}
Our termination proof is based on a lexicographic group of induction measures\\
$\langle |\Gm|_e, |\Gm|_\Leftrightarrow, |\Gm|_\forall, |\Gm|_{\al}, |\Gm|_\to + |\Gm| \rangle$
on the worklist $\Gm$. The definitions of these measures can be found in Figure~\ref{fig:measures}. The first two, 
$|\cdot|_e$ and $|\cdot|_\Leftrightarrow$, measure the total size of terms
and the total difficulty of judgments, respectively. In the latter, check judgements
count for 2, inference judgments for 1 and function inference judgments for 3.
The next two measures, $|\cdot|_\forall$ and $|\cdot|_\to$, count the total number of
universal quantifiers and function types, respectively. Finally,
$|\cdot|_{\al}$ counts the number of existential variables in the worklist.

\begin{figure}[t]
\hfill \framebox{$|\Gm|_e$} \hfill  \llap{Size measure.}
\begin{gather*}
\begin{aligned}
|\Gm|_e &= \textstyle\sum_{e\Lto A\in\Gm}|e|_e + \sum_{e\To_a \jg\in\Gm}|e|_e +
    \sum_{\appInfAlg{A}{e}\in\Gm}|e|_e\\
\text{where } |x|_e &= |()|_e = 1 \qquad |\lam e|_e = |e|_e + 1\\
|e_1~e_2|_e &= |e_1|_e + |e_2|_e + 1 \qquad |e:A|_e = |e| + 1
\end{aligned}
\end{gather*}
\hfill \framebox{$|\Gm|_\Leftrightarrow$} \hfill  \llap{Judgment measure.}
\begin{gather*}
\begin{aligned}
|\Gm|_\Leftrightarrow &= 2\cdot\#_{e\Lto A\in\Gm} +
    \#_{e\To_a \jg\in\Gm} + 3\cdot\#_{\appInfAlg{A}{e}\in\Gm}
\end{aligned}
\end{gather*}
\hfill \framebox{$|\Gm|_\forall$} \hfill  \llap{Polymorphism measure.}
\begin{gather*}
\begin{aligned}
|\Gm|_\forall &= \textstyle\sum_{e\Lto A\in\Gm}|A|_\forall + \sum_{\appInfAlg{A}{e}\in\Gm}|A|_\forall +
    \sum_{A\le B\in\Gm} |A|_\forall + |B|_\forall\\
\text{where } |1|_\forall &= |a|_\forall = |\al|_\forall = 1\\
|A\to B|_\forall &= |A|_\forall + |B|_\forall \qquad |\all A|_\forall = |A|_\forall + 1
\end{aligned}
\end{gather*}
\hfill \framebox{$|\Gm|_\to$} \hfill  \llap{Function measure.}
\begin{gather*}
\begin{aligned}
|\Gm|_\to &= \textstyle\sum_{e\Lto A\in\Gm}|A|_\to + \sum_{\appInfAlg{A}{e}\in\Gm}|A|_\to +
    \sum_{A\le B\in\Gm} |A|_\to + |B|_\to\\
\text{where } |1|_\to &= |a|_\to = |\al|_\to = 1\\
|A\to B|_\to &= |A|_\to + |B|_\to + 1 \qquad |\all A|_\to = |A|_\to
\end{aligned}
\end{gather*}
\hfill \framebox{$|\Gm|_{\al}$} \hfill  \llap{Existential measure.}
\begin{gather*}
\begin{aligned}
|\Gm|_{\al} &= \#_{\al\in\Gm}\\
\end{aligned}
\end{gather*}
\caption{Worklist measures}
\label{fig:measures}
\end{figure}

It is not difficult to see that all but two algorithmic reduction rules
decrease the group of measures.  (The result of Rule 29 could be directly
reduced by Rule 28, which decreases the measures.) The two exceptions are Rules
10 and 11. Both rules increase the number of existential variables without
decreasing the number of universal quantifiers. However, they are both
immediately followed by Rule 7, which breaks the subtyping problem into two
smaller problems of the form $\al \le A$ and $A \le \al$ which we call
\emph{instantiation judgements}.

We now show that entirely reducing these smaller problems leaves the worklist
in a state with an overall smaller measure. Our starting point is a 
worklist of the form $\Gm,\Gm_i$ where $\Gm_i$ are instantiation judgments.
\begin{gather*}
\begin{aligned}
\Gm_i &:= \nil \mid \Gm_i, \al\le A \mid \Gm_i, A\le\al \quad
    \text{where } \al\notin FV(A) \cup FV(\Gm_i)
\end{aligned}
\end{gather*}
Fully reducing these instantiation judgments at the top of the worklist
has a twofold impact. Firstly, new entries may be pushed onto the worklist which
are not instantiation judgments. This only happens when $\Gm_i$ contains a universal quantifier
that is reduced by Rule 8 or 9. The new entries then are of the form $\Gm_\le$:
\begin{gather*}
\begin{aligned}
\Gm_\le &:= \nil \mid \Gm_\le, a \mid \Gm_\le, \al \mid \Gm_\le, A\le B
\end{aligned}
\end{gather*}
Secondly, reducing the instantiation judgments may also affect the remainder of the worklist $\Gm$,
by solving existing existentials and introducing new ones. This worklist update is captured in the 
update judgement $\Gm \jExt \Gm'$ defined in Figure~\ref{fig:worklist_ext}.
For instance, existential variable instantiation, 
i.e. $\Gm_L,\al,\Gm_R \jExt \Gm_L, \al[1], \al[2], [\al[1] \to \al[2] / \al]\Gm_R$,
can be derived as a combination of the three rules that define the update relation.

\begin{figure}
\hfill \framebox{$\Gm\jExt\Gm'$} \hfill \llap{$\Gm$ updates to $\Gm'$}.
\begin{gather*}
\inferrule*[right=$\mathtt{\jExt id}$]
    {~}{\Gm\jExt\Gm}
\qquad
\inferrule*[right=$\mathtt{\jExt solve}$]
    {|A|_\forall = 0 \quad \Gm_L,[A/\al]\Gm_R \jExt \Gm'}{\Gm_L,\al,\Gm_R \jExt \Gm'}
\qquad
\inferrule*[right=$\mathtt{\jExt \al}$]
    {\Gm_L,\al,\Gm_R \jExt \Gm'}{\Gm_L,\Gm_R \jExt \Gm'}
\end{gather*}
\caption{Worklist Update}\label{fig:worklist_ext}
\end{figure}

The good news is that worklist update does not affect the three main worklist measures:
\newcommand{\equivGm}[1]{|\Gm|_{#1} = |\Gm'|_{#1}}\label{lemma:inst:invariant}
\begin{lemma}[Measure Invariants of Worklist Extension]
If $\Gm\jExt\Gm'$ then $\equivGm{e}$ and $\equivGm\Leftrightarrow$ and $\equivGm\forall$.
\end{lemma}

Moreover, we can characterise the reduction of the instantiation judgments as follows.
\begin{lemma}[Instantiation Decidability]\label{lemma:inst:decidable}
For any well-formed algorithmic worklist $(\Gm, \Gm_i)$,
\begin{enumerate}[1)]
    \item When $|\Gm_i|_\forall = 0$, $(\Gm, \Gm_i) \redto \Gm'$,
        where $|\Gm'|_{\al} = |\Gm|_{\al} - |\Gm_i|$ and $\Gm\jExt\Gm'$.
    \item When $|\Gm_i|_\forall > 0$, $(\Gm, \Gm_i) \redto (\Gm', \Gm_\le)$,
        where $|\Gm_\le|_\forall = |\Gm_i|_\forall - 1$ and $\Gm\jExt\Gm'$.
\end{enumerate}
\end{lemma}
Hence, reducing the instantiation judgment prefix $\Gm_i$ either
decreases the number of universal quantifiers or 
the algorithm either decreases the number of $\forall$'s
or solves one existential variable per instantiation judgment.
The proof of this lemma proceeds by induction on the measure $2*|\Gm_i|_\to + |\Gm_i|$
of the instantiation judgment list $\Gm_i$.

Let us go back to the whole algorithm and summarize our findings.
The termination theorem is shown through a lexicographic group of induction measures
$\langle |\Gm|_e, |\Gm|_\Leftrightarrow, |\Gm|_\forall, |\Gm|_{\al}, |\Gm|_\to + |\Gm| \rangle$
which is decreased by nearly all rules.
In the exceptional case that the measure does not decrease immediately, 
we encounter an instantiation judgment at the top of the worklist. We can then
make use of Lemma~\ref{lemma:inst:decidable} to show that $|\Gm|_{\al}$ or $|\Gm|_\forall$ decreases
when that instantiation judgment is consumed. Moreover, Lemma~\ref{lemma:inst:invariant} establishes
that no higher-priority measure component increases. Hence, in the exceptional case we have an overall measure decrease too.

Combining all three main results (soundness, completeness and termination), we conclude that the declarative system is decidable
by means of our algorithm.
\begin{corollary}[Decidability of Declarative Typing]
Given \emph{wf }$\Om$, it is decidable whether $\Om\redto\nil$ or not.
\end{corollary}
\subsection{Abella}

\section{Discussion}

% \subsection{Overlapping Rules} Already discussed in previous sections

This section discusses some insights that we gained from our work and contrasts
the scoping mechanisms we have employed with those in DK's algorithm.
We also discuss a way to improve the precision of their scope tracking.
Furthermore we discuss and sketch an extension of our algorithm with
an elaboration to a target calculus, and discuss an extension of our algorithm
with scoped type variables~\cite{scoped-type-variables}.

\jimmy{TODO: update the summary}

\begin{comment}
\subsection{Implementation}
Anything to say about the implementation? Do we have one?
\end{comment}

\subsection{Contrasting our scoping mechanisms with DK's scoping}\label{sec:discussion:scoping}

A nice feature of our worklists is that, simply by interleaving variable declarations and
judgment chains, they make the scope of variables
precise.  DK's algorithm deals with garbage collecting variables in a
different way: through type variable or existential variable
markers (as discussed in Section~\ref{ssec:DK_Algorithm}).  Despite
the sophistication employed in DK's algorithm to keep scoping precise,
there is still a chance that unused existential variables leak their
scope to an output context and accumulate indefinitely.
For example, the derivation of the judgment $(\lam x)~() \Lto 1$ is as follows
$$
\inferrule*
{
    \inferrule*
    {
        \inferrule*
        {
            \inferrule*
            {\ldots x \To \al \ldots \\ \ldots \al \le \bt \ldots}
            {\Gm, \al, \bt, x:\al \vdash x \Lto \bt \dashv \Gm_1, x:\al}
        }
        {\Gm \vdash \lam x \To \al \to \bt \dashv \Gm_1}
        \\
        \inferrule*
        {\ldots () \Lto \al \ldots}
        {\Gm_1 \vdash \appInf{\al \to \al}{()}{\al} \dashv \Gm_2}
    }
    {\Gm \vdash (\lam x)~() \To \al \dashv \Gm_2}
    \\
    \inferrule*
    {~}
    {\Gm_2 \vdash 1 \le 1 \dashv \Gm_2}
}
{\Gm \vdash (\lam x)~() \Lto 1 \dashv \Gm, \al = 1, \bt = \al}
$$
where $\Gm_1 := (\Gm, \al, \bt = \al$) solves $\bt$,
and $\Gm_2 := (\Gm, \al = 1, \bt = \al$) solves both $\al$ and $\bt$.

If the reader is not familiar with DK's algorithm,
he/she might be confused about the inconsistent types across judgment.
As an example, $(\lam x)~()$ synthesizes $\al$,
but the second premise of the subsumption rule uses $1$ for the result.
This is because a context application $[\Gm,\al=1,\bt=\al]\al = 1$ happens between the premises.

The existential variables $\al$ and $\bt$ are clearly not used after the subsumption rule,
but according to the algorithm, they are kept in the context
until some parent judgment recycles a block of variables,
or to the very end of a type inference task.
In that sense, DK's algorithm does not control the scoping of variables precisely.
%However, it is a minor issue that does not affect the soundness and completeness properties.

Two rules we may blame for not garbage collecting correctly are the inference rule for lambda
functions and an application inference rule:
$$
\inferrule*[right=$\mathtt{DK\_{\to}I{\To}}$]
{\Gamma, \al, \bt, x:\al \vdash e \Lto \bt \dashv \Delta, x:\al, \Theta}
{\Gamma \vdash \lam e \To \al \to \bt \dashv \Delta}
\qquad
\inferrule*[right=$\mathtt{DK\_\forall App}$]
{\Gamma, \al \vdash [\al/a] \appInf{A}{e}{C} \dashv \Delta}
{\Gamma \vdash \appInf{\all A}{e}{C} \dashv \Delta}
$$
In contrast, Rule 25 of our algorithm collects the existential variables
right after the second judgment chain,
and Rule 27 collects one existential variable similarly:
\[\Gm \Vdash \lam e \To_a \jg \rrule{25}
\Gm,\al,\bt \Vdash [\al\to\bt/a]\jg, x:\al \Vdash e\Lto \bt\]
\[\Gm \Vdash \appInfAlg{\all A}{e} \rrule{27} \Gm,\al \Vdash \appInfAlg{[\al/a]A}{e}\]
It seems impossible to achieve a similar outcome in DK's system by only modifying these two rules.
Taking $\mathtt{DK\_{\to}I{\To}}$ as an example,
the declaration or solution for $\al$ and $\bt$ may be referred to by subsequent judgments.
Therefore leaving $\al$ and $\bt$ in the output context is the only choice,
when the subsequent judgments cannot be consulted.

To fix the problem, one possible modification is on the algorithmic subsumption rule,
as garbage collection for a checking judgment is always safe:
$$
\inferrule*[right=$\mathtt{DK\_Sub}$]
{\Gamma, \blacktriangleright_{\al} \vdash e \To A \dashv \Theta \\
\Theta \vdash [\Theta]A \le [\Theta]B \dashv \Delta, \blacktriangleright_{\al}, \Delta'}
{\Gamma \vdash e \Lto B \dashv \Delta}
$$
Here we employ the markers in a way they were originally not intended for.
We create a dummy fresh existential $\al$ and add a marker to the input context of the inference judgment.
After the subtyping judgment is processed we look for the marker and drop everything afterwards.
We pick this rule because it is the only one where a checking judgment calls an inference judgment.
That is the point where our algorithm recycles variables---right after a judgment chain is satisfied.
After applying this patch, to the best of our knowledge,
DK's algorithm behaves equivalently to our algorithm in terms of variable scoping.
However, this exploits markers in a way they were not intended to be used and seems ad-hoc.

% \subsection{Scoped type variables} Not discussed for this paper

%-------------------------------------------------------------------------------
\subsection{Elaboration}

Type-inference algorithms are often extended with an
associated elaboration. For example, for languages with implicit
polymorphism, it is common to have an elaboration to a variant of
System F~\cite{reynolds1983types}, which recovers type information and explicit type
applications. Therefore a natural question is whether our algorithm
can also accommodate such elaboration.
While our algorithmic reduction does not elaborate to System F,
we believe that it is not difficult to extend the algorithm with a (type-directed) elaboration.
We explain the rough idea as follows.

Since the judgment form of our algorithmic worklist contains a collection of judgments,
elaboration expressions are also generated as a list of equal length to
the number of judgments (\emph{not judgment chains}) in the worklist.
As usual, subtyping judgments translate to coercions (denoted by $f$ and
represented by System F functions),
all three other types of judgments translate to terms in System F (denoted by $t$).

Let $\Phi$ be the elaboration list, containing translated type coercions and terms:
$$\Phi ::= \nil \mid \Phi, f \mid \Phi, t$$

Then the form of our algorithmic judgment becomes:
$$\Gamma \hookrightarrow \Phi$$

We take Rule 18 as an example, rewriting small-step reduction in a relational style,
$$
\inferrule*[right=$\mathtt{Translation\_18}$]
{\Gamma \Vdash e \To_a a \le B \hookrightarrow \Phi, f, t}
{\Gamma \Vdash e \Lto B \hookrightarrow \Phi, f t}
$$
As is shown in the conclusion of the rule,
a checking judgment at the top of the worklist corresponds to a top element for elaboration.
The premise shows that one judgment chain may relate to more than one elaboration elements,
and that the outer judgment, being processed before inner ones,
elaborates to the top element in the elaboration list.

Moreover, existential variables need special treatment, since they may be solved at any point,
or be recycled if not solved within their scopes.
If an existential variable is solved, we not only propagate the solution to the other judgments,
but also replace occurrences in the elaboration list.
If an existential variable is recycled, meaning that it is not constrained,
we can pick any well-formed monotype as its solution.
The unit type $1$, as the simplest type in the system, is a good choice.

\subsection{Lexically-scoped type variables}
We further extend the type system to support a simple extension,
lexically-scoped type variables~\cite{scoped-type-variables}.
We have a complete Abella formalization for this extension proving
all the metatheory discussed in the paper.

From a
practical point of view, such extension enables the programmer to
refer to type variables in a generic function signature when
implementing that function.  For example,
$$(\lam{\lam[y] (x:a)}) : \all[a~b] a \to b \to a$$
has an inner annotation $(x:a)$ that refers to the outer generic type variable $a$.
This is not a surprising feature, since the declarative system already accepts these kind of programs
$$
\inferrule*[right=$\mathtt{DeclAnno}$]
{
    \Psi \vdash \all A \quad
    \inferrule*[right=$\mathtt{Decl\forall I}$]
    {\Psi,a \vdash e \Lto A}
    {\Psi\vdash e\Lto \all A}
}
    {\Psi\vdash (e: \all A)\To \all A}
$$

But there are issues about the well-formedness condition.
Normally $\Psi \vdash (e : A)$ is defined by $\Psi \vdash e$ and $\Psi \vdash A$.
However, when $A = \all A'$, the type variable $a$ is not introduced to the context of $e$,
therefore $\Psi \vdash e$ is not derivable.

To address the problem,
we add a syntax that explicitly bind the generic type variable into a function and its annotation.
The syntax of expression is extended with one more constructor:
$$
\begin{array}{l@{\qquad}lcl}
\text{Expressions}\qquad&e &::=&\quad x \mid () \mid \lam e \mid e_1~e_2 \mid (e:A) \mid \Lambda a.~e:A
\end{array}
$$

The new type-lambda syntax $\Lambda a.~e:A$ actually annotates the generic function $e$ with $\all A$,
while making $a$ visible inside the body of the function.
The well-formedness judgments are updated naturally:
$$
\inferrule*[right=$\mathtt{wf_d}\Lambda$]
    {\Psi,a \vdash e \quad \Psi,a \vdash A}
    {\Psi \vdash \Lambda a.~e:A}
\qquad
\inferrule*[right=$\mathtt{wf\_}\Lambda$]
    {\Gm,a \vdash e \quad \Gm,a \vdash A}
    {\Gm \vdash \Lambda a.~e:A}
$$

Corresponding rules are introduced for both the declarative system and the algorithmic system:
$$
\inferrule*[right=$\mathtt{Decl}\Lambda$]
    {\Psi,a \vdash A \quad \Psi,a \vdash e\Lto A}
    {\Psi\vdash \Lambda a.~e:A \To \all A}
$$
$$\Gm \Vdash \Lambda a.~e:A \To_b \jg \algrule \Gm \Vdash [(\all A)/b]\jg, a \Vdash e \Lto A$$

In practice, programmers would not write the syntax $\Lambda a.~e : A$ directly.
The ScopedTypeVariables extension of Haskell is effective only
when the type signature is explicitly universally quantified
(the compiler translates into an expression similar to $\Lambda a.~e : A$);
otherwise the program means the normal syntax $e : \all A$,
and one may not later refer to the type variable $a$.

We also obtained all three desired properties for the extended system,
namely soundness, completeness and decidability.


\section{Related Work}

Throughout the paper we have already discussed much of the closest related work.
In this section we summarize the key differences and novelties, and we discuss some other
related work.

\paragraph{Predicative Higher-Ranked Polymorphism Type Inference Algorithms}
Higher-ranked polymorphism is a convenient and practical feature of
programming languages.  Since full type-inference for System F is
undecidable~\cite{wells1999typability}, various decidable partial
type-inference algorithms were developed.
% No need to cite here: the following text is describing the point ~\cite{}.
The declarative system of this paper,
proposed by \citet{dunfield2013complete}, is \emph{predicative}: 
$\forall$'s only instantiate to monotypes.  The monotype restriction
on instantiation is considered reasonable and practical for most
programs, except for those that require sophisticated forms of
higher-order polymorphism.  In those cases, the bidirectional system
accepts guidance through type annotations, which allow polymorphic types.
Such annotations also improve
readability of the program, and are not much of a burden in practice.

DK's algorithm is shown to be sound, complete and decidable in 70 pages of manual proofs.
Though carefully written, some of the proofs are incorrect
(see discussion in Appendix~\ref{appendix:false_lemmas} and \ref{appendix:subsumption}),
which creates difficulties when formalizing them in a proof assistant.
In their follow-up work \citet{DunfieldIndexed} enrich the bidirectional higher-rank system with
existentials and indexed types.
With a more complex declarative system, they developed a proofs of over 150 pages.
It is even more difficult to argue its correctness for every single detail within such a big development.
Unfortunately, we find that their Lemma 26 (Parallel Admissibility) appears to have the same issue 
as lemma 29 in \citet{dunfield2013complete}: the conclusion is false. We also discuss
the issue in more detail in Appendix~\ref{appendix:false_lemmas}. 
\bruno{I think you will need to justify this better and in the appendix you must show
the lemma and counter-example.}

\citet{jones2007practical} developed another higher-rank predicative bidirectional type system.
Their subtyping relation is enriched with \emph{deep skolemisation},
which is more general than ours and allows more valid relations.
In comparison to DK's system, they do not use the application inference judgment,
resulting in a complicated mechanism for implicit instantiation taken care by the unification process for the algorithm.
A manual proof is given, showing that the algorithm is sound and
complete with respect to their declarative specification.

\paragraph{Impredicative Higher-Ranked Polymorphism Type Inference Algorithms}
Impredicative System F allows instantiation with polymorphic types,
but unfortunately its subtyping system is already undecidable~\cite{tiuryn1996subtyping}.
Works on partial impredicative type-inference algorithms~\cite{le2003ml,leijen2008hmf,vytiniotis2008fph}
navigate a variety of design tradeoffs for a decidable algorithm.
As a result, such algorithms tend to be more complicated, and thus less adopted in practice.
Recent work proposed \emph{Guarded Impredicative Polymorphism}~\cite{Serrano2018},
as an improvement on GHC's type inference algorithm with impredicative instantiation.
They make use of local information in $n$-ary applications to
infer polymorphic instantiations with a relatively simple specification and unification algorithm.
Although not all impredicative instantiations can be handled well,
their algorithm is already quite useful in practice.

\paragraph{Mechanical Formalization of Polymorphic Subtyping}
In all previous work on type inference for higher-ranked polymorphism
(predicative and impredicative) discussed above, proofs and
metatheory are manual. The only partial effort on mechanizing aspects of type inference
for higher-ranked types is
the Abella formalization of \emph{polymorphic subtyping} by \citet{itp2018}.
The judgment form of worklist $\Gm \vdash \Omega$ used in the formalization simplifies
the propagation of existential variable instantiations.
However, the approach has two main drawbacks:
it does not collect unused variable declarations effectively;
and the simple form of judgment cannot handle inference modes, which output types.
The new worklist introduced in this paper inherits the simplicity of propagating instantiations,
but overcomes both of the issues by mixing judgments with declarations
and using the continuation-passing-style judgment chains. Furthermore,
we formalize the complete bidirectional type system by
\citet{dunfield2013complete}, whereas Zhao et al. only formalize
the subtyping relation. 

\paragraph{Mechanical Formalizations of Other Type-Inference Algorithms}
Since the publication of the {\sc POPLMark} challenge~\cite{aydemir2005mechanized},
many theorem provers and packages provide new methods for dealing
with variable binding~\cite{aydemir2008engineering,urban2008nominalTech,chlipala2008parametric}.
More and more type systems are formalized with these tools.
However, mechanising certain algorithmic aspects, like unification and
constraint solving, has received very little attention and is still challenging.
Moreover, while most tools supports local (input) contexts in a neat way,
many practical type-inference algorithms require
more complex binding structures with output contexts or various forms of constraint solving procedures.

Algorithm $\mathcal{W}$,
as one of the classic type inference algorithms for polymorphic type systems,
has been manually proven to be sound and complete
with respect to the Hindley-Milner type system~\cite{hindley1969principal,milner1978theory,damas1982principal}.
After around 15 years, the algorithm was formally verified by
\citet{naraschewski1999type} in Isabelle/HOL~\cite{nipkow2002isabelle}.
The treatment of new variables was tricky at that time, while the overall structure follow the
structure of Damas's manual proof closely.
%Another complication is the encoding of substitutions,
%which they chose to formalize as a function instead of an association list.
Later on, other researchers~\cite{dubois2000proving,dubois1999certification}
formalized algorithm $\mathcal{W}$ in Coq~\cite{Coq}.
Nominal techniques~\cite{urban2008nominalTech} in Isabelle/HOL have been
developed to help programming language formalizations, and are used for a similar
verification~\cite{urban2008nominal}. Moreover, Garrigue~\cite{garrigue2015certified}
mechanized a type inference algorithm,
with the help of locally nameless cofinite quantification~\cite{LocallyNameless},
for Core ML extended with structural polymorphism and recursion.

\paragraph{Ordered Contexts in Type Inference}
Gundry et al.~\cite{gundry2010type} revisit algorithm $\mathcal{W}$ and
propose a new unification algorithm with the help of ordered contexts.
Similar to DK's algorithm, information of meta variables flow from input contexts to output contexts.
Not surprisingly, its information increase relation has a similar role to DK's context extension.
Our algorithm, in contrast,
eliminates the use of output contexts and their variable solution records ($\al = \tau$),
simplifying the information propagation process through immediate substitution
by collecting all the judgments in a single worklist.

\paragraph{Lists of Judgements in Unification}
Some work~\cite{Reed2009,Abel2011higher} adopts a similar idea to this paper
in work on unification for dependently typed languages. Similarly to our work
the algorithms need to be very careful about scoping, since the order of variable
declarations is fundamental in a dependently typed setting. 
Their algorithms simplify a collection of unification constraints progressively in a single-step style.
In comparison, our algorithm mixes variable declarations with judgments,
resulting in a simpler judgment form,
while processing them in a similar way to those algorithms.
One important difference is that contexts are
duplicated in their unification judgments, which complicates the unification process,
since the information of each local context needs to be synchronized.
Instead we make use of the natural of ordered context to control the scoping of unification variables.
While their algorithms focus only on unification,
our algorithm also deals with other type of judgments like type inference.
A detailed discussion is in Section~\ref{sec:overview:list}.

\section{Conclusion}

In this paper we have provided the first mechanized formalization of type
inference for higher-ranked polymorphism. This contribution is made possible by
a new type inference algorithm for DK's declarative type system that
incorporates two novel mechanization-friendly ideas. Firstly, we merge the
traditional type context with the recently proposed concept of judgment chains,
to accurately track variable scopes and to easily propagate substitutions.
Secondly, we use a continuation-passing style to return types from the
synthesis mode to subsequent tasks.

We leave extending our algorithm with elaboration to future work, as well as
investigating whether the problems we have found in DK's manual proofs for
their algorithm can be addressed.

% % \jimmy{Discuss the efficient algorithm and its equivalence}
% % The algorithm is already efficient
% 
% Outline:
% \begin{itemize}
%     \item difficulties following DK's formalization
%     \item a blend of ideas from various, including sources DK's ordered contexts
%     \item list of judgments + some novel ideas
%     \item lessons learned: problems identified in proofs and scoping
%     \item Future work
% \end{itemize}
% 
% Our judgment form, the worklist, is shown to be an elegant way to
% formalize higher-order type infernce algorithms.
% By mixing judgment chains with variable declarations in a single sort,
% variable scopings are easily controlled by the ordered context in a natural way.
% 
% The formal proof in Abella convinces us of the correctness of our algorithm,
% which further indicates that DK's algorithm, sharing a similar solving procedure as ours,
% should be a correct one.
% 
% \paragraph{Future work}
% \begin{itemize}
%     \item Scoped type variables?
%     \item Go beyond predicative instantiations
%     \item More complicated features, such as dependent type
% \end{itemize}
% 
% We have not yet explore lexically-scoped type variables~\cite{jones2003lexically}
% as a practical extension to our system.
% Formalization of a different variable scoping scheme is also
% a challenging task for formal proofs via proof assistants.
% 
% Moreover, the garbage collection process for type variables and
% existential variables do worth further attention:
% the inference of $\lam x$ leaves an unsolved $\al$ in the context,
% which we simply remove the variable after it is no longer referred to.
% Is it possible that we infer better types by making better use of that piece of information?
% How do we deal with, at least some ``trivial'' impredicative instantiations?
% 
% Last but not least, a further expectation is that our system would
% adapt to various complicated type inference algorithms,
% such as ones for dependent types.



%% Acknowledgments
\begin{acks}                            %% acks environment is optional
                                        %% contents suppressed with 'anonymous'
  %% Commands \grantsponsor{<sponsorID>}{<name>}{<url>} and
  %% \grantnum[<url>]{<sponsorID>}{<number>} should be used to
  %% acknowledge financial support and will be used by metadata
  %% extraction tools.

We sincerely thank the anonymous reviewers for their insightful
comments. Ningning Xie found the issue with Lemma 29 in DK's formalization that we reported 
on this article. This work has been sponsored by the Hong Kong Research
Grant Council projects number 17210617 and 17258816, and by the
Research Foundation - Flanders.
\end{acks}


%% Bibliography
\bibliography{main}


%% Appendix
\newpage
\appendix

\section{Appendix}

\subsection{Formal Definitions for Worklist Measures}

\begin{figure}[h!]
    \begin{minipage}[t]{.5\textwidth}
        \centering\framebox{$|\Gm|_e$, $|\jg|_e$, $|e|_e$} {Size measure.}
        \begin{gather*}
            \begin{aligned}
            |\Gm|_e &= \sum_{\jg \in \Gm} |\jg|_e\\[2mm]
            |A \le B|_e &= 0\\
            |e \Lto A|_e &= |e|_e\\
            |e \To \jg|_e &= |e|_e + |\jg|_e\\
            |\appInfAlg{A}{e}|_e &= |e|_e + |\jg|_e\\[2mm]
            |x|_e &= |()|_e = 1\\
            \qquad |\lam e|_e &= |e|_e + 1\\
            |e_1~e_2|_e &= |e_1|_e + |e_2|_e + 1\\
            |e:A|_e &= |e|_e + 1
            \end{aligned}
        \end{gather*}
        
        \centering\framebox{$|\Gm|_\Leftrightarrow$, $|\jg|_\Leftrightarrow$} {Judgment measure.}
        \begin{gather*}
            \begin{aligned}
            |\Gm|_\Leftrightarrow &= \sum_{\jg \in \Gm} |\jg|_e\\[2mm]
            % 2\cdot\#_{e\Lto A\in\Gm} +
            %    \#_{e\To_a \jg\in\Gm} + 3\cdot\#_{\appInfAlg{A}{e}\in\Gm}
            |A \le B|_\Leftrightarrow &= 0\\
            |e \Lto A|_\Leftrightarrow &= 2\\
            |e \To \jg|_\Leftrightarrow &= |\jg|_\Leftrightarrow + 1\\
            |\appInfAlg{A}{e}|_\Leftrightarrow &= |\jg|_\Leftrightarrow + 3
            \end{aligned}
        \end{gather*}
        \centering\framebox{$|\Gm|_{\al}$} {Existential measure.}
        \begin{gather*}
            \begin{aligned}
            |\Gm|_{\al} &= \#_{\al\in\Gm}\\
            \end{aligned}
        \end{gather*}
    \end{minipage}%
    \begin{minipage}[t]{.5\textwidth}
        \centering\framebox{$|\Gm|_\forall$, $|\jg|_\forall$, $|A|_\forall$} {Polymorphism measure.}
        \begin{gather*}
            \begin{aligned}
            |\Gm|_\Leftrightarrow &= \sum_{\jg \in \Gm} |\jg|_\forall\\[2mm]
            |A \le B|_\forall &= |A|_\forall + |B|_\forall\\
            |e \Lto A|_\forall &= |A|_\forall\\
            |e \To \jg|_\forall &= |\jg|_\forall\\
            |\appInfAlg{A}{e}|_\forall &= |A|_\forall + |\jg|_\forall\\[2mm]
            |1|_\forall &= |a|_\forall = |\al|_\forall = 1\\
            |A\to B|_\forall &= |A|_\forall + |B|_\forall\\
            |\all A|_\forall &= |A|_\forall + 1
            \end{aligned}
        \end{gather*}
        
        \centering\framebox{$|\Gm|_\to$, $|\jg|_\to$, $|A|_\to$} {Function measure.}
        \begin{gather*}
            \begin{aligned}
            |\Gm|_\to &= \sum_{\jg \in \Gm} |\jg|_\forall\\[2mm]
            |A \le B|_\forall &= |A|_\forall + |B|_\forall\\
            |e \Lto A|_\forall &= |A|_\forall\\
            |e \To \jg|_\forall &= |\jg|_\forall\\
            |\appInfAlg{A}{e}|_\forall &= |A|_\forall + |\jg|_\forall\\[2mm]
            |1|_\to &= |a|_\to = |\al|_\to = 1\\
            |A\to B|_\to &= |A|_\to + |B|_\to + 1\\
            |\all A|_\to &= |A|_\to
            % \sum_{e\Lto A\in\Gm}|A|_\to + \sum_{\appInfAlg{A}{e}\in\Gm}|A|_\to +
            %     \sum_{A\le B\in\Gm} |A|_\to + |B|_\to\\
            \end{aligned}
        \end{gather*}
    \end{minipage}
\caption{Worklist measures}
\label{fig:measures}
\end{figure}

\subsection{Broken Lemmas in DK's papers}
\label{appendix:false_lemmas}

We found false lemmas in the manual proofs of DK' two papers
\cite{dunfield2013complete} and \cite{DunfieldIndexed}.
\begin{itemize}
    \item
        In the first paper, Lemma 29 on page 9 of its appendix says:
        \begin{lemma}[Parallel Admissibility of~\cite{dunfield2013complete}]~\\
        If $\Gamma_L \longrightarrow \Delta_L$ and
        $\Gamma_L, \Gamma_R \longrightarrow \Delta_L, \Delta_R$ then:
        \begin{enumerate}
            \item $\Gamma_L,\al,\Gamma_R \longrightarrow \Delta_L, \al, \Delta_R$
            \item If $\Delta_L \vdash \tau'$ then
                $\Gamma_L,\al,\Gamma_R \longrightarrow \Delta_L, \al = \tau', \Delta_R$.
            \item If $\Gamma_L \vdash \tau$ and $\Delta_L \vdash \tau'$ and
                $[\Delta_L]\tau = [\Delta_L]\tau'$, then
                $\Gamma_L, \al = \tau, \Gamma_R \longrightarrow \Delta_L, \al = \tau', \Delta_R$.
        \end{enumerate}
        \end{lemma}

        We give a counter-example to this lemma:\\
        Pick $\Gm_L = \nil, \Gm_R = \bt, \Delta_L = \bt, \Delta_R = \nil$, then both conditions
        $\nil\longrightarrow \bt$ and $\bt\longrightarrow \bt$ hold, but the first conclusion
        $\al,\bt \longrightarrow \bt,\al$ does not hold.
    \item
        In the second paper, as an extended work to the first paper, Lemma 26 on page 22 of its appendix says:
        \begin{lemma}[Parallel Admissibility of~\cite{DunfieldIndexed}]~\\
        If $\Gamma_L \longrightarrow \Delta_L$ and
        $\Gamma_L, \Gamma_R \longrightarrow \Delta_L, \Delta_R$ then:
        \begin{enumerate}
            \item $\Gamma_L,\al:\kappa,\Gamma_R \longrightarrow \Delta_L, \al:\kappa, \Delta_R$
            \item If $\Delta_L \vdash \tau' : \kappa$ then
                $\Gamma_L,\al:\kappa,\Gamma_R \longrightarrow \Delta_L, \al:\kappa = \tau', \Delta_R$.
            \item If $\Gamma_L \vdash \tau : \kappa$ and $\Delta_L \vdash \tau' types$ and
                $[\Delta_L]\tau = [\Delta_L]\tau'$, then
                $\Gamma_L, \al:\kappa = \tau, \Gamma_R \longrightarrow \Delta_L, \al:\kappa = \tau', \Delta_R$.
        \end{enumerate}
        \end{lemma}
        
        A similar counter-example is given:\\
        Pick $\Gm_L = \nil, \Gm_R = \bt:\star, \Delta_L = \bt:\star, \Delta_R = \nil$, then both conditions
        $\nil\longrightarrow \bt:\star$ and $\bt:\star\longrightarrow \bt:\star$ hold, but the first conclusion
        $\al:\kappa,\bt:\star \longrightarrow \bt:\star,\al:\kappa$ does not hold.
\end{itemize}

\subsection{A Fix to DK's Subsumption Lemma}
\label{appendix:subsumption}

Let's first recall the lemma:
\begin{lemma}[Subsumption]
Given $\Psi' \le \Psi$:
\begin{enumerate}
    \item If $\Psi \vdash e \Lto A$ and $\Psi \vdash A \le A'$ then $\Psi' \vdash e \Lto A'$;
    \item If $\Psi \vdash e \To B$ then there exists
        $B'$ s.t. $\Psi \vdash B' \le B$ and $\Psi' \vdash e \To B'$;
    \item If $\Psi \vdash \appInf{A}{e}{C}$ and $\Psi \vdash A' \le A$,
        then there exists $C'$ s.t. $\Psi \vdash C' \le C$ and $\Psi' \vdash \appInf{A'}{e}{C'}$.
\end{enumerate}
\end{lemma}

\paragraph{Problems in DK's Manual Proof}
We tried to follow DK's manual proof for this lemma.
While being written in a clear format and providing enough details,
the proof is not fully accepted by the proof assistant.
Specifically, the first two applications of induction hypotheses on page 22 are not accepted.
Either of them seems to use a slightly different ``induction hypothesis'' than the true one.

\paragraph{A Fix to the Proof}
We make use of our worklist measures for the induction.
Recall that $|e|_e$ measures term size;
and the judgment measure counts checking as 2, inference as 1 and application inference as 3;
and $|A|_\forall$ counts the number of $\forall$'s in a type.

The proof is by a nested mutual induction on the lexicographical order of the measures
$$\langle |e|_e, |\cdot|_\Leftrightarrow, |A|_\forall + |A'|_\forall \rangle,$$
where the second measure is 2 for checking (1), 1 for inference (2) and 3 for application inference (3);
and the third measure does not apply to case (2) since no $A$ is used.

All but two cases can be finished easily following the declarative typing derivation,
and the proof shares a similar structure to DK's.
One tricky case related to Rule $\mathtt{Decl\forall I}$ indicates that $A$ has the shape $\all A_0$,
thus the subtyping relation derives from either $\mathtt{{\le}\forall L}$ or $\mathtt{{\le}\forall R}$.
For each of the case, the third measure $|A|_\forall + |A'|_\forall$ decreases
(the $\mathtt{{\le}\forall L}$ case requires a type substitution lemma obtaining
$\Psi \vdash e \Lto [\tau/a]A_0$ from the typing derivation).

Another tricky case is $\mathtt{Decl{\to}I}$.
When the subtyping relation is derived from $\mathtt{{\le}{\to}}$,
a simple application of induction hypothesis finishes the proof.
When the subtyping relation is derived from $\mathtt{{\le}\forall R}$,
$|A'|_\forall$ decreases, and thus the induction hypothesis finishes this case.

\subsection{Translation Table for the Proof Scripts}
In the proof scripts, we use textual relational definitions
rather than the symbolic ones used in the paper.
The mapping should be helpful for anyone who reads the script.

\afterpage{%
\clearpage% Flush earlier floats (otherwise order might not be correct)
\begin{landscape}% Landscape page

\begin{figure}[ht]
\centering
\begin{tabular}{c|c|c|c}
Symbol & Abella textual relation & File (.thm) & Description\\\hline
\hline
$A$    & {\tt ty, wft, wfta} & typing &
    The "type" type, decl./alg. well-formedness\\\hline
$e$    & {\tt tm, wftm} & typing &
    The "term" type, (alg.) well-formedness\\\hline
$\Psi, \Gm$ & {\tt olist, wfj} & typing &
    {\tt olist} is Abella's built-in list type, i.e. {\tt [o]}\\\hline
$\jg$  & {\tt judgment, wfjg} & typing &
    Judgment chain and its well-formedness\\\hline
\hline
$\Gm\rto\Gm'$ & {\tt reduction} & smallStep &
    Algorithmic reduction: paper version\\\hline
$\Gm\redto\nil$ & {\tt judge} & typing &
    ({\tt judge} $\Gm$): a success reduction on $\Gm$\\\hline
\hline
$\Gm \sto \Om$ & {\tt tex} & trans &
    "transfer existential variables"\\\hline
$\Om\redto\nil$ & {\tt dc, dcl} & trans, dcl &
    Declarative chain representation\\\hline
\hline
$\Psi'\le\Psi$ & {\tt esub} & declTyping & Declarative context subtyping\\\hline
\hline
$\Gm\jExt\Gm'$ & {\tt jExt} & inst\_decidable & "Judgment extension"\\\hline
$\Gm_i$ & {\tt iexp} & inst\_decidable & instantiation judgments ($\Gm \vdash_{wf} \Gm_i$)\\\hline
$\Gm_\le$ & {\tt subExp} & inst\_decidable & subtyping judgments\\\hline
\hline
$|\cdot|_e$ & {\tt tmSize, tmSizeJ, tmSizel} & declTyping, decidability &
    Size measure for term, judgment chain or context\\\hline
$|\cdot|_\Leftrightarrow$ & {\tt m\_judgeJ, m\_judge} & decidability &
    Judgment measure for judgment chain or context\\\hline
$|\cdot|_{\al}$ & {\tt nVar} & inst\_decidable &
    Existential measure for context\\\hline
$|\cdot|_\forall$ & {\tt order, orderJ, orderl} & order, inst\_decidable &
    Polymorphism measure for type, judgment chain or context\\\hline
$|\cdot|_\to$ & {\tt depth, depthJ, depthl} & order, decidability &
    Function measure for type, judgment chain or context\\\hline
\end{tabular}
\caption{Translation Table for the Proof Scripts}
\label{fig:translation_table}
\end{figure}

\end{landscape}
\clearpage% Flush page
}


\end{document}
